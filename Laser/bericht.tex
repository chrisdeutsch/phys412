% PAKETE UND DOKUMENTKONFIGURATION
\documentclass[11pt, a4paper]{article}

% Encoding für Umlaute
\usepackage[utf8]{inputenc}
\usepackage[T1]{fontenc}

% Silbentrennung
\usepackage[ngerman]{babel}

% erweiterte Matheumgebungen und Formelnummer mit Sectionnummer
\usepackage{amsmath}
\numberwithin{equation}{section}

% Braket Notation
\usepackage{braket}

% zusätzliche mathematische Schriftarten
\usepackage{amsfonts}

% verschiedene mathematische Symbole
\usepackage{amssymb}

% Einheiten setzen z.B. \SI{10}{\kilo\gram\meter\per\second\squared}
% Fehler: \SI{10 +- 0,2e-4}{\metre}
\usepackage{siunitx}
\sisetup{
  output-decimal-marker={,},
  separate-uncertainty
}

% Einheitendefinitionen
\DeclareSIUnit{\dBm}{dBm}

% Operatordefinitionen
\DeclareMathOperator{\erf}{erf}

% Randbreiten
\usepackage[left=3.5cm,right=3.5cm,top=3cm,bottom=3cm,twoside]{geometry}

% Bilder einfügen
\usepackage{graphicx}

% Verweise innerhalb des Dokuments
\usepackage{hyperref}
\hypersetup{
	colorlinks = true,
	allcolors = {black}
}

% bessere Tabellenlayouts
\usepackage{booktabs}
\usepackage{multirow}

% Seitenlayout (Kopfzeile)
\usepackage{fancyhdr}

% Float Barriers
\usepackage{placeins}

% Pakete für gedrehte Subfigures
\usepackage{caption}
\usepackage{subcaption}
\usepackage{rotating}

% Caption-Setup
\captionsetup{font={small}}
\renewcommand{\thefigure}{\thesection.\arabic{figure}}
\renewcommand{\thesubfigure}{\alph{subfigure}}
\renewcommand{\thetable}{\thesection.\arabic{table}}
\renewcommand{\thesubtable}{\alph{subtable}}

% Manuelle Silbentrennung
\hyphenation{Re-so-na-tor Mo-den-ab-stand}

% Tiefe des Inhaltsverzeichnisses (Level: 1 sections, 2 subsections,
% 3 subsubsections)
\setcounter{tocdepth}{3}

% FANCYHDR SETUP
\pagestyle{fancy}
\fancyhead[EL,OR]{\thepage}
\fancyhead[ER]{\leftmark}
\fancyhead[OL]{\rightmark}

\renewcommand{\sectionmark}[1]{
\markboth{\thesection{} #1}{\thesection{} #1}
}
\renewcommand{\subsectionmark}[1]{
\markright{\thesubsection{} #1}
}

% DOKUMENTINFORMATIONEN
\title{P442 \\ Laser}

\author{Christopher Deutsch\footnote{christopher.deutsch@uni-bonn.de} \and Christian Bespin\footnote{christian.bespin@uni-bonn.de}}

\date{\today}

\begin{document}

\begin{titlepage}

\maketitle

% DURCHFÜHRUNGSDATUM UND TUTOR
\begin{center}
\begin{tabular}{l r}
Durchführung: & 15./16. Dezember 2014 \\
Gruppe: & $\alpha$ 2 \\
Tutor: & Tobias Macha
\end{tabular}
\end{center}

% ZUSAMMENFASSUNG
\begin{abstract}
\noindent

\end{abstract}

\end{titlepage}

% INHALTSVERZEICHNIS
\tableofcontents
% Neue Seite nach TOC
\newpage

% INHALT VERSUCHSPROTOKOLL

\section{Einführung}

In diesem Praktikumsversuch wird ein Helium-Neon-Laser (HeNe-Laser) als Experimentierlaser aufgebaut und seine Eigenschaften wie beispielsweise Strahlprofil, Modenabstände und Polarisation für verschiedene Resonatorlängen vermessen.
Der HeNe-Laser wurde erstmals 1960 von Javan, Bennett und Herriott \cite{javan} entwickelt und stellt den ersten Laser dar, der kontinuierliches Laserlicht erzeugte.

\section{Theorie}

\subsection{Aktives Medium}
\subsubsection{Übergänge und Einsteinkoeffizienten}
\begin{figure}[h]
	\centering
	\includegraphics[width=0.5\textwidth]{./figures/einsteinkoeffizienten.pdf}
	\caption{Erklärung der Einsteinkoeffizienten zwischen zwei Energieniveaus mit Besetzungszahlen $N$ und Entartungsgrad $g$.}
	\label{fig:einsteinkoeff}
\end{figure}
\begin{itemize}
	\item \textbf{Absorption:} Durch Absorption eines Photons der Energie $E_2 - E_1 = \hbar \omega_{12}$ aus dem Lichtfeld mit der spektralen Energiedichte $\rho(\omega)$ kann ein Elektron vom Zustand $\ket{1}$ in den Zustand $\ket{2}$ übergehen.
	Mathematisch beschreibt der Einsteinkoeffizient $B_{12}$ die Rate der Besetzungsänderung des Zustandes mit der Energie $E_2$:
	\begin{align}
		\frac{\mathrm{d} N_2}{\mathrm{d} t} = N_1 B_{12} \rho\left( \omega_{12} \right)
	\end{align}
	
	\item \textbf{stimulierte Emission:} Ein Photon des Lichtfeldes der Energie $\hbar \omega_{12}$ induziert einen Übergang vom Zustand $\ket{2}$ nach $\ket{1}$, unter Emission eines zum Photon des Lichtfeldes identischen Photons.
	Die mathematische Beschreibung der Rate erfolgt über den Koeffizienten $B_{21}$:
	\begin{align}
		\frac{\mathrm{d} N_2}{\mathrm{d} t} = N_2 B_{21} \rho\left( \omega_{12} \right)
	\end{align}
	
	\item \textbf{spontante Emission:}
	Unabhängig vom äußeren Lichtfeld zerfällt ein angeregter Zustand unter Emission eines Photons in den Grundzustand.
	Die Zerfallsrate wird dabei durch den Koeffizienten $A_{21}$ beschrieben:
	\begin{align}
		\frac{\mathrm{d} N_2}{\mathrm{d} t} = - N_2 A_{21}
	\end{align}
	Aus dieser Gleichung folgt direkt der reziproke Zusammenhang mit der Lebensdauer $\tau$ des angeregten Zustandes:
	\begin{align}
		\tau = \frac{1}{A_{21}}
	\end{align}
\end{itemize}
Weiterhin wird der Zusammenhang der Einsteinkoeffizienten durch:
\begin{align}
	g_1 B_{12} = g_2 B_{21}  \qquad A_{21} = \frac{\hbar \omega_{12}^3}{\pi^2 c^3}
\end{align}
erklärt.

\subsubsection{Niveauschemata aktiver Medien}
Aus den Ratengleichungen des vorigen Abschnitts folgt für das Zweiniveausystem:
\begin{align}
	\frac{\mathrm{d}(N_1 - N_2)}{\mathrm{d} t} = N_2 A_{21} + B_{21} \rho(\omega_{12}) \left( N_2 - \frac{g_2}{g_1} N_1 \right) \stackrel{\mathrm{Gleichgewicht}}{=} 0 \text{,}
\end{align}
wobei man den Besetzungsunterschied definiert als:
\begin{align}
	\sigma = N_2 - \frac{g_2}{g_1} N_1 \text{.}
\end{align}
Um eine Verstärkung des Lichtfeldes durch stimulierte Emission zu erreichen, ist eine Besetzungsinversion $\sigma > 0$ nötig.
Aufgrund der spontanten Emission kann in einem Zweiniveausystem im Gleichgewicht niemals eine nichtnegativer Besetzungsunterschied erreicht werden, weshalb für aktive Medien keine Zweiniveausysteme verwendet werden.
In Abbildung \ref{fig:niveauschema} wurde beispielhaft das Niveauschema für Lasermedien mit drei und vier Niveaus aufgetragen.
\begin{figure}[h]
	\centering
	\includegraphics[width=0.75\textwidth]{./figures/niveausystem_3_4.pdf}
	\caption{Niveau-Schemata gepumpter aktiver Medien}
	\label{fig:niveauschema}
\end{figure}

\subsection{Gaußstrahlen}
\label{gaussstrahlen}
Zur Beschreibung von transversal begrenzten Strahlen in Resonatoren oder im freien Raum sind ebene Wellen aufgrund ihrer unendlichen Ausdehnung ungeeignet.
Stattdessen werden Gaußstrahlen zur Beschreibung solcher Wellen verwendet.
Diese sind Lösungen der paraxialen Helmholtzgleichung und keine exakte Lösung der Wellengleichung \cite{linden}.
Die axiale und transversale Intensitätsverteilung solcher Strahlen wird beschrieben durch:
\begin{align}
	I(z,r) = I_0 \left( \frac{w_0}{w(z)} \right)^2 \exp\left(- \frac{2 r^2}{w^2(z)} \right) \text{,}
\end{align}
wobei der Strahlradius $w(z)$ definiert ist, als der Abstand $r$ von der Ausbreitungsachse, bei der die Intensität um den Faktor $1/ \mathrm{e}^2$ abgefallen ist.
Der axiale Verlauf des Strahlprofils lässt sich dann beschreiben durch:
\begin{align}
	w(z) = w_0 \sqrt{1 + \left( \frac{z}{z_0} \right)^2} \text{,}
	\label{eq:gauss_axialprofil}
\end{align}
wobei die Strahltaille $w_0$ (minimaler Strahlradius) gegeben ist durch:
\begin{align}
	w_0 = \sqrt{\frac{\lambda z_0}{\pi}}
	\label{eq:rayleigh_laenge}
\end{align}
und die Rayleigh-Länge $z_0$ definiert ist, als der axiale Abstand von der Strahltaille, bei der sich die Intensität halbiert:
\begin{align}
	&I(z = \pm z_0, r =0) = \frac{I_0}{2}
\end{align}
Ist die Wellenlänge $\lambda$ des Strahls bekannt, so seine Intensitätsverteilung eindeutig durch Angabe der Strahltaille $w_0$ oder der Rayleigh-Länge $z_0$ definiert.

Krümmungsradius:
\begin{align}
	R(z) = z \left( 1 + \left( \frac{z_0}{z} \right)^2 \right)
\end{align}

\subsection{Resonatoren}
Resonatoren sind Anordnungen optischer Elemente in denen ein Lichtstrahl auf einem geschlossenen Pfad zirkulieren kann.
In diesem Versuch wird ein linearer Resonator verwendet, was bedeutet, dass jeder Lichstrahl in sich selbst reflektiert wird und dadurch bei kontinuierlichem Betrieb eine stehende Welle im Resonator ausbildet.
Resonatoren zeichnen sich dadurch aus, dass in ihnen nur bestimmte Frequenzen\footnote{Diese Resonanzen haben endliche aber von Null verschiedene Linienbreite.} konstruktiv ausbreiten können.
Beispiele für die Verwendung von Resonatoren sind:
\begin{itemize}
	\item \textbf{Laserresonatoren:} Durch Hinzufügen eines aktiven Mediums in den Resonatorzwischenraum kann es zu einer Verstärkung des zirkulierenden Lichts durch stimulierte Emission kommen.
	Dadurch kann ein nahezu monochromatischer kohärenter Laserstrahl mit hoher Intensität im Resonator erzeugt werden.	
	
	\item \textbf{Filter:} Aufgrund der Resonanz bei bestimmten Frequenzen eignen sich Resonatoren auch als Filter.
	In diesem Versuch wird aus diesem Grund ein durchstimmbares\footnote{Durch Längenänderung des Resonators kann die resonante Frequenz variiert werden.} Fabry-Pérot-Interferometer aus zwei sphärischen Spiegeln in konfokalem Aufbau verwendet, um in Abschnitt \ref{ssec:spektrumanalysator} das Frequenzspektrum des aufgebauten Lasers zu untersuchen.
\end{itemize}

\subsubsection{Moden in Resonatoren}
Unter Resonatormoden versteht man Verteilungen des elektrischen Feldes, die sich nach einem Umlauf im Resonator in ihrer Form reproduzieren, wobei es zu Verlusten oder Verstärkungen kommen kann.
Im einfachsten Fall kann die \textbf{axiale Modenstruktur} für $\mathrm{TEM}_{q00}$-Fundamentalmode durch Gaußstrahlen aus Abschnitt \ref{gaussstrahlen} beschrieben werden.
Dabei ist die Bedingung für das ausbilden einer stehenden Welle im Resonator gegeben durch:
\begin{align}
	\nu_q = q \cdot \frac{c}{2 L} \qquad q \in \mathbb{N} \setminus \{0\}
\end{align}
Der freie Spektralbereich $\Delta \nu_\mathrm{FSR}$, das heißt der Frequenzabstand der Axialmoden, kann direkt abgelesen werden:
\begin{align}
	\Delta \nu_\mathrm{FSR} = \frac{c}{2 L}
	\label{eq:modenabstand}
\end{align}
Darüber hinaus muss sich auch die transversale Verteilung der Felder reproduzieren.
Dadurch kommt es zu einer \textbf{transversalen Modenstruktur}, welche durch Hermite-Gauß'sche Funktionen beschrieben wird.
In Abbildung \ref{fig:transversalmoden} wurde die qualitative Intensitätsverteilung im Querschnitt einer $\mathrm{TEM}_{nm}$-Mode dargestellt.
\begin{figure}[h]
	\centering
	\includegraphics[width=0.8\textwidth]{./figures/transversalmoden.png}
	\caption{Qualitative Darstellung der Hermite-Gauß'schen $\mathrm{TEM}_{nm}$-Moden}
	\label{fig:transversalmoden}
\end{figure}
Im Allgemeinen ist die Resonanzfrequenz eines Resonators abhängig von der propagierenden Transversalmode, sodass die Resonanzfrequenz gegeben ist durch \cite{siegman}:
\begin{align}
	\nu_{qnm} = q \cdot \frac{c}{2 L} + \left( n + m + 1 \right) \cdot \delta \nu_\mathrm{t} \qquad (\mathrm{TEM}_{qnm}\text{-Mode})\text{,}
\end{align}
wobei $\delta \nu_\mathrm{t}$ den transversalen Modenabstand beschreibt und abhängig von der Resonatorgeometrie ist.

\subsubsection{Stabilität}
Betrachtet man einen Resonator aus zwei sphärischen Spiegeln mit Radien $R_i$ und Abstand $L$ und nimmt an, dass deren transversale Ausdehnung groß genug ist, um Beugungsverluste an den Rändern der Spiegel zu vernachlässigen, so charakterisiert man einen stabilen Resonator dadurch, dass dieser einen Gaußstrahl mit fester Strahltaille $w_0$ an beiden Spiegeln in sich selbst reflektiert \cite{siegman}.
Dies ist genau dann der Fall, wenn der Krümmungsradius $R(z)$ des Gaußstrahls\footnote{Das Koordinatensystem wird so gewählt, dass $z$ den Abstand von der Strahltaille angibt.} an der jeweiligen Spiegelposition dem des Spiegels entspricht.
Unter Beachtung der Vorzeichen der Krümmungsradien\footnote{Aus dem Resonator betrachtet, entspricht ein positiver Krümmungsradius einem konkaven Spiegel.} erhält man so:
\begin{align}
	R_1 = - z_1 - \frac{z_0^2}{z_1}  \qquad
	R_2 = z_2 + \frac{z_0^2}{z_2} \text{.}
\end{align}
Wie in \cite{siegman} gezeigt, erhält man unter der Nebenbedingung $L = z_2 - z_1$ nur dann reelle und endliche Strahlparameter, wenn die Stabilitätsbedingung:
\begin{align}
	0 \le g_1 g_2 \le 1
\end{align}
erfüllt ist.
Die $g$-Faktoren sind dabei definiert als:
\begin{align}
	g_i = 1 - \frac{L}{R_i} \text{.}
\end{align}

\subsubsection{Halbsymmetrischer Resonator}
Da in diesem Versuch ein halbsymmetrischer Resonator ($R_1 = \infty$, $g_1 \rightarrow 1$ und $R_2 = R$, $g_2 \rightarrow g$) verwendet wird, soll hier in Anbetracht der Auswertung einige wichtige Eigenschaften aufgeführt werden.
Da der Planspiegel einen unendlichen Krümmungsradius hat, liegt die Strahltaille an der Position des ebenen Spiegels, da ein Gaußstrahl nur an der Strahltaille einen unendlichen Krümmungsradius aufweist.
Dann ist die Weite der Taille gegeben durch \cite{siegman}:
\begin{align}
	w_0^2 = w_1^2 = \frac{L \lambda}{\pi} \sqrt{\frac{g}{1 - g}}
	\label{eq:strahltaille_halbsym}
\end{align}
und gemäß Abschnitt \ref{gaussstrahlen} ist das Strahlprofil im Resonator somit eindeutig festgelegt.

\section{Durchführung und Auswertung}
Die ausführliche Durchführung ist der Versuchsanleitung \cite{anleitung} zu entnehmen.
Sollten Abweichungen bei der Durchführung auftreten, so werden diese im jeweiligen Unterkapitel dargestellt.

\subsection{Aufbau des Helium-Neon Experimentierlasers}
Nach der Justierung des Lasers wurde für jede der vier untersuchten Resonatorlängen $L$ die Länge des Resonators mit einem Maßband vermessen, wobei der Fehler der Längenmessung auf $\Delta L = \SI{0.4}{\centi\metre}$ abgeschätzt wird.
Diese Abschätzung begründet sich darin, dass die exakte Lage der Spiegel in der Fassungen nicht bekannt ist.
Anschließend bestimmen wir die mittlere Spannung an der Photodiode $U_\mathrm{PD}$ mithilfe der \textit{Measure}-Funktion des Oszilloskops, wobei deren Fehler anhand der beobachteten Schwankungen auf $\Delta U_\mathrm{PD} = \SI{0.2}{\milli\volt}$ geschätzt wird.
Unter der Annahme, dass die Photodiode im linearen Bereich arbeitet, was aufgrund des \SI{50}{\ohm}-Abschlusswiderstandes am Oszilloskop gewährleistet ist, kann mithilfe des Umrechnungsfaktors $\SI{1}{\milli\watt} / \SI{17.5}{\milli\volt}$ aus der Versuchsbeschreibung \cite{anleitung} eine Umrechnung in eine Leistung erfolgen:
\begin{align}
	P = \frac{\SI{1}{\milli\watt}}{\SI{17.5}{\milli\volt}} \cdot U_\mathrm{PD}
	\label{eq:umrechnung_watt}
\end{align}
Dabei erfolgt die Fehlerfortpflanzung mit demselben Umrechnungsfaktor.
Die gemessenen Größen und die Umrechnung wurde in Tabelle \ref{tab:intensitaeten} durchgeführt.
\begin{table}[h]
	\centering
	\begin{tabular}{SSSSSS}
\toprule
{$L$ / \si{\centi\metre}} & {$\Delta L$ / \si{\centi\metre}} & {$U_\mathrm{PD}$ / \si{\milli\volt}} & {$\Delta U_\mathrm{PD}$ / \si{\milli\volt}} & {$P$ / \si{\milli\watt}} & {$\Delta P$ / \si{\milli\watt}} \\
\midrule
50.0 & 0.4 & 66.1 & 0.2 & 3.78 & 0.02 \\
69.0 & 0.4 & 60.4 & 0.2 & 3.45 & 0.02 \\
79.5 & 0.4 & 58.0 & 0.2 & 3.31 & 0.02 \\
89.2 & 0.4 & 45.0 & 0.2 & 2.57 & 0.02 \\
\bottomrule
\end{tabular}
	\caption{Gemessene Leistungen bei verschiedenen Resonatorlängen. Die Photospannungen wurden um den Untergrund bei blockiertem Laserstrahl korregiert.}
	\label{tab:intensitaeten}
\end{table}
Auffällig ist die Abnahme der Intensität mit der Resonatorlänge, welche dadurch erklärt werden kann, dass das vom Strahl eingenommene Volumen in der Entladungsröhre mit steigender Länge $L$ abnimmt.
Das verringerte Volumen führt dazu, dass die Verstärkung des aktiven Mediums abnimmt, da dieses in dem Volumen gesättigt wird.
Darüber hinaus ist auch der Weg des Laserstrahls durch Luft länger, wodurch es zu höheren Verlusten aufgrund von Streuung kommt.

\subsection{Bestimmung der Wellenlänge des Lasers mit einem Reflexionsgitter}
\subsubsection{Durchführung}
\begin{figure}[h]
	\centering
	\includegraphics[width=0.9\textwidth]{./figures/wellenlaenge_lineal.pdf}
	\caption{Wellenlängenbestimmung nach Schawlow \cite{schawlow} und die im Versuch gemessenen Längen $d$}
	\label{fig:schawlow}
\end{figure}
Zur Bestimmung der Wellenlänge des Helium-Neon Lasers nach Schawlow \cite{schawlow} wird ein Stahlmaßstab mit eingravierter Skaleneinteilung als Reflexionsgitter verwendet.
Dazu wird der Laserstrahl in flachem Winkel auf die Millimeter-Skala des Maßstabes geleitet, sodass ein scharfes Interferenzmuster auf dem Whiteboard in einer Entfernung (von der Mitte der beleuchteten Stelle am Lineal) von $x_0 = \SI{252 +- 2}{\centi\metre}$ entsteht.
Anschließend werden die Maxima des Musters markiert, wobei darauf geachtet werden muss, dass bei der nullten Beugungsordnung begonnen wird, um negative Beugungsordnungen zu vermeiden.
Die nullte Beugungsordnung kann daran erkannt werden, dass sie maximale Intensität aufweist.
Nachdem ausreichend viele Ordnungen markiert wurden, wird das Lineal aus dem Strahlengang entfernt und der Auftreffpunkt des Laserstrahls auf dem Whiteboard markiert.
Schließlich wird mit einem Maßband die Position $d$ aller Beugungsmaxima relativ zum Auftreffpunkt des Lasers ohne Lineal gemessen (vergleiche dazu auch Abbildung \ref{fig:schawlow}).
Die Schärfe des Interferenzmusters ließ es dabei zu, die Längen $d$ mit einem abgeschätzten Fehler von $\Delta d = \SI{0.4}{\centi\metre}$ zu bestimmen, wobei die Hauptfehlerquelle in der Bestimmung des Intensitätsmaximums liegt und nicht in der Längenmessung mit dem Maßband.

\subsubsection{Auswertung}
In \cite{schawlow} wird gezeigt, dass für die Gittergleichung in Kleinwinkelnäherung ($\frac{y_n}{x_0} \ll 1$) gegeben ist:
\begin{align}
	n \lambda = \frac{1}{2} g \left( \frac{y_n^2 - y_0^2}{x_0^2} \right) \text{,}
	\label{eq:gittergleichung_schawlow}
\end{align}
wobei $g$ die Gitterkonstante ist und ansonsten die Notation aus Abbildung \ref{fig:schawlow} verwendet wird.
Um aus den gemessenen Werten $d_n$ die Wellenlänge zu bestimmen, muss zunächst die Position $y_0$ des reflektierten Strahls auf dem Schirm berechnet werden, welche gemäß Abbildung \ref{fig:schawlow} gegeben ist durch:
\begin{align}
	y_0 = \frac{d_0}{2} \text{,}
\end{align}
sodass mit dem gemessenen Wert $d_0 = \SI{19.7 +- 0.4}{\centi\metre}$ folgt:
\begin{align}
	y_0 = \SI{9.85 +- 0.20}{\centi\metre} \text{.}
\end{align}
So kann die Position $y_n$ des $n$-ten Interferenzmaximums berechnet werden durch:
\begin{align}
	y_n &= d_n - y_0 \\
	\Delta y_n &= \sqrt{\Delta d_n^2 + \Delta y_0^2}
\end{align}
und anschließend mit der genäherten Gittergleichung \ref{eq:gittergleichung_schawlow} die Wellenlänge des Lasers für die einzelnen Maxima berechnet werden, wobei für die Gitterkonstante $g = \SI{1}{\milli\metre}$ gilt, da die Beugung an der Millimeter-Skala des Maßstabes stattfindet. 
Der Fehler der berechneten Wellenlänge $\Delta \lambda$ ist gegeben durch Gauß'sche Fehlerfortpflanzung unter der Vernachlässigung des Fehlers in der Gitterkonstanten $g$:
\begin{align}
	\Delta \lambda = \frac{g}{2 n} \sqrt{\left( \frac{2 y_n}{x_0^2} \right)^2 \cdot \Delta y_n^2 + \left( \frac{2 y_0}{x_0^2} \right)^2 \cdot \Delta y_0^2 + \left(\frac{2\left(y_n^2 - y_0^2 \right)}{x_0^3}\right)^2 \cdot \Delta x_0^2} \text{.}
\end{align}
Die gemessenen Abstände $d$ sowie die berechneten Werte wurden in Tabelle \ref{tab:wellenlaengen_berechnung} aufgetragen.
\begin{table}[h]
	\centering
	\begin{tabular}{SSSSSSS}
\toprule
{$n$} & {$d_n$ / \si{\centi\metre}} & {$\Delta d_n$ / \si{\centi\metre}} & {$y_n$ / \si{\centi\metre}} & {$\Delta y_n$ / \si{\centi\metre}} & {$\lambda$ / \si{\nano\metre}} & {$\Delta \lambda$ / \si{\nano\metre}} \\
\midrule
1  & 23.2 & 0.4 & 13.35 & 0.45 & 639 & 100 \\
2  & 26.0 & 0.4 & 16.15 & 0.45 & 645 & 60  \\
3  & 28.1 & 0.4 & 18.25 & 0.45 & 619 & 46  \\
4  & 30.2 & 0.4 & 20.35 & 0.45 & 624 & 38  \\
5  & 32.0 & 0.4 & 22.15 & 0.45 & 620 & 34  \\
6  & 34.0 & 0.4 & 24.15 & 0.45 & 638 & 31  \\
7  & 35.4 & 0.4 & 25.55 & 0.45 & 625 & 28  \\
8  & 37.1 & 0.4 & 27.25 & 0.45 & 635 & 27  \\
9  & 38.5 & 0.4 & 28.65 & 0.45 & 633 & 25  \\
10 & 39.7 & 0.4 & 29.85 & 0.45 & 625 & 24  \\
11 & 41.3 & 0.4 & 31.45 & 0.45 & 639 & 23  \\
12 & 42.4 & 0.4 & 32.55 & 0.45 & 632 & 22  \\
13 & 43.5 & 0.4 & 33.65 & 0.45 & 627 & 21  \\
14 & 44.8 & 0.4 & 34.95 & 0.45 & 632 & 21  \\
15 & 46.0 & 0.4 & 36.15 & 0.45 & 635 & 20  \\
16 & 47.1 & 0.4 & 37.25 & 0.45 & 635 & 20  \\
17 & 47.9 & 0.4 & 38.05 & 0.45 & 626 & 19  \\
18 & 49.2 & 0.4 & 39.35 & 0.45 & 635 & 19  \\
19 & 50.4 & 0.4 & 40.55 & 0.45 & 641 & 19  \\
\bottomrule
\end{tabular}
	\caption{Messdaten und Berechnung zur Wellenlängenbestimmung}
	\label{tab:wellenlaengen_berechnung}
\end{table}
Um aus den berechneten Wellenlängen für die einzelnen Maxima eine Wellenlänge zu bestimmen, wird der varianzgewichtete Mittelwert:
\begin{align}
	\bar{\lambda} = \frac{\sum_i \frac{\lambda_i}{\Delta \lambda_i^2}}{\sum_i \frac{1}{\Delta \lambda_i^2}}
\end{align}
verwendet, wobei dessen Fehler gegeben ist durch:
\begin{align}
\Delta \bar{\lambda} = \frac{1}{\sqrt{\sum_i \frac{1}{\Delta \lambda_i^2}}} \text{.}
\end{align}
Mit den Werten aus Tabelle \ref{tab:wellenlaengen_berechnung} folgt für die varianzgewichtete mittlere Wellenlänge:
\begin{align}
	\bar{\lambda} = \SI{632.3 +- 5.6}{\nano\metre} \text{.}
\end{align}
Der Vergleich mit dem Literaturwert \cite{NISTSpectra} für den ($\mathrm{5s} \rightarrow \mathrm{3p}$)-Übergang von Neon:
\begin{align}
	\lambda = \SI{632.81646}{\nano\metre}
	\label{eq:HeNe_wellenlaenge}
\end{align}
liefert eine gute Übereinstimmung innerhalb des statistischen Fehlers.

\subsubsection{Vergleich mit der exakten Rechnung}
Im Folgenden soll der Einfluss der Näherung bei der Berechnung der Wellenlänge untersucht werden.
Dazu betrachtet man die Gittergleichung mit dem Einfallswinkel $\alpha$ und den Ausfallswinkel der $n$-ten Interferenzordnung $\beta_n$, welche jeweils zur Gitterebene gemessen werden (vergleiche Abbildung \ref{fig:schawlow}):
\begin{align}
	n \lambda = g \left( \cos(\alpha) - \cos(\beta_n) \right) \text{.}
	\label{eq:wellenlaenge_exakt}
\end{align}
Wegen $\alpha = \beta_0$ kann der Kosinus ausgedrückt werden als (Satz des Pythagoras):
\begin{align}
	\cos(\alpha) &= \frac{x_0}{\sqrt{x_0^2 + y_0^2}} \\
	\cos(\beta_n) &= \frac{x_0}{\sqrt{x_0^2 + y_n^2}} \text{,}
\end{align}
womit die relative Abweichung der Näherung von der exakten Rechnung aus den Gleichungen \ref{eq:gittergleichung_schawlow} und \ref{eq:wellenlaenge_exakt} folgt:
\begin{align}
	\frac{\Delta \lambda}{\lambda} = \frac{\lambda_\mathrm{KWN} - \lambda}{\lambda} = \frac{1}{2} \frac{\frac{y_n^2 - y_0^2}{x_0^2}}{ \frac{x_0}{\sqrt{x_0^2 + y_0^2}} - \frac{x_0}{\sqrt{x_0^2 + y_n^2}} } - 1
\end{align}
Diese Funktion ist in Abhängigkeit von $y_n$ in Abbildung \ref{fig:abweichung_wellenlaenge} aufgetragen, wobei für $y_0$ und $x_0$ die im Versuch gemessenen Werte verwendet werden.
\begin{figure}[h]
	\centering
	\input{./plots/abweichung_wellenlaenge.tex}
	\caption{Relative Abweichung der Näherungsformel von der exakten Berechnung bei der im Versuch verwendeten Konfiguration: $y_0 = \SI{9.85}{\centi\metre}$, $x_0 = \SI{252}{\centi\metre}$. Der blaue Pfeil markiert die größte vermessene Interferenzordnung $n = 19$.}
	\label{fig:abweichung_wellenlaenge}
\end{figure}
Man sieht, dass die Abweichung für größere Ordnungen $n$ und damit größere Abstände $y_n$ rapide zunimmt, da $\frac{y_n}{x_0} \ll 1$ nicht mehr gegeben ist.
Demnach weist die größte vermessene Ordnung $n = \num{19}$ die größte relative Abweichung auf, welche wie man in Abbildung \ref{fig:abweichung_wellenlaenge} ablesen kann $\Delta \lambda / \lambda \approx \SI{2}{\percent}$ beträgt.
Dies entspricht schon einer Überschätzung der Wellenlänge von mehr als \SI{10}{\nano\metre} und liegt damit schon in der Größenordnung des statistischen Fehlers der Einzelmessungen.
Bemerkenswert ist, dass wir die Tendenz die Wellenlänge zu überschätzen in Messdaten nicht beobachtet werden kann, weshalb davon ausgegangen werden muss, dass ein systematischer Fehler diese kompensiert.
Eine mögliche Ursache kann sein, dass das Lineal einen kleinen Winkel zur Senkrechten auf der Wand aufwies und somit eine tendenzielle Unterschätzung der berechneten Wellenlänge zur Folge hat.

\subsection{Untersuchung der Polarisation des Lasers}
\label{ssec:polarisation}
Zur Untersuchung der Polarisation des hier verwendeten Experimentierlasers wird der Intensitätsverlauf an einer Photodiode in Abhängigkeit des Winkels eines davor stehenden Linearpolarisators betrachtet.
Dieser ist in der Lage, einfallendes Licht in eine vorgegebene Richtung linear zu polarisieren.
Für den Fall, dass das einfallende Licht bereits linear polarisiert ist (wie für den hier verwendeten Laser angenommen werden kann), gilt für die hinter dem Polarisator registrierte Intensität (Gesetz von Malus):
\begin{align}
I=I_0\cdot\cos^2\alpha
\end{align}
In dem Versuch wird tatsächlich der Zusammenhang:
\begin{align}
	I = I_0 \cdot \cos^2 (\varphi - \varphi_0)
\end{align}
gemessen.
Dabei ist $\varphi$ der Polarsiatorwinkel und $\varphi_0$ beschreibt die Polarisationsrichtung. 

\subsubsection{Überschrift ganz weglassen?}
\textbf{Überschrift passt nicht so wirklich.}
Der Winkel $\varphi$ am Polarisator wurde in Schritten von \SI{10}{\degree} gedreht und jeweils die Photospannung $U_\text{PD}$ an der Photodiode in \si{\milli\volt} notiert.
In Tabelle \ref{tab:malus} sind die Messwerte bei einer Resonatorlänge von \SI{50}{\centi\metre} festgehalten worden.
Der Fehler $\Delta\varphi$ für den Winkel ergibt sich aus der Tatsache, dass der Polarisator in Schritten von \SI{2}{\degree} eingestellt werden konnte, der Fehler für die Photospannung wurde aufgrund der Schwankungen, die zu beobachten waren (es wurde mit der \emph{Measure}-Funktion des Oszilloskops gemessen), auf \SI{0.3}{\milli\volt} abgeschätzt.
Diese Messwerte werden im Anschluss mit \eqref{eq:umrechnung_watt} in Intensitäten umgerechnet, in ein Diagramm eingetragen und mit \texttt{Gnuplot} eine Kurve der Form
\begin{align}
f(x) = a\cdot\cos^2(x-b) + c
\end{align}
an die Daten angepasst.
Dabei beschreibt $a$ die Intensität $I_0$, der Parameter $b$ berücksichtigt eine eventuelle Abweichung, dass die Polarisatoreinstellung $\varphi=0$ nicht der Polarisationsrichtung des Lasers entspricht und $c$ beschreibt den auftretenden Untergrund der Spannungs- / Intensitätsmessung.
Als Ergebnis der Anpassung erhält man
\begin{align}
a&=\SI{2.24+-0.04}{\milli\watt}\\
b&=\SI{1.7+-0.6}{\degree}\\
c&=\SI{0.04+-0.03}{\milli\watt}
\end{align}
bei einem reduzierten $\chi^2$ von \num{25.4}.
\begin{figure}
	\centering
	\begin{tabular}{SSSS}
	\toprule
	{Drehwinkel $\varphi / \si{\degree}$} & {Fehler $\Delta\varphi / \si{\degree}$} & {Intensität $I / \si{\milli\volt}$} & {Fehler $\Delta I / \si{\milli\volt}$} \\
	\midrule
	0   & 1 & 39.50 & 0.3 \\
	10  & 1 & 40.00 & 0.3 \\
	20  & 1 & 38.00 & 0.3 \\
	30  & 1 & 33.80 & 0.3 \\
	40  & 1 & 27.90 & 0.3 \\
	50  & 1 & 20.20 & 0.3 \\
	60  & 1 & 13.40 & 0.3 \\
	70  & 1 & 6.45  & 0.3 \\
	80  & 1 & 2.62  & 0.3 \\
	90  & 1 & 0.60  & 0.3 \\
	100 & 1 & 1.10  & 0.3 \\
	110 & 1 & 3.50  & 0.3 \\
	120 & 1 & 8.60  & 0.3 \\
	130 & 1 & 14.70 & 0.3 \\
	140 & 1 & 21.20 & 0.3 \\
	150 & 1 & 27.70 & 0.3 \\
	160 & 1 & 33.40 & 0.3 \\
	170 & 1 & 37.00 & 0.3 \\
	180 & 1 & 38.50 & 0.3 \\
	190 & 1 & 37.50 & 0.3 \\
	200 & 1 & 34.70 & 0.3 \\
	210 & 1 & 29.60 & 0.3 \\
	220 & 1 & 22.90 & 0.3 \\
	230 & 1 & 16.30 & 0.3 \\
	240 & 1 & 9.50  & 0.3 \\
	250 & 1 & 4.10  & 0.3 \\
	260 & 1 & 1.30  & 0.3 \\
	270 & 1 & 0.30  & 0.3 \\
	280 & 1 & 2.00  & 0.3 \\
	290 & 1 & 5.60  & 0.3 \\
	300 & 1 & 10.90 & 0.3 \\
	310 & 1 & 17.60 & 0.3 \\
	320 & 1 & 24.40 & 0.3 \\
	330 & 1 & 30.50 & 0.3 \\
	340 & 1 & 35.60 & 0.3 \\
	350 & 1 & 38.70 & 0.3 \\
	360 & 1 & 39.50 & 0.3 \\
	\bottomrule
\end{tabular}
	\caption{Anpassung einer zu erwartenden Kurve an die Messwerte}
	\label{fig:malus}
\end{figure}
Man kann an der Anpassung gut erkennen, dass die Intensität wie erwartet dem Gesetz von Malus folgt.
Dies bestätigt auch die Annahme, dass der Laser durch die an der Entladungsröhre befestigten Brewster-Fenster mit linear polarisiertem Licht betrieben wird.
Die Abweichungen der Anpassung (das reduzierte $\chi^2$ von \num{25.4} bestätigt die schlechte Übereinstimmung) von den Messwerten lassen sich durch eine nicht ideale Justage erklären.
Es ist davon auszugehen, dass der Laserstrahl den Polarisationsfilter nicht genau an dessen Rotationsachse durchtreten hat.
Geht man außerdem davon aus, dass der Transmissionsgrad des Filters Variationen über dessen Oberfläche aufweist, so würde dies erklären, weshalb das Intensitätsmaximum bei $\sim \SI{180}{\degree}$ kleiner ist als dies bei $\sim \SI{0}{\degree}$ bzw. $\SI{360}{\degree}$ der Fall ist.

\subsubsection{Polarisationsgrad des Lasers}
Nachfolgend soll der Polarisationsgrad des Lasers bestimmt werden.
Dieser ist gegeben durch:
\begin{align}
P=\frac{I_{\parallel}-I_{\perp}}{I_{\parallel}+I_{\perp}}
\end{align}
Der Fehler ergibt sich mit Gauß'scher Fehlerfortpflanzung zu:
\begin{align}
\Delta P=\sqrt{\left(1+\frac{I_{\perp}}{(I_{\parallel}+I_{\perp})^2}\right)^2\left(\Delta I_{\parallel}\right)^2 + \left(\frac{I_{\perp}}{(I_{\parallel}+I_{\perp})^2}-\frac{1}{I_{\parallel}+I_{\perp}}\right)^2\left(\Delta I_{\perp}\right)^2}
\end{align}
Hierbei wird angenommen, dass $I_\parallel$ die Intensität ist, bei der der Linearpolarisaor parallel zur Polarisationsrichtung des Lichts steht und so ein Transmissionsmaximum erreicht wird.
Für $I_\perp$ ist die Polarisationsrichtung des Lichts senkrecht zur Richtung des Polarisators, es liegt ein Transmisssionsminimum vor.
Aus der Anpassung (s.o.) kann die maximale Intensität als Summe der Anpassungsparameter $a$ und $c$ geschrieben werden, das Minimum ist gerade gleich $c$. 
Dann ergibt sich $P$ mit $I_\parallel=\SI{2.28+-0.05}{\milli\watt}$ und $I_\perp=\SI{0.04+-0.03}{\milli\watt}$ (die Fehler wurden mit Gauß'scher Fehlerfortpflanzung berechnet) zu:
\begin{align}
P=\num{0.97+-0.06}
\end{align}
Daran kann man sehen, dass das hier verwendete Laserlicht sehr stark (linear) polarisiert ist.

\subsection{Messung des Strahlprofils und des Stabilitätsgebiets des Lasers}
Im Folgenden soll das Strahlprofil im Resonator bei verschiedenen Resonatorlängen bestimmt werden.
Es werden dabei beispielhaft für eine Resonatorlänge Tabellen und Graphen in diesem Abschnitt aufgeführt.
Eine Zusammenstellung aller vermessenen Resonatorlängen findet sich in Anhang \ref{app:strahlprofil}.

\subsubsection{Messmethode}
Die Messung der Strahlprofils, beziehungsweise einer dazu proportionalen Größe (siehe Abschnitt \ref{sssec:spaltbreite_strahlradius}), erfolgt mithilfe eines Messschiebers, der in den Strahlengang gebracht wird.
Dabei wird die in \cite{anleitung} dargestellte Methode, bei der der Messschieber bei konstanter Spaltbreite longitudinal im Resonator verschoben wird, verwendet.
Konkret wird der Messschieber auf die konstante Breite $d$ eingestellt und an eine Position im Resonator gebracht, an der bei transversaler Verschiebung der Laser kurz aufblitzt.
Dann wird dessen Abstand $z$ vom ebenen Resonatorspiegel langsam vergrößert und beobachtet, wann der Strahl nichtmehr aufleuchtet, wenn der Messschieber gegen die Strahlachse bewegt wird.
Ist dieser Punkt erreicht, so wird der Abstand $z$ wieder verkleinert bis der Laser gerade noch anfängt zu arbeiten.
Dann kann der Abstand $z$ des Messschiebers vom ebenen Resonatorspiegel vermessen werden.
Dieser ist gleichzeitig der Abstand von der Strahltaille, da diese bei halbsymmetrischen Resonatoren auf dem ebenen Spiegel liegt (siehe auch Abschnitt (REF!)).

Bei dieser Methode treten Fehler in der Bestimmung der Position $z$ des Messschiebers und dessen Spaltbreite $d$ auf.
Der Fehler in der Position setzt sich zusammen aus dem Messfehler sowie einem zusätzlichen Beitrag aufgrund von nicht optimaler Ausrichtung (senkrecht zur optischen Achse) des Messschiebers.
Dazu kommt, dass die Position $z$ hinter dem Resonator nicht direkt mit der auf der optischen Bank angebrachten Skala vermessen werden konnte, weshalb für Messpunkte hinter dem Resonator ein größerer Fehler angenommen wurde, als für solche vor dem Resonator.
Für den Fehler in der Spaltbreite $d$ wird $\Delta = \SI{0.02}{\milli\metre}$ angenommen, was darin begründet liegt, dass trotz einer Auflösung von \SI{0.01}{\milli\metre} bei typischen (digitalen) Messschiebern nur eine Genauigkeit von \SI{0.02}{\milli\metre} erreicht wird \cite{messschieber_katalog}.
\begin{table}[h]
	\centering
	\begin{tabular}{SSSS}
\toprule
{$z \, / \, \si{\milli\metre}$} & {$\Delta z \, / \, \si{\milli\metre}$} & {$d \, / \, \si{\milli\metre}$} & {$\Delta d \, / \, \si{\milli\metre}$} \\
\midrule
10  & 2 & 0.67 & 0.01 \\
27  & 2 & 0.68 & 0.01 \\
442 & 4 & 1.04 & 0.01 \\
492 & 4 & 1.10 & 0.01 \\
542 & 4 & 1.16 & 0.01 \\
581 & 4 & 1.21 & 0.01 \\
639 & 4 & 1.26 & 0.01 \\
674 & 4 & 1.31 & 0.01 \\
708 & 4 & 1.36 & 0.01 \\
744 & 4 & 1.38 & 0.01 \\
\bottomrule
\end{tabular}
	\caption{Messdaten zum Strahlprofil im Resonator der Länge $L = \SI{795 +- 4}{\milli\metre}$}
	\label{tab:ex_messdaten_strahlprofil}
\end{table}


\subsubsection{Zusammenhang zwischen Spaltbreite am Messschieber und Strahlradius}
\label{sssec:spaltbreite_strahlradius}
Zunächst gilt es zu beachten, dass der Strahlradius $w$ keine scharfe Begrenzung der transversalen Intensitätsverteilung eines Gaußstrahls darstellt.
Wie in Abschnitt \ref{gaussstrahlen} erklärt, ist das transversale Profil der $\mathrm{TM}_{00}$-Mode gegeben durch eine Gaußfunktion, anhand derer der Strahlradius definiert ist als der Abstand von der Strahlachse, bei dem die Intensität auf $1/\mathrm{e}^2$ abgefallen ist.
Durch Begrenzung der transversalen Strahlbreite auf die Spaltbreite des Messschiebers $d$, kommt es zu zusätzlichen Verlusten aufgrund von Reflexion und Absorption an den Backen des Messschiebers.
Diese Verluste führen zu einer Änderung der Schwellenbedingung für Laseroperation, was bei dieser Messmethode ausgenutzt wird, um eine zum Strahlradius $w$ proportionale Größe $d$ zu messen.
Da der Laser durch die Verschiebung des Messschiebers entlang der Strahlachse, stehts an der Schwelle operiert, ist die Strahlleistung im Spalt $P_\mathrm{Spalt}$ konstant bezüglich der verschiedener Messpositionen $z$.
Im Folgenden soll gezeigt werden, dass dann in der Tat gilt:
\begin{align}
	d = \alpha \cdot w(z) \text{,}
\end{align}
wobei $\alpha$ unabhängig vom Abstand $z$ von der Strahltaille ist.
Dazu betrachtet man die Leistung des Strahls auf der Fläche des durch den Messschiebers gebildeten Spaltes mit der Breite $d$:
\begin{align}
	P_\mathrm{Spalt} &= I_0 \left( \frac{w_0}{w(z)} \right)^2 \int_{A_\mathrm{Spalt}} \exp\left( -\frac{2 r^2}{w^2(z)} \right) \mathrm{d}A
\end{align}
mit $A_\mathrm{Spalt} = \left\{(x,y) \in \mathbb{R}^2 \middle| |x| < \frac{d}{2} \right\}$ und $r^2 = x^2 + y^2$.
Ausführen der Integration liefert die (konstante) Spaltleistung:
\begin{align}
	P_\mathrm{Spalt} &= \frac{1}{2} \pi I_0 w_0^2 \,\erf\left( \frac{d}{\sqrt{2} w(z)} \right)
\end{align}
Mit dem Ansatz $d(z) = \alpha(z) \cdot w(z)$ folgt:
\begin{align}
	\alpha(z) = \sqrt{2} \, \erf^{-1}\left( \frac{2}{\pi} \cdot \frac{P_\mathrm{Spalt}}{I_0 w_0^2}\right) = \mathrm{konst.}
\end{align}
Der Proportionalitätsfaktor $\alpha$ ist somit tatsächlich unabhängig von der Position des Messschiebers.

\subsubsection{Experimentelle Bestimmung des Strahlprofils}
Mit den Überlegungen aus dem vorigen Abschnitt wählen wir als Anpassungshypothese:
\begin{align}
	 r(z) = \frac{d(z)}{2} = \alpha \cdot w_\mathrm{exp.}(z) \text{,}
\end{align}
wobei das Profil $w_\mathrm{exp.}(z)$, unabhängig von der Rayleigh-Länge $z_0$, durch die Strahltaille $w_0$ ausgedrückt wird:
\begin{align}
	w_\mathrm{exp.}(z) = w_0 \sqrt{1 + \left( \frac{\lambda}{\pi} \right)^2 \left( \frac{z}{w_0^2}\right)^2}
\end{align}
(Folgerung aus Gleichungen \ref{eq:gauss_axialprofil} und \ref{eq:rayleigh_laenge}).
Das Halbieren\footnote{Auch der Fehler $\Delta r$ halbiert sich entsprechend: $\Delta r = \Delta d / 2$} der Spaltbreite ist dabei optional und wurde durchgeführt um eine zum Strahlradius $w$ ähnlichere Größe zu erhalten.

Somit kann die Anpassung an die Messpunkte durch die \emph{Methode der kleinsten Quadrate} erfolgen.
Dabei werden die Proportionalitätskonstante $\alpha$ und die Strahltaille $w_0$ als Anpassungsparameter gewählt.
Die Messdaten, Anpassung, experimentelles und theoretisches Strahlprofil wurden in Abbildung \ref{fig:ex_strahlradius} aufgetragen.
Dabei ist das theoretische Strahlprofil eindeutig durch die aus der Resonatorgeometrie folgenden Strahltaille (Gleichung \ref{eq:strahltaille_halbsym}) und der Wellenlänge des Lasers $\lambda$ festgelegt.
Für die Wellenlänge des HeNe-Lasers wurde der Wert aus Gleichung \ref{eq:HeNe_wellenlaenge} verwendet.
Die resultierenden Anpassungsparameter für alle Resonatorlängen wurden in Tabelle \ref{tab:strahlradius_fitergebnis} aufgetragen.
\begin{table}[h]
	\centering
	\begin{tabular}{SSSSSSS}
\toprule
{$L \, / \, \si{\milli\metre}$} & {$\Delta L \, / \, \si{\milli\metre}$} & {$w_0 \, / \, \si{\milli\metre}$} & {$\Delta w_0 \, / \, \si{\milli\metre}$} & {$\alpha$} & {$\Delta \alpha$} & {$\chi_\mathrm{red.}^2$}\\
\midrule
500 & 4 & 0.3239 & 0.0029 & 1.149 & 0.005 & 0.52 \\
690 & 4 & 0.3093 & 0.0039 & 1.223 & 0.006 & 3.54 \\
795 & 4 & 0.2888 & 0.0052 & 1.195 & 0.010 & 6.23 \\
892 & 4 & 0.2557 & 0.0043 & 1.206 & 0.012 & 7.46 \\
\bottomrule
\end{tabular}
	\caption{Ergebnisse der Anpassung des Strahlprofils an die aufgenommenen Daten}
	\label{tab:strahlradius_fitergebnis}
\end{table}
\begin{figure}[h]
	\centering
	\input{./plots/strahlradien/strahlradius_80cm.tex}
	\caption{Anpassung des Strahlprofils an die Messdaten für die Resonatorlänge $L = \SI{795 +- 4}{\milli\metre}$}
	\label{fig:ex_strahlradius}
\end{figure}

Bemerkenswert ist, dass man bei den letzten\footnote{bezüglich des Abstandes $z$ von der Strahltaille} Messpunkten im Strahlprofil ein systematisches Abknicken beobachten kann.
Die Ursache liegt vermutlich darin, dass an der Knickstelle das Aufblitzen des Lasers von der anderen Seite des Messschiebers beobachtet wurde, da der sphärische Resonatorspiegel den Blick auf den Messschieber beeinträchtigt hat.
Da der Reflexionsgrad beider Seiten des Messschiebers unterschiedlich ist, kommt es hier zu einem systematischen Fehler aufgrund der unterschiedlichen Helligkeiten des Reflexes an den Messbacken.

Zusammenfassend kann man sagen, dass des experimentell bestimmte Strahlprofil gut mit dem Theoretischen übereinstimmt.
Einzig für die Resonatorlänge $L = \SI{500}{\milli\metre}$ kommt es zu einer signifikanten Abweichung, da auf dem Messbereich nur wenige Messpunkte zur Verfügung standen.


\subsection{Aufbau der optischen Diode}

Die optische Diode wird in diesem Versuch durch eine Kombination von Linearpolarisator und $\lambda/4$-Platte erreicht.
Sie wird gemäß der Praktikumsanleitung aufgebaut und justiert.
Zur korrekten Einstellung der Diode wird der Polarisator nun so gedreht, dass das transmittierte Licht ein Maximum erreicht.
Hierfür findet man bei der Einstellung von \SI{254}{\degree} das Transmissionsminimum und verdreht den Polarisator um weitere \SI{90}{\degree} (dies folgt aus dem Gesetz von Malus, siehe Abschnitt \ref{ssec:polarisation}), so dass das Transmissionsmaximum bei einer Einstellung von \SI{344}{\degree} erreicht wird.\\
\\
Vor der optischen Diode kann bei der Resonatorlänge \SI{50}{\centi\metre} eine Intensität von \SI{66.2+-0.1}{\milli\volt} gemessen werden, hinter der Diode beträgt die Intensität \SI{47.2+-0.1}{\milli\volt}.
Die Leistungsverluste an der Diode (z.B. Reflexionen an der Vorderseite der $\lambda/4$-Platte) belaufen sich somit auf ca. \SI{1}{\milli\watt}.
Dabei wurde wie in \eqref{eq:umrechnung_watt} der Umrechnungsfaktor \num{17.5} zwischen der gemessenen Intensität in \si{\milli\volt} und \si{\milli\watt} verwendet.

\subsection{Optischer Spektrumanalysator}
\label{ssec:spektrumanalysator}

\subsubsection{Durchführung}

Der optische Spektrumanalysator dient der Beobachtung der Modenabstände des Lasers.
Dazu wird ein externer konfokaler Resonator genutzt, dessen Transmission maximal wird, wenn die Frequenz der jeweiligen Mode in Resonanz mit der Frequenz des exteren Resonator ist.
Die Länge des Resonators wird mit einer anliegenden Wechselspannung kontinuierlich verändert und die Intensität mit einer Photodiode in Abhängigkeit der Länge des Resonators (die proportional zur angelegten Spannung ist), beobachtet.
Um diesen Zusammenhang sicherzustellen wird als Signal für die Wechselspannung ein Sägezahn eingestellt, damit die gemessenen Zeitdifferenzen proportional zur Spannungsdifferenz sind und durch den Piezo-Effekt (Längenänderung ist proportional zu angelegter Spannung) auch die Längendifferenz des externen Resonators proportional zur messbaren Zeitdifferenz ist.
Der Spektrumanalysator wird wie in \cite{anleitung} beschrieben aufgebaut und justiert.
\textbf{Die Modulationsfrequenz der am externen Resonator anliegenden Wechselspannung wird dabei möglichst genau} (die Ausgangsfrequenz des Generators schwankt leicht und lässt sich nur schwer auf einen genauen Wert einstellen) \textbf{auf \SI{50}{\hertz} eingestellt, um äußere Störfaktoren zu minimieren.}
Dies betrifft vor allem die Lampen im Versuchsraum, die mit gewöhnlicher \SI{50}{\hertz} Wechselspannung betrieben werden.\\
\\
Nach der Justage muss kontrolliert werden, dass die Längenänderung des Resonators groß genug ist, um mindestens einen freien Spektralbereich zu sehen.
\textbf{Dies ist gegeben, wenn auf dem Oszilloskop zwei gaußförmige Einhüllende über dem Modenspektrum (das aus Lorentzkurven aufgebaut ist) zu beobachten sind.}

\subsubsection{Messwerte}

Für alle Resonatorlängen wird hier nach dem gleichen Verfahren vorgegangen.
Nach korrekter Justage muss die Amplitude der Wechselspannung entsprechend angepasst werden, damit mindestens ein freier Spektralbereich von den Piezoelementen im externen Resonator durchfahren wird.
Anschließend kann der Intensitätsverlauf auf dem digitalen Oszilloskop beobachtet werden.
Die dabei gemessenen Verläufe wurden in den Abbildungen \ref{fig:ALL0028}, sowie \ref{fig:ALL0026}, \ref{fig:ALL0022} und \ref{fig:ALL0014} (im Anhang) gezeigt. 
Man sieht in allen Abbildungen deutlich die zwei gaußförmigen Einhüllenden in einem freien Spektralbereich.
Zur späteren Auswertung wurde in die Grafiken ebenfalls eine Anpassungskurve eingezeichnet.
Diese folgt aus einer \emph{least-squares}-Anpassung (Methode kleinster Quadrate) mit \texttt{Gnuplot} auf Basis einer Superposition von $n$ Lorentzprofilen:
\begin{align}
f(x)=\sum_{i=1}^{n}a_i\cdot\frac{b_i^2}{(x-c_i)^2+b_i^2}
\end{align}
Dabei beschreibt $a_i$ die Höhe, $b_i$ die Breite und $c_i$ die Lage des $i$-ten Maximums.
Da die Anpassungen an die hier gemessenen Werte auf sehr vielen freien Parametern beruhen, ist die Angabe von Startwerten sehr wichtig.
Trotzdem konnten einzelne, flache Maxima nicht immer angepasst werden, so dass sie nicht weiter berücksichtigt wurden.
Weiterhin wurden die Höhen einzelner Maxima nicht korrekt angepasst, wie man in den Darstellungen gut sehen kann.
Da für die quantitative Auswertung jedoch nur die $b_i$ (die Schwerpunkte) benötigt werden, wurden die von \texttt{Gnuplot} gefundenen Werte für diese Parameter trotzdem verwendet.
\begin{figure}[htb]
\centering
\begin{tabular}{@{}lcccc|cc@{}}
\toprule
Nummer $i$   & $x_i$ / \si{\milli\second} & $\Delta(x_i)$ / \si{\milli\second} & $\Delta T$ / \si{\milli\second} & $\Delta(\Delta T)$ / \si{\milli\second} & $T_\text{konf.}$ / \si{\milli\second} & $\Delta T_\text{konf.}$ / \si{\milli\second} \\ \midrule
1            & 5,360  & 0,007   & \multirow{2}{*}{0,974} & \multirow{2}{*}{0,008}          & 4,474     & 0,011       \\
2            & 6,334  & 0,003   & \multirow{2}{*}{0,898} & \multirow{2}{*}{0,004}          & 4,271     & 0,004       \\
3            & 7,232  & 0,003   & \multirow{2}{*}{0,882} & \multirow{2}{*}{0,005}          & 4,145     & 0,005       \\
4            & 8,114  & 0,004   & \multirow{2}{*}{0,885} & \multirow{2}{*}{0,011}          & 4,051     & 0,005       \\
5            & 8,999  & 0,010   &                        &                & 3,947     & 0,023       \\
6            & 9,834  & 0,008   & \multirow{2}{*}{0,771} & \multirow{2}{*}{0,009}          &           &             \\
7            & 10,605 & 0,003   & \multirow{2}{*}{0,772} & \multirow{2}{*}{0,004}          &           &             \\
8            & 11,377 & 0,003   & \multirow{2}{*}{0,788} & \multirow{2}{*}{0,005}          &           &             \\
9            & 12,166 & 0,003   & \multirow{2}{*}{0,780} & \multirow{2}{*}{0,021}          &           &             \\
10           & 12,946 & 0,021   &                        &                &           &             \\
Mittelwert: &        &         & 0,840                  & 0,002          & 4,179     & 0,003       \\ \bottomrule
\end{tabular}
\caption{Resonatorlänge 50cm ALL0028}
\label{fig:ALL0028}
\end{figure}
\subsubsection{Bestimmung der Modenabstände}
Der konfokale Modenabstand ist gegeben durch \cite{siegman}:
\begin{align}
\Delta\nu_\text{konf.}=\frac{c}{4\,l}
\end{align}
Hierbei entspricht $l$ der Länge des externen Resonators (hier \SI{5}{\centi\metre}), woraus für diesen Versuch folgt:
\begin{align}
\Delta\nu_\text{konf.}=\SI{1.499}{\giga\hertz}
\end{align}
Daraus folgt für den freien Spektralbereich des konfokalen Resonators\footnote{In der Literatur wird der konfokale Modenabstand oft gleichgesetzt mit dem freien Spektralbereich. Hier soll mit dem freien Spektralbereich nur der Abstand zweier longitudinaler Moden bezeichnet werden, nicht der Bereich zwischen zwei Transmissionsmaxima des Fabry-Pérot-Interferometers.} (siehe auch \eqref{eq:modenabstand}):
\begin{align}
\Delta\nu_\text{FSR}=\frac{c}{2\,l}=2\,\Delta\nu_\text{kon}=\SI{2.998}{\giga\hertz} \text{,}
\end{align}
Man beobachtet, dass für ungerade $n+m$ allerdings auch Transversalmoden TEM$_nm$ in dem Frequenzspektrum zu erkennen sind.
Sie fallen genau zwischen die beobachtbaren Maxima zweier longitudinaler Moden.
Daher ergibt sich der Abstand einer longitudinalen Mode zu einer dieser transversalen Moden als konfokaler Modenabstand, der den freien Spektralbereich gerade halbiert.
Die zwei beobachteten Gaußprofile entsprechen daher einer longitudinalen und einer der oben genannten Transversalmoden.\\
\\
Die Berechnung des longitudinalen Modenabstands wird hier beispielhaft für den Resonator der Länge \SI{50+-0.4}{\centi\metre} durchgeführt, die Tabellen für alle anderen Resonatorlängen befinden sich im Anhang.
In Tabelle \ref{tab:ALL0028} wurden dabei die jeweiligen Positionen $x_i$ der Maxima (Ergebnisse der Anpassung) notiert und die Abstände $\Delta T$ zweier aufeinanderfolgender Maxima innerhalb eines Gaußprofils berechnet.
Der Fehler ergibt sich mit Gaußscher Fehlerfortpflanzung aus:
\begin{align}
\Delta(\Delta T)=\Delta(x_{i+1}-x_i)=\sqrt{(\Delta x_{i+1})^2+(\Delta x_i)^2}
\label{eq:fehler_differenz}
\end{align}
Außerdem wurde der Abstand zweier gleicher longitudinaler Moden (d.h. beispielweise für den Fall von fünf Moden pro Gaußprofil der Abstand $T_\text{konf.}=x_{i+5}-x_i$, siehe auch Grafik \ref{fig:ALL0028}) errechnet, der Fehler dieses Abstands berechnet sich analog zu \eqref{eq:fehler_differenz}.
Dieser Wert entspricht gerade dem konfokalen Modenabstand.
Durch Bildung des varianzgewichteten Mittelwerts über die Werte für beide Abstände ergeben sich die ebenfalls in die Tabelle eingetragenen Werte für den Abstand zweier transversaler Moden und den konfokalen Modenabstand. 
\begin{table}[h]
\centering
\resizebox{\columnwidth}{!}{%
\begin{tabular}{@{}lcccc|cc@{}}
\toprule
Nummer $i$   & $x_i$ / \si{\milli\second} & $\Delta(x_i)$ / \si{\milli\second} & $\Delta T$ / \si{\milli\second} & $\Delta(\Delta T)$ / \si{\milli\second} & $T_\text{konf.}$ / \si{\milli\second} & $\Delta T_\text{konf.}$ / \si{\milli\second} \\ \midrule
1            & 5,360  & 0,007   & \multirow{2}{*}{0,974} & \multirow{2}{*}{0,008}          & 4,474     & 0,011       \\
2            & 6,334  & 0,003   & \multirow{2}{*}{0,898} & \multirow{2}{*}{0,004}          & 4,271     & 0,004       \\
3            & 7,232  & 0,003   & \multirow{2}{*}{0,882} & \multirow{2}{*}{0,005}          & 4,145     & 0,005       \\
4            & 8,114  & 0,004   & \multirow{2}{*}{0,885} & \multirow{2}{*}{0,011}          & 4,051     & 0,005       \\
5            & 8,999  & 0,010   &                        &                & 3,947     & 0,023       \\
6            & 9,834  & 0,008   & \multirow{2}{*}{0,771} & \multirow{2}{*}{0,009}          &           &             \\
7            & 10,605 & 0,003   & \multirow{2}{*}{0,772} & \multirow{2}{*}{0,004}          &           &             \\
8            & 11,377 & 0,003   & \multirow{2}{*}{0,788} & \multirow{2}{*}{0,005}          &           &             \\
9            & 12,166 & 0,003   & \multirow{2}{*}{0,780} & \multirow{2}{*}{0,021}          &           &             \\
10           & 12,946 & 0,021   &                        &                &           &             \\
Mittelwert: &        &         & 0,840                  & 0,002          & 4,179     & 0,003       \\ \bottomrule
\end{tabular}}
\caption{Messwerte und Berechnung zum longitudinalen Modenabstand. $x_i$ beschreibt die Position der Maxima, $\Delta T$ die Differenz zweier aufeinander folgender Positionen und $T_\text{konf.}$ den Abstand zweier gleicher Tranversalmoden. Dieser ist getrennt dargestellt, da die Werte keinen zeilenweisen Bezug zu den $x_i$ haben.}
\label{tab:ALL0028}
\end{table}
Mit den Mittelwerten kann nun der Abstand zweier Moden in Einheiten des konfokalen Modenabstands gefunden werden.
Man erhält so für die hier betrachtete Resonatorlänge \SI{50+-0.4}{\centi\metre}\footnote{Aus der Proportionalität zwischen Zeit und Spannung am Piezoelement und damit der Länge des externen Resonators gilt dieses Verhältnis ebenso für die Frequenzen.}:
\begin{align}
\frac{\Delta T}{T_\text{konf.}}= \num{0.201+-0.001}
\end{align}
Aus der Gaußschen Fehlerfortpflanzung folgt für den Fehler:
\begin{align}
\Delta\left(\frac{\Delta T}{T_\text{konf.}}\right)=\sqrt{\frac{(\Delta(\Delta T))^2}{T_\text{konf.}^2}+\frac{(\Delta T)^2(\Delta T_\text{konf.})^2}{T_\text{konf.}^4}}
\end{align}
Die Berechnung des Modenabstands erfolgt nach:
\begin{align}
\Delta\nu_\text{long.}=\frac{\Delta\nu_\text{konf.}}{T_\text{konf.}} \cdot \Delta T 
\end{align}
Damit können die gemessenen Zeiten für den longitudinalen Modenabstand in Frequenzen umgerechnet werden.
Der konfokale Modenabstand (hier \SI{1.499}{\giga\hertz}, s.o.) ist bekannt und kann so als Ausgangspunkt zur Umrechnung zwischen Frequenz- und Zeitdomäne genutzt werden.
Man erhält damit für den longitudinalen Modenabstand:
\begin{align}
\Delta\nu_\text{long.}=\SI{301.3+-0.7}{\mega\hertz}
\end{align}
Der Fehler wurde abermals mit Gaußscher Fehlerfortpflanzung und mit den einzelnen Fehlern für $\Delta T$ und $T_\text{konf.}$ berechnet mit:
\begin{align}
\Delta\left(\Delta\nu_\text{long.}\right)=\Delta\nu_\text{konf.}\cdot\sqrt{\frac{(\Delta(\Delta T))^2}{T_\text{konf.}^2}+\frac{(\Delta T)^2(\Delta T_\text{konf.})^2}{T_\text{konf.}^4}}
\end{align}

\subsubsection{Ergebnisse und Fazit}

Mit dieser Vorgehensweise erhält man für alle vier Resonatorlängen die in Tabelle \ref{tab:modenabstand} eingetragenen Ergebnisse für den Abstand zweier Moden und den longitudinalen Modenabstand.
\begin{table}[htb]
\centering
\begin{tabular}{@{}lcccc@{}}
\toprule
Resonatorlänge $L$ / \si{\centi\metre}& $\nu_\text{long.}$ / \si{\mega\hertz} &$\Delta\nu_\text{long.}$ / \si{\mega\hertz} & $\frac{\Delta T}{T_\text{konf.}}$ & $\Delta\left(\frac{\Delta T}{T_\text{konf.}}\right)$ \\ \midrule
\num{50+-0.4}			              & 301,3 & 0,7 & 0,201 & 0,001 \\
\num{69+-0.4}                         & 219,8 & 0,4 & 0,147 & 0,001 \\
\num{79.5+-0.4}                       & 188,7 & 0,4 & 0,126 & 0,001 \\
\num{89.2+-0.4}                       & 169,6 & 0,4 & 0,113 & 0,001 \\ \bottomrule
\end{tabular}
\label{tab:modenabstand}
\caption{Ergebnisse für $\nu_\text{long.}$ und $\frac{\Delta T}{T_\text{konf.}}$ für alle Resonatorlängen.}
\end{table}
Eine Bewertung der gefundenen Werte erfolgt später im Vergleich zu den ermittelten Modenabständen mit einer optischen Schwebung.

\subsection{Präzise Messung des Modenabstandes mittels einer optischen Schwebung}
In diesem Teil soll der Modenabstand der longitudinalen Moden im Resonator durch eine optische Schwebung gemessen werden.
Dazu betrachtet man die Überlagerung des elektrischen Feldes zweier Axialmoden mit den Kreisfrequenzen $\omega_1$ und $\omega_2$ auf einer Photodiode.
Die Superposition der elektrischen Felder kann geschrieben werden als:
\begin{align}
	E(t) = E_1 \sin(\omega_1 t) + E_2 \sin(\omega_2 t)
	\label{eq:superposition_efeld}
\end{align}
Da die Photodiode die Intensität $I$ des auftreffenden Lichts misst und diese gegeben ist durch $I \propto E^2$, erhält man durch quadrieren von Gleichung \ref{eq:superposition_efeld} und Anwendung der trigonometrischen Additionstheoreme das resultierende Frequenzspektrum der Intensität, welche aus den Frequenzen $\omega_1 + \omega_2, |\omega_1 - \omega_2|$ sowie den zweiten Harmonischen $2\omega_1, 2\omega_2$ besteht.
Mit Ausnahme von dem Heterodynsignal mit der Frequenz $|\omega_1 - \omega_2|$, welche in der Größenordnung von einigen $\SI{100}{\mega\hertz}$ liegt, sind die restlichen auftretenden Frequenzen so groß ($\sim \si{\peta\hertz}$), dass diese aufgrund der endlichen Bandbreite der Photodiode ($\sim \si{\giga\hertz}$) nicht gemessen werden können.
Auch die Frequenz der optischen Schwebung ist so hoch, dass diese nicht direkt mit dem vorhandenen analogen Oszilloskop (Bandbreite $B = \SI{35}{\mega\hertz}$) gemessen werden kann.
Daher wird in einer zweiten Stufe das Signal der Schwebung mit dem eines Hochfrequenzgenerators der Frequenz $\nu_\mathrm{HF}$ gemischt.

\subsubsection{Mathematische Beschreibung der Funktionsweise eines Mischers}
Die Wirkung eines solchen \textbf{Mischers} auf zwei Eingangssignale $x(t) = \hat{x} \sin(\omega_1 t)$ und $y(t) = \hat{y} \sin(\omega_2 t)$ lässt sich auf zwei verschiedenen Weisen realisieren \cite{horowitz_hill}:
\begin{itemize}
	\item \textbf{multiplikative Mischung:} Durch Multiplikation der beiden Signale $x(t)$ und $y(t)$ führt die Anwendung der trigonometrischen Identität:
	\begin{align}
		\sin(\omega_1 t) \sin(\omega_2 t) = \frac{1}{2} \cos\left[ (\omega_1 - \omega_2) t \right] - \frac{1}{2} \cos\left[ (\omega_1 + \omega_2) t\right]
	\end{align}
	auf das resultierende Signal:
	\begin{align}
		(x\cdot y)(t) = \frac{\hat{x} \, \hat{y}}{2} \left\{ \cos\left[ (\omega_1 - \omega_2) t \right] - \cos\left[ (\omega_1 + \omega_2) t\right]\right\} \text{.}
	\end{align}
	Das resultierende Spektrum enthält sowohl die Summe als auch die Differenz der beiden Eingangsfrequenzen.
	\item \textbf{Mischung durch Anwendung einer nichtlinearen Operation auf die Summe der Signale:} Um die Differenzfrequenz $|\omega_1 - \omega_2|$ zu bestimmen, ist es nicht nötig das Produkt der beiden Eingänge zu bilden.
	Alternativ reicht es aus die Summe beider Signale zu bilden und eine nichtlineare Operation $f$ auf diese anzuwenden.
	Beschreibt man diese durch eine Potenzreihenentwicklung:
	\begin{align}
		f(x) = \sum_{n=0}^{\infty} a_n x^n
	\end{align}
	und wendet diese auf die Summe der Eingangssignale $(x+y)(t)$ an, so folgt für das Ausgangssignal:
	\begin{align}
		f(x+y)(t) = \sum_{n=0}^{\infty} a_n \left[ \hat{x} \sin(\omega_1 t) + \hat{y} \sin(\omega_2 t) \right]^n
	\end{align}
	Sofern mindestens einer der geraden Koeffizienten $a_n$ für $n = 2, 4, 6\dots$ nicht verschwindet, führt dies zum Auftreten eines Terms mit der Frequenz $|\omega_1 - \omega_2|$.
	Dies kann mit dem binomischen Lehrsatz sowie einigen trigonometrischen Identitäten gezeigt werden, dennoch soll hier darauf verzichtet werden.
	Der Spezialfall $f(x) = x^2$ ist bereits bei der Mischung der elektrischen Felder der verschiedenen Axialmoden in der Photodiode aufgetreten, da die gemessene Intensität quadratisch mit der Summe der Feldstärken $E_1$ und $E_2$ zusammenhängt.
	Demnach agiert bereits die Photodiode als Mischer für die elektrischen Feldstärken der verschiedenen Moden.
\end{itemize}

\subsubsection{Messung des Modenabstandes durch Mischung mit der Hochfrequenz}
Wie bereits erwähnt, wird das Signal der optischen Schwebung mit der Frequenz $\Delta \nu = |\omega_1 - \omega_2| / (2\pi)$ mit dem eines Hochfrequenzgenerators $\nu_\mathrm{HF}$ gemischt.
Dabei wird der \textbf{Leistungspegel} des Generators auf $L = \SI{+7}{\dBm}$ eingestellt. Um diese logarithmische Größe in Einheiten einer Leistung zu erhalten, betrachtet man die Definition des Leistungspegels $L$: 
\begin{align}
	L = 10 \cdot \log_{10}\left( \frac{P}{P_0}\right) \text{,}
\end{align}
wobei $P_0$ als Referenzgröße dient.
Für das Dezibel-Milliwatt (\si{dBm}) ist diese gegeben durch $P_0 = \SI{1}{\milli\watt}$, womit für die Leistung des Hochfrequenzgenerators folgt:
\begin{align}
	P = P_0 \cdot 10^{\frac{L}{10}} = \SI{1}{\milli\watt} \cdot 10^{\num{0.7}} \approx \SI{5}{\milli\watt}
\end{align}
Schließlich betrachtet man das Resultat der Mischung am Oszilloskop, welches die Differenzfrequenz $\nu = |\nu_\mathrm{HF} - \Delta \nu|$ aufweist.
Aufgrund der begrenzten Bandbreite des Oszilloskops ist es nötig, eine überschlagsmäßige Berechnung des Modenabstands $\Delta \nu$ durchzuführen, um einen Anhaltspunkt für die Frequenzeinstellung des Hochfrequenzgenerators zu erhalten.
Diese erfolgt über den theoretischen Axialmodenabstand (siehe \eqref{eq:modenabstand}):
\begin{align}
	\Delta \nu = \frac{c}{2 L} \text{,}
\end{align}
wobei $L$ die gemessene Resonatorlänge ist.
Fällt die Hochfrequenz $\nu_\mathrm{HF}$ und der Modenabstand $\Delta \nu$ innerhalb der Bandbreite des Oszilloskops zusammen, so kann ein sinusförmiges Signal gemessen werden.
(mehr Beschreibung zum Signal)
Anschließend wird die Hochfrequenz $\nu_\mathrm{HF}$ schrittweise verändert, sodass die Periode des Signals zunimmt.
Durch diese Methodik nähert man sich langsam der minimalen Signalfrequenz, wobei diese nicht beliebig genau bestimmt werden kann, da bei einer Annäherung in \SI{50}{\kilo\hertz}-Schritten das Signal in der Nähe des Minimums im Rauschen untergeht.
Ist dieser Punkt erreicht, wird die Frequenz des Signalgenerators notiert und dient als Approximation\footnote{Streng genommen müsste noch die Frequenz des Restsignals auf dem Oszilloskops vermessen werden. Da dieses aber im Rauschen nicht auszumachen ist und dessen Frequenz klein ist gegen den Modenabstand $\Delta \nu$, kann dieser Beitrag vernachlässigt werden.} des Modenabstandes $\Delta \nu$.
Aus der verwendeten Schrittbreite von \SI{50}{\kilo\hertz} folgt ein abgeschätzter Fehler für die Bestimmung des Frequenzminimums von $\Delta \nu_\mathrm{HF} = \SI{25}{\kilo\hertz}$ (halbe Schrittbreite).
Die gemessenen Modenabstände wurden in Tabelle \ref{tab:messdaten_optische_schwebung} aufgetragen.
\begin{table}[h]
	\centering
	\begin{tabular}{@{}llll@{}}
\toprule
  {$L \, / \, \si{\centi\metre}$} & {$\Delta L \, / \, \si{\centi\metre}$} & {$\nu_\mathrm{HF} \, / \, \si{\mega\hertz}$} & {$\Delta \nu_\mathrm{HF} \, / \, \si{\mega\hertz}$} \\
  \midrule
  50.0 & 0.4 & 299.360 & 0.025 \\ 
  69.0 & 0.4 & 216.150 & 0.025 \\
  79.5 & 0.4 & 187.450 & 0.025 \\
  89.2 & 0.4 & 168.650 & 0.025 \\
  \bottomrule
\end{tabular}
	\caption{Messwerte des longitudinalen Modenabstandes. Die eingestellte Hochfrequenz im Frequenzminimum approximiert den axialen Modenabstand $\nu_\mathrm{HF} \approx \Delta \nu$.}
	\label{tab:messdaten_optische_schwebung}
\end{table}

Diese Messwerte können im Hinblick auf den theoretischen Modenabstand in Gleichung \ref{eq:modenabstand} einen Wert für die Lichtgeschwindigkeit $c$ liefern.
Dazu wird die doppelte Resonatorlänge $2 L$ gegen die inverse Frequenz des Hochfrequenzgenerators im Frequenzminimum des gemischten Signals auf.
Der Graph sollte daher dem theoretischen Zusammenhang:
\begin{align}
	2 L = c \cdot \frac{1}{\nu_\mathrm{RF}}
\end{align}
folgen, wobei die Näherung $\Delta \nu \approx \nu_\mathrm{HF}$ verwendet wird.
Diese Form der Linearisierung wird gewählt, da bei einer Anpassung mit der \emph{Methode der kleinsten Quadrate} nur die Fehler in $y$-Richtung in die Gewichtung der Messpunkte eingehen.
Da der Fehler in der Frequenzbestimmung im Verhältnis zur Resonatorlängenbestimmung klein ist, wurde diese Form gewählt um eine sinnvolle Gewichtung bei der Geradenanpassung zu erzielen.
Mithilfe von \texttt{Gnuplot} wird eine Ursprungsgerade an die linearisierten Datenpunkte angepasst, wobei die Fehler gegeben sind durch:
\begin{align}
\Delta(2L) &= 2 \Delta L \\
\Delta \left( \frac{1}{\nu_\mathrm{HF}} \right) &= \frac{\Delta \nu_\mathrm{HF}}{\nu_\mathrm{HF}^2}
\end{align}
Die Linearisierung wurde in Abbildung \ref{fig:linearisierung_modenabstand} aufgetragen.
\begin{figure}[h]
	\centering
	\input{./plots/linearisierung_modenabstand.tex}
	\caption{Linearisierung des gemessenen Modenabstandes zur Bestimmung der Lichtgeschwindigkeit. Der Fehler der inversen Frequenz liegt innerhalb der Breite der Datenpunkte.}
	\label{fig:linearisierung_modenabstand}
\end{figure}
Die Güte der Anpassung ist gegeben durch das reduzierte Chi-Quadrat $\chi_\mathrm{red.}^2 = \num{0.8}$ und bestätigt das zugrundeliegende Modell.
Die Lichtgeschwindigkeit $c$ lässt sich sofort an der Steigung der angepassten Ursprungsgerade ablesen und ist gegeben durch:
\begin{align}
	c &= \SI{29.929 +- 0.073}{\centi\metre\per\nano\second} \nonumber\\
	&= \SI{2.9929 +- 0.0073 e8}{\metre\per\second}
\end{align}
Dieser Wert stimmt gut mit der Definition der Lichtgeschwindigkeit im Vakuum:
\begin{align}
	c_0 = \SI{299792458}{\metre\per\second}
\end{align}
überein.

\subsection{Modenspektrum des Helium-Neon-Lasers}
In diesem Abschnitt soll der Einfluss der thermischen Bewegung der Neonatome in der Entladungsröhre auf den Helium-Neon-Laser untersucht werden.
Mit der Masse des Neon-Atoms $m_\mathrm{Ne} = \SI{20.18}{\atomicmassunit}$ \cite{iupac_periodic_table} und der abgeschätzten Temperatur $T = \SI{400}{\kelvin}$ innerhalb der Röhre, ist die Geschwindigkeitsverteilung der Neon-Atome eindeutig gegeben durch die Maxwell-Boltzmann-Verteilung.
Die wahrscheinlichste Geschwindigkeit eines Neon-Atoms in der Entladungsröhre ist somit gegeben durch:
\begin{align}
	v_\mathrm{w} = \sqrt{\frac{2 k_\mathrm{B} T}{m}}
\end{align}
Bei der oben abgeschätzten Temperatur von \SI{400}{\kelvin} ergibt sich:
\begin{align}
	v_\mathrm{w} \approx \SI{574}{\metre\per\second}
\end{align}
Betrachtet man die Geschwindigkeitsverteilung entlang der optischen Achse des Lasers, so ist diese gaußförmig um Null.
Dadurch kommt es ebenfalls zu einem gaußförmigen Verstärkungsprofils des aktiven Mediums aufgrund des Dopplereffekts.
Die Halbwertsbreite (FWHM) dieses Profils ist gegeben durch \cite{meschede}:
\begin{align}
	\Delta \nu_\mathrm{D} = \frac{\nu_0}{c} \sqrt{\frac{8 k_\mathrm{B} T \ln(2)}{m}} \text{,}
\end{align}
wobei $\nu_0$ die unverschobene Frequenz des Laserübergangs ist.
Mit den oben angegeben Werten und der Frequenz des Laserübergangs ($\frac{\nu_0}{c} = \frac{1}{\lambda_0}$, $\lambda_0 = \SI{632.8}{\nano\metre}$), folgt die Halbwertsbreite des Verstärkungsprofils:
\begin{align}
	\Delta \nu_\mathrm{D} \approx \SI{1.51}{\giga\hertz}
\end{align}
Um abschätzen zu können, wie viele longitudinale Resonatormoden innerhalb dieses Verstärkungsprofil fallen, berechnet man den freien Spektralbereich des Laserresonators durch:
\begin{align}
	\Delta \nu_\mathrm{FSR} = \frac{c}{2 L} \text{.}
\end{align}
Für einen Resonator der Länge $L = \SI{30}{\centi\metre}$ erhält man somit einen freien Spektralbereich von:
\begin{align}
	\Delta \nu_\mathrm{FSR} \approx \SI{500}{\mega\hertz}
\end{align}
Die Anzahl der Moden die in die Dopplerbreite des Verstärkungsprofils fallen ist somit gegeben durch das Verhältnis von Dopplerbreite zum freien Spektralbereich.
Für den \SI{30}{\centi\metre} Resonator fallen \num{3} Moden innerhalb die Halbwertsbreite.

Für die in diesem Versuch verwendeten Resonatorlänge von \SI{50}{\centi\metre} bis \SI{100}{\centi\metre} folgt für den freien Spektralbereich:
\begin{align}
\SI{150}{\mega\hertz} \le \Delta \nu_\mathrm{FSR} \le \SI{300}{\mega\hertz}
\end{align}
Analog ergibt sich für die Anzahl $N$ der Moden innerhalb der Dopplerbreite des Verstärkungsprofils:
\begin{align}
	&N(L = \SI{50}{\centi\metre}) = 5 \\
	&N(L = \SI{100}{\centi\metre}) = 10
\end{align}
In Abbildung \ref{fig:ALL0028} des optischen Spektrumanalysators für den Resonator der Länge $L = \SI{50 +- 0.4}{\centi\metre}$ kann man in der Tat fünf longitudinale Moden erkennen.
Auch die restlichen Aufnahmen folgen dem Trend der steigenden Anzahl der Moden bei größerer Resonatorlänge.



\section{Fazit}

\section{Grafikstorage}

\clearpage
% BIBLIOGRAPHIE
\vspace{\fill}
% Maximale Anzahl der Einträge in Klammer
% Zitieren mit \cite{lamport94}
\begin{thebibliography}{19}
\bibitem{javan}
	A. Javan, W. R. Bennett, Jr., and D. R. Herriott,
	\emph{Population Inversion and Continuous Optical Maser Oscillation in a Gas Discharge Containing a He-Ne Mixture},
	Phys. Rev. Lett. 6, 106 – Published 1 February 1961

\bibitem{linden}
	S. Linden,
	Skript zur Vorlesung \emph{physik311: Optik und Wellenmechanik} (Stand: 31. Januar 2014),
	Physikalisches Institut, Universität Bonn

\bibitem{siegman}
	A. E. Siegman,
	\emph{Lasers},
	University Science Books 1986,
	Chapter 19: Stable Two-Mirror Resonators

\bibitem{anleitung}
	Physikalisches Praktikum IV: Atome, Moleküle, Festkörper,
	Versuchsbeschreibung \emph{P442: Laser} (Stand: 12. September 2014),
	Universität Bonn
	
\bibitem{messschieber_katalog}
	\emph{Mitutoyo Messgeräte-Katalog},
	ABSOLUTE AOS Digimatic Messschieber,\\
	\url{http://www2.mitutoyo.de/ebooks/german/handmessgeraete/index.html} (Letzter Abruf: 22. Dezember 2014)	

\bibitem{schawlow}
	A. L. Schawlow,
	\emph{Measuring the Wavelength of Light with a Ruler},
	Am. J. Phys., Volume 33, Issue 11 (1965)

\bibitem{NISTSpectra}
	Kramida, A., Ralchenko, Yu., Reader, J., and NIST ASD Team (2014).
	\emph{NIST Atomic Spectra Database} (ver. 5.2).
	\url{http://physics.nist.gov/asd} (Letzter Abruf: 18. Dezember 2014).
	National Institute of Standards and Technology, Gaithersburg, MD.
	
\bibitem{horowitz_hill}
	Paul Horowitz, Winfred Hill,
	\emph{The Art of Electronics Second Edition},
	Cambridge University Press 1989,
	Chapter 13.12: High frequency and high-speed techniques -- Radiofrequency circuit elements

\bibitem{iupac_periodic_table}
	International Union of Pure and Applied Chemistry,
	\emph{IUPAC Periodic Table of the Elements} (Stand: 1. Mai 2013),
	\url{http://iupac.org/reports/periodic_table/}

\bibitem{meschede}
	Dieter Meschede,
	\emph{Optik, Licht und Laser} (3. Auflage),
	Vieweg Teubner 2008
	
	
 
\end{thebibliography}

\clearpage

% APPENDIX
\begin{appendix}
\section{Anhang}
\subsection{Messwerte der Photospannung hinter dem Polarisator}
\begin{table}[htb]
	\centering
	\resizebox{0.85\textwidth}{!}{
	\begin{tabular}{SSSS}
	\toprule
	{Drehwinkel $\varphi / \si{\degree}$} & {Fehler $\Delta\varphi / \si{\degree}$} & {Intensität $I / \si{\milli\volt}$} & {Fehler $\Delta I / \si{\milli\volt}$} \\
	\midrule
	0   & 1 & 39.50 & 0.3 \\
	10  & 1 & 40.00 & 0.3 \\
	20  & 1 & 38.00 & 0.3 \\
	30  & 1 & 33.80 & 0.3 \\
	40  & 1 & 27.90 & 0.3 \\
	50  & 1 & 20.20 & 0.3 \\
	60  & 1 & 13.40 & 0.3 \\
	70  & 1 & 6.45  & 0.3 \\
	80  & 1 & 2.62  & 0.3 \\
	90  & 1 & 0.60  & 0.3 \\
	100 & 1 & 1.10  & 0.3 \\
	110 & 1 & 3.50  & 0.3 \\
	120 & 1 & 8.60  & 0.3 \\
	130 & 1 & 14.70 & 0.3 \\
	140 & 1 & 21.20 & 0.3 \\
	150 & 1 & 27.70 & 0.3 \\
	160 & 1 & 33.40 & 0.3 \\
	170 & 1 & 37.00 & 0.3 \\
	180 & 1 & 38.50 & 0.3 \\
	190 & 1 & 37.50 & 0.3 \\
	200 & 1 & 34.70 & 0.3 \\
	210 & 1 & 29.60 & 0.3 \\
	220 & 1 & 22.90 & 0.3 \\
	230 & 1 & 16.30 & 0.3 \\
	240 & 1 & 9.50  & 0.3 \\
	250 & 1 & 4.10  & 0.3 \\
	260 & 1 & 1.30  & 0.3 \\
	270 & 1 & 0.30  & 0.3 \\
	280 & 1 & 2.00  & 0.3 \\
	290 & 1 & 5.60  & 0.3 \\
	300 & 1 & 10.90 & 0.3 \\
	310 & 1 & 17.60 & 0.3 \\
	320 & 1 & 24.40 & 0.3 \\
	330 & 1 & 30.50 & 0.3 \\
	340 & 1 & 35.60 & 0.3 \\
	350 & 1 & 38.70 & 0.3 \\
	360 & 1 & 39.50 & 0.3 \\
	\bottomrule
\end{tabular}}
	\caption{Messwerte der Spannung an der Photodiode in Abhängigkeit des Winkels am Linearpolarisator}
	\label{tab:malus}
\end{table}
\clearpage

\subsection{Bestimmung des Strahlprofils im Resonator}
\label{app:strahlprofil}
\FloatBarrier
\begin{table}[ht]
	\centering
	\begin{tabular}{SSSS}
\toprule
{$z \, / \, \si{\milli\metre}$} & {$\Delta z \, / \, \si{\milli\metre}$} & {$d \, / \, \si{\milli\metre}$} & {$\Delta d \, / \, \si{\milli\metre}$} \\
\midrule
9   & 2 & 0.74 & 0.02 \\
38  & 2 & 0.75 & 0.02 \\
447 & 4 & 0.98 & 0.02 \\
450 & 4 & 0.99 & 0.02 \\
461 & 4 & 1.00 & 0.02 \\
488 & 4 & 1.01 & 0.02 \\
\bottomrule
\end{tabular}
	\caption{Messdaten zum Strahlprofil im Resonator der Länge $L = \SI{500 +- 4}{\milli\metre}$}
	\label{tab:strahlradius_50}
\end{table}
\begin{figure}[hb]
	\centering
	\input{./plots/strahlradien/strahlradius_50cm.tex}
	\caption{Anpassung des Strahlprofils an die Messdaten für die Resonatorlänge $L = \SI{500 +- 4}{\milli\metre}$}
	\label{fig:strahlradius_50}
\end{figure}
\clearpage
\begin{table}[t]
	\centering
	\begin{tabular}{SSSS}
\toprule
{$z \, / \, \si{\milli\metre}$} & {$\Delta z \, / \, \si{\milli\metre}$} & {$d \, / \, \si{\milli\metre}$} & {$\Delta d \, / \, \si{\milli\metre}$} \\
\midrule
12  & 2 & 0.74 & 0.02 \\
27  & 2 & 0.75 & 0.02 \\
32  & 2 & 0.76 & 0.02 \\
450 & 4 & 1.06 & 0.02 \\
483 & 4 & 1.10 & 0.02 \\
505 & 4 & 1.13 & 0.02 \\
554 & 4 & 1.18 & 0.02 \\
591 & 4 & 1.21 & 0.02 \\
618 & 4 & 1.23 & 0.02 \\
649 & 4 & 1.26 & 0.02 \\
666 & 4 & 1.28 & 0.02 \\
\bottomrule
\end{tabular}
	\caption{Messdaten zum Strahlprofil im Resonator der Länge $L = \SI{690 +- 4}{\milli\metre}$}
	\label{tab:strahlradius_70}
\end{table}
\begin{figure}[b]
	\centering
	\input{./plots/strahlradien/strahlradius_70cm.tex}
	\caption{Anpassung des Strahlprofils an die Messdaten für die Resonatorlänge $L = \SI{690 +- 4}{\milli\metre}$}
	\label{fig:strahlradius_70}
\end{figure}
\clearpage
\begin{table}[t]
	\centering
	\begin{tabular}{SSSS}
\toprule
{$z \, / \, \si{\milli\metre}$} & {$\Delta z \, / \, \si{\milli\metre}$} & {$d \, / \, \si{\milli\metre}$} & {$\Delta d \, / \, \si{\milli\metre}$} \\
\midrule
10  & 2 & 0.67 & 0.01 \\
27  & 2 & 0.68 & 0.01 \\
442 & 4 & 1.04 & 0.01 \\
492 & 4 & 1.10 & 0.01 \\
542 & 4 & 1.16 & 0.01 \\
581 & 4 & 1.21 & 0.01 \\
639 & 4 & 1.26 & 0.01 \\
674 & 4 & 1.31 & 0.01 \\
708 & 4 & 1.36 & 0.01 \\
744 & 4 & 1.38 & 0.01 \\
\bottomrule
\end{tabular}
	\caption{Messdaten zum Strahlprofil im Resonator der Länge $L = \SI{795 +- 4}{\milli\metre}$}
	\label{tab:strahlradius_80}
\end{table}
\begin{figure}[b]
	\centering
	\input{./plots/strahlradien/strahlradius_80cm.tex}
	\caption{Anpassung des Strahlprofils an die Messdaten für die Resonatorlänge $L = \SI{795 +- 4}{\milli\metre}$}
	\label{fig:strahlradius_80}
\end{figure}
\FloatBarrier
\begin{table}[t]
	\centering
	\begin{tabular}{SSSS}
\toprule
{$z \, / \, \si{\milli\metre}$} & {$\Delta z \, / \, \si{\milli\metre}$} & {$d \, / \, \si{\milli\metre}$} & {$\Delta d \, / \, \si{\milli\metre}$} \\
\midrule
7   & 2 & 0.59 & 0.02 \\
12  & 2 & 0.60 & 0.02 \\
35  & 2 & 0.63 & 0.02 \\
454 & 4 & 1.09 & 0.02 \\
474 & 4 & 1.13 & 0.02 \\
497 & 4 & 1.16 & 0.02 \\
543 & 4 & 1.21 & 0.02 \\
581 & 4 & 1.26 & 0.02 \\
628 & 4 & 1.34 & 0.02 \\
669 & 4 & 1.40 & 0.02 \\
768 & 4 & 1.54 & 0.02 \\
\bottomrule
\end{tabular}
	\caption{Messdaten zum Strahlprofil im Resonator der Länge $L = \SI{892 +- 4}{\milli\metre}$}
	\label{tab:strahlradius_90}
\end{table}
\begin{figure}[b]
	\centering
	\input{./plots/strahlradien/strahlradius_90cm.tex}
	\caption{Anpassung des Strahlprofils an die Messdaten für die Resonatorlänge $L = \SI{892 +- 4}{\milli\metre}$}
	\label{fig:strahlradius_90}
\end{figure}
\clearpage

\subsection{Bestimmung der Modenabstände mit dem optischen Spektrumanalysator}
\begin{figure}[ht]
\centering
\resizebox{0.75\textwidth}{!}{
\begin{tabular}{@{}lcccc|cc@{}}
\toprule
Nummer $i$   & $x_i$ / \si{\milli\second} & $\Delta(x_i)$ / \si{\milli\second} & $\Delta T$ / \si{\milli\second} & $\Delta(\Delta T)$ / \si{\milli\second} & $T_\text{konf.}$ / \si{\milli\second} & $\Delta T_\text{konf.}$ / \si{\milli\second} \\ \midrule
1  & 6,222  & 0,007 & \multirow{2}{*}{0,739} & \multirow{2}{*}{0,008} & 4,624 & 0,011 \\
2  & 6,961  & 0,002 & \multirow{2}{*}{0,708} & \multirow{2}{*}{0,003} & 4,471 & 0,003 \\
3  & 7,669  & 0,002 & \multirow{2}{*}{0,671} & \multirow{2}{*}{0,003} & 4,318 & 0,003 \\
4  & 8,340  & 0,003 & \multirow{2}{*}{0,653} & \multirow{2}{*}{0,003} & 4,219 & 0,003 \\
5  & 8,993  & 0,003 & \multirow{2}{*}{0,649} & \multirow{2}{*}{0,003} & 4,137 & 0,003 \\
6  & 9,642  & 0,003 & \multirow{2}{*}{0,650} & \multirow{2}{*}{0,003} & 4,059 & 0,003 \\
7  & 10,292 & 0,004 &       &       & 4,018 & 0,005 \\
8  & 10,847 & 0,009 & \multirow{2}{*}{0,586} & \multirow{2}{*}{0,009} &       &       \\
9  & 11,433 & 0,002 & \multirow{2}{*}{0,554} & \multirow{2}{*}{0,003} &       &       \\
10 & 11,987 & 0,002 & \multirow{2}{*}{0,573} & \multirow{2}{*}{0,003} &       &       \\
11 & 12,560 & 0,002 & \multirow{2}{*}{0,570} & \multirow{2}{*}{0,003} &       &       \\
12 & 13,130 & 0,002 & \multirow{2}{*}{0,572} & \multirow{2}{*}{0,003} &       &       \\
13 & 13,701 & 0,003 & \multirow{2}{*}{0,608} & \multirow{2}{*}{0,005} &       &       \\
14 & 14,310 & 0,004 &       &       &       &       \\
Mittelwert:   &        &       & 0,624 & 0,001 & 4,236 & 0,002 \\ \bottomrule
\end{tabular}}
\caption{Resonatorlänge 70cm ALL0026}
\label{fig:ALL0026}
\end{figure}
\begin{table}[hb]
\centering
\resizebox{\columnwidth}{!}{%
\begin{tabular}{@{}lcccc|cc@{}}
\toprule
Nummer $i$   & $x_i$ / \si{\milli\second} & $\Delta(x_i)$ / \si{\milli\second} & $\Delta T$ / \si{\milli\second} & $\Delta(\Delta T)$ / \si{\milli\second} & $T_\text{konf.}$ / \si{\milli\second} & $\Delta T_\text{konf.}$ / \si{\milli\second} \\ \midrule
1  & 6,222  & 0,007 & \multirow{2}{*}{0,739} & \multirow{2}{*}{0,008} & 4,624 & 0,011 \\
2  & 6,961  & 0,002 & \multirow{2}{*}{0,708} & \multirow{2}{*}{0,003} & 4,471 & 0,003 \\
3  & 7,669  & 0,002 & \multirow{2}{*}{0,671} & \multirow{2}{*}{0,003} & 4,318 & 0,003 \\
4  & 8,340  & 0,003 & \multirow{2}{*}{0,653} & \multirow{2}{*}{0,003} & 4,219 & 0,003 \\
5  & 8,993  & 0,003 & \multirow{2}{*}{0,649} & \multirow{2}{*}{0,003} & 4,137 & 0,003 \\
6  & 9,642  & 0,003 & \multirow{2}{*}{0,650} & \multirow{2}{*}{0,003} & 4,059 & 0,003 \\
7  & 10,292 & 0,004 &       &       & 4,018 & 0,005 \\
8  & 10,847 & 0,009 & \multirow{2}{*}{0,586} & \multirow{2}{*}{0,009} &       &       \\
9  & 11,433 & 0,002 & \multirow{2}{*}{0,554} & \multirow{2}{*}{0,003} &       &       \\
10 & 11,987 & 0,002 & \multirow{2}{*}{0,573} & \multirow{2}{*}{0,003} &       &       \\
11 & 12,560 & 0,002 & \multirow{2}{*}{0,570} & \multirow{2}{*}{0,003} &       &       \\
12 & 13,130 & 0,002 & \multirow{2}{*}{0,572} & \multirow{2}{*}{0,003} &       &       \\
13 & 13,701 & 0,003 & \multirow{2}{*}{0,608} & \multirow{2}{*}{0,005} &       &       \\
14 & 14,310 & 0,004 &       &       &       &       \\
Mittelwert:   &        &       & 0,624 & 0,001 & 4,236 & 0,002 \\ \bottomrule
\end{tabular}}
\caption{Messwerte und Berechnung zum longitudinalen Modenabstand. $x_i$ beschreibt die Position der Maxima, $\Delta T$ die Differenz zweier aufeinander folgender Positionen und $T_\text{konf.}$ den Abstand zweier gleicher Tranversalmoden. Dieser ist getrennt dargestellt, da die Werte keinen zeilenweisen Bezug zu den $x_i$ haben.}
\end{table}
\clearpage
\begin{figure}[ht]
\centering
\resizebox{0.75\textwidth}{!}{
\begin{tabular}{@{}lcccc|cc@{}}
\toprule
Nummer $i$   & $x_i$ / \si{\milli\second} & $\Delta(x_i)$ / \si{\milli\second} & $\Delta T$ / \si{\milli\second} & $\Delta(\Delta T)$ / \si{\milli\second} & $T_\text{konf.}$ / \si{\milli\second} & $\Delta T_\text{konf.}$ / \si{\milli\second} \\ \midrule
1  & 4,118  & 0,003 & \multirow{2}{*}{0,513} & \multirow{2}{*}{0,003} & 3,840 & 0,004 \\
2  & 4,631  & 0,002 & \multirow{2}{*}{0,501} & \multirow{2}{*}{0,003} & 3,734 & 0,003 \\
3  & 5,131  & 0,002 & \multirow{2}{*}{0,491} & \multirow{2}{*}{0,003} & 3,648 & 0,003 \\
4  & 5,622  & 0,002 & \multirow{2}{*}{0,484} & \multirow{2}{*}{0,003} & 3,582 & 0,003 \\
5  & 6,107  & 0,002 & \multirow{2}{*}{0,475} & \multirow{2}{*}{0,003} & 3,527 & 0,003 \\
6  & 6,582  & 0,002 & \multirow{2}{*}{0,488} & \multirow{2}{*}{0,005} & 3,474 & 0,003 \\
7  & 7,070  & 0,005 &       &       & 3,429 & 0,009 \\
8  & 7,958  & 0,003 & \multirow{2}{*}{0,407} & \multirow{2}{*}{0,004} &       &       \\
9  & 8,365  & 0,002 & \multirow{2}{*}{0,414} & \multirow{2}{*}{0,003} &       &       \\
10 & 8,779  & 0,002 & \multirow{2}{*}{0,426} & \multirow{2}{*}{0,003} &       &       \\
11 & 9,205  & 0,002 & \multirow{2}{*}{0,429} & \multirow{2}{*}{0,003} &       &       \\
12 & 9,634  & 0,002 & \multirow{2}{*}{0,422} & \multirow{2}{*}{0,003} &       &       \\
13 & 10,055 & 0,002 & \multirow{2}{*}{0,444} & \multirow{2}{*}{0,008} &       &       \\
14 & 10,499 & 0,008 &       &       &       &       \\
Mittelwert:   &        &       & 0,457 & 0,001 & 3,624 & 0,002 \\ \bottomrule
\end{tabular}}
\caption{Resonatorlänge 80cm ALL0022}
\label{fig:ALL0022}
\end{figure}
\begin{table}[hb]
\centering
\resizebox{\columnwidth}{!}{%
\begin{tabular}{@{}lcccc|cc@{}}
\toprule
Nummer $i$   & $x_i$ / \si{\milli\second} & $\Delta(x_i)$ / \si{\milli\second} & $\Delta T$ / \si{\milli\second} & $\Delta(\Delta T)$ / \si{\milli\second} & $T_\text{konf.}$ / \si{\milli\second} & $\Delta T_\text{konf.}$ / \si{\milli\second} \\ \midrule
1  & 4,118  & 0,003 & \multirow{2}{*}{0,513} & \multirow{2}{*}{0,003} & 3,840 & 0,004 \\
2  & 4,631  & 0,002 & \multirow{2}{*}{0,501} & \multirow{2}{*}{0,003} & 3,734 & 0,003 \\
3  & 5,131  & 0,002 & \multirow{2}{*}{0,491} & \multirow{2}{*}{0,003} & 3,648 & 0,003 \\
4  & 5,622  & 0,002 & \multirow{2}{*}{0,484} & \multirow{2}{*}{0,003} & 3,582 & 0,003 \\
5  & 6,107  & 0,002 & \multirow{2}{*}{0,475} & \multirow{2}{*}{0,003} & 3,527 & 0,003 \\
6  & 6,582  & 0,002 & \multirow{2}{*}{0,488} & \multirow{2}{*}{0,005} & 3,474 & 0,003 \\
7  & 7,070  & 0,005 &       &       & 3,429 & 0,009 \\
8  & 7,958  & 0,003 & \multirow{2}{*}{0,407} & \multirow{2}{*}{0,004} &       &       \\
9  & 8,365  & 0,002 & \multirow{2}{*}{0,414} & \multirow{2}{*}{0,003} &       &       \\
10 & 8,779  & 0,002 & \multirow{2}{*}{0,426} & \multirow{2}{*}{0,003} &       &       \\
11 & 9,205  & 0,002 & \multirow{2}{*}{0,429} & \multirow{2}{*}{0,003} &       &       \\
12 & 9,634  & 0,002 & \multirow{2}{*}{0,422} & \multirow{2}{*}{0,003} &       &       \\
13 & 10,055 & 0,002 & \multirow{2}{*}{0,444} & \multirow{2}{*}{0,008} &       &       \\
14 & 10,499 & 0,008 &       &       &       &       \\
Mittelwert:   &        &       & 0,457 & 0,001 & 3,624 & 0,002 \\ \bottomrule
\end{tabular}}
\caption{Messwerte und Berechnung zum longitudinalen Modenabstand. $x_i$ beschreibt die Position der Maxima, $\Delta T$ die Differenz zweier aufeinander folgender Positionen und $T_\text{konf.}$ den Abstand zweier gleicher Tranversalmoden. Dieser ist getrennt dargestellt, da die Werte keinen zeilenweisen Bezug zu den $x_i$ haben.}
\end{table}
\clearpage
\begin{figure}[ht]
\centering
\resizebox{0.75\textwidth}{!}{
\begin{tabular}{@{}lcccc|cc@{}}
\toprule
Nummer $i$   & $x_i$ / \si{\milli\second} & $\Delta(x_i)$ / \si{\milli\second} & $\Delta T$ / \si{\milli\second} & $\Delta(\Delta T)$ / \si{\milli\second} & $T_\text{konf.}$ / \si{\milli\second} & $\Delta T_\text{konf.}$ / \si{\milli\second} \\ \midrule
1  & 4,754  & 0,006 & \multirow{2}{*}{0,508} & \multirow{2}{*}{0,007} & 5,096 & 0,040 \\
2  & 5,262  & 0,003 & \multirow{2}{*}{0,461} & \multirow{2}{*}{0,003} & 4,056 & 0,003 \\
3  & 5,723  & 0,002 & \multirow{2}{*}{0,454} & \multirow{2}{*}{0,003} & 3,996 & 0,003 \\
4  & 6,177  & 0,002 & \multirow{2}{*}{0,472} & \multirow{2}{*}{0,003} & 3,937 & 0,003 \\
5  & 6,649  & 0,002 & \multirow{2}{*}{0,476} & \multirow{2}{*}{0,003} & 3,888 & 0,003 \\
6  & 7,124  & 0,002 & \multirow{2}{*}{0,442} & \multirow{2}{*}{0,003} & 3,844 & 0,010 \\
7  & 7,567  & 0,010 &       &       & 3,841 & 0,021 \\
8  & 9,849  & 0,040 & \multirow{2}{*}{0,531} & \multirow{2}{*}{0,040} &       &       \\
9  & 9,318  & 0,004 & \multirow{2}{*}{0,401} & \multirow{2}{*}{0,003} &       &       \\
10 & 9,719  & 0,003 & \multirow{2}{*}{0,395} & \multirow{2}{*}{0,004} &       &       \\
11 & 10,114 & 0,002 & \multirow{2}{*}{0,423} & \multirow{2}{*}{0,003} &       &       \\
12 & 10,537 & 0,002 & \multirow{2}{*}{0,431} & \multirow{2}{*}{0,010} &       &       \\
13 & 10,968 & 0,003 & \multirow{2}{*}{0,439} & \multirow{2}{*}{0,023} &       &       \\
14 & 11,408 & 0,021 &       &       &       &       \\
Mittelwert:   &        &       & 0,448 & 0,001 & 3,957 & 0,002 \\ \bottomrule
\end{tabular}}
\caption{Resonatorlänge 90cm ALL0014}
\label{fig:ALL0014}
\end{figure}
\begin{table}[hb]
\centering
\resizebox{\columnwidth}{!}{%
\begin{tabular}{@{}lcccc|cc@{}}
\toprule
Nummer $i$   & $x_i$ / \si{\milli\second} & $\Delta(x_i)$ / \si{\milli\second} & $\Delta T$ / \si{\milli\second} & $\Delta(\Delta T)$ / \si{\milli\second} & $T_\text{konf.}$ / \si{\milli\second} & $\Delta T_\text{konf.}$ / \si{\milli\second} \\ \midrule
1  & 4,754  & 0,006 & \multirow{2}{*}{0,508} & \multirow{2}{*}{0,007} & 5,096 & 0,040 \\
2  & 5,262  & 0,003 & \multirow{2}{*}{0,461} & \multirow{2}{*}{0,003} & 4,056 & 0,003 \\
3  & 5,723  & 0,002 & \multirow{2}{*}{0,454} & \multirow{2}{*}{0,003} & 3,996 & 0,003 \\
4  & 6,177  & 0,002 & \multirow{2}{*}{0,472} & \multirow{2}{*}{0,003} & 3,937 & 0,003 \\
5  & 6,649  & 0,002 & \multirow{2}{*}{0,476} & \multirow{2}{*}{0,003} & 3,888 & 0,003 \\
6  & 7,124  & 0,002 & \multirow{2}{*}{0,442} & \multirow{2}{*}{0,003} & 3,844 & 0,010 \\
7  & 7,567  & 0,010 &       &       & 3,841 & 0,021 \\
8  & 9,849  & 0,040 & \multirow{2}{*}{0,531} & \multirow{2}{*}{0,040} &       &       \\
9  & 9,318  & 0,004 & \multirow{2}{*}{0,401} & \multirow{2}{*}{0,003} &       &       \\
10 & 9,719  & 0,003 & \multirow{2}{*}{0,395} & \multirow{2}{*}{0,004} &       &       \\
11 & 10,114 & 0,002 & \multirow{2}{*}{0,423} & \multirow{2}{*}{0,003} &       &       \\
12 & 10,537 & 0,002 & \multirow{2}{*}{0,431} & \multirow{2}{*}{0,010} &       &       \\
13 & 10,968 & 0,003 & \multirow{2}{*}{0,439} & \multirow{2}{*}{0,023} &       &       \\
14 & 11,408 & 0,021 &       &       &       &       \\
Mittelwert:   &        &       & 0,448 & 0,001 & 3,957 & 0,002 \\ \bottomrule
\end{tabular}}
\caption{Messwerte und Berechnung zum longitudinalen Modenabstand. $x_i$ beschreibt die Position der Maxima, $\Delta T$ die Differenz zweier aufeinander folgender Positionen und $T_\text{konf.}$ den Abstand zweier gleicher Tranversalmoden. Dieser ist getrennt dargestellt, da die Werte keinen zeilenweisen Bezug zu den $x_i$ haben.}
\end{table}
\clearpage


\end{appendix}

\end{document}
