% PAKETE UND DOKUMENTKONFIGURATION
\documentclass[11pt, a4paper]{article}

% Encoding für Umlaute
\usepackage[utf8]{inputenc}
\usepackage[T1]{fontenc}

% Silbentrennung
\usepackage[english]{babel}

% erweiterte Matheumgebungen und Formelnummer mit Sectionnummer
\usepackage{amsmath}
\numberwithin{equation}{section}

% Braket Notation
\usepackage{braket}
\usepackage{isotope}
\usepackage{mhchem}

% zusätzliche mathematische Schriftarten
\usepackage{amsfonts}

% verschiedene mathematische Symbole
\usepackage{amssymb}

% Einheiten setzen z.B. \SI{10}{\kilo\gram\meter\per\second\squared}
% Fehler: \SI{10 +- 0,2e-4}{\metre}
\usepackage{siunitx}
\sisetup{
  separate-uncertainty
}

% Einheitendefinitionen
\DeclareSIUnit{\skt}{Skt.}
\DeclareSIUnit{\gauss}{G}

% Operatordefinitionen
\DeclareMathOperator{\erf}{erf}

% Randbreiten
\usepackage[left=3.5cm,right=3.5cm,top=3cm,bottom=3cm,twoside]{geometry}

% Bilder einfügen
\usepackage{graphicx}

% Verweise innerhalb des Dokuments
\usepackage{hyperref}
\hypersetup{
	colorlinks = true,
	allcolors = {black}
}

% bessere Tabellenlayouts
\usepackage{booktabs}
\usepackage{multirow}
\usepackage{multicol}

% Seitenlayout (Kopfzeile)
\usepackage{fancyhdr}

% Float Barriers
\usepackage{placeins}

% Pakete für gedrehte Subfigures
\usepackage{caption}
\usepackage{subcaption}
\usepackage{rotating}

% Paket für textumflossene Abbildungen und Tabellen
\usepackage{wrapfig}

\usepackage{float}

% Caption-Setup
\captionsetup{font={small}}
\renewcommand{\thefigure}{\thesection.\arabic{figure}}
\renewcommand{\thesubfigure}{\alph{subfigure}}
\renewcommand{\thetable}{\thesection.\arabic{table}}
\renewcommand{\thesubtable}{\alph{subtable}}

% Manuelle Silbentrennung
\hyphenation{Re-so-na-tor Mo-den-ab-stand Re-so-na-tor-län-ge Spek-t-ro-me-ter}

% Tiefe des Inhaltsverzeichnisses (Level: 1 sections, 2 subsections,
% 3 subsubsections)
\setcounter{tocdepth}{3}

% FANCYHDR SETUP
\pagestyle{fancy}
\fancyhead[EL,OR]{\thepage}
\fancyhead[ER]{\leftmark}
\fancyhead[OL]{\rightmark}

\renewcommand{\sectionmark}[1]{
\markboth{\thesection{} #1}{\thesection{} #1}
}
\renewcommand{\subsectionmark}[1]{
\markright{\thesubsection{} #1}
}

% DOKUMENTINFORMATIONEN
\title{Warsaw\\Orbit Determination of 2004 BL86}

\author{Christopher Deutsch\footnote{christopher.deutsch@uni-bonn.de} \and Christian Bespin\footnote{christian.bespin@uni-bonn.de}}

\date{\today}

\begin{document}

\begin{titlepage}

\maketitle

% DURCHFÜHRUNGSDATUM UND TUTOR
\begin{center}
\begin{tabular}{l r}
Execution: & May 25, 2015 \\
Group: & $\alpha$ 6 \\
Tutor: & Rui Zhang
\end{tabular}
\end{center}

% ZUSAMMENFASSUNG
\begin{abstract}
\noindent 
This report shall be concerned with determining the orbital parameters of the asteroid 2004 B86 from terestial observation with the Pi in the Sky telescope in Chile.
\end{abstract}

\end{titlepage}

% INHALTSVERZEICHNIS
\tableofcontents
% Neue Seite nach TOC
\newpage

% INHALT VERSUCHSPROTOKOLL

\section{Introduction}
In the night of January 26/27, 2015 a near-earth approach of the asteroid 2004 BL86 took place and was observed with the telescope \emph{Pi in the Sky} in Chile, San Pedro de Atacama.\\
Observatory location:
\begin{itemize}
	\item longitudude: \SI{-68.18000000}{\degree}
	\item latitude: \SI{-22.95332778}{\degree}
	\item altitude: \SI{2430}{\meter}
\end{itemize}
The asteroid was visible on more than a thousand photos which where shot with an exposure time of \SI{10}{s}, which where shot from 3:10 to 6:30 UT.
After some preprocessing of the collected images the position of the asteroid could be determined using stationary reference stars.

\section{Theory}

\subsection{Equatorial Coordinate System}
\begin{figure}[h]
	\centering
	\includegraphics[width=0.7\textwidth]{./figures/equatorial_cs.png}
	\caption{Credit: \url{https://commons.wikimedia.org/wiki/File:Ra_and_dec_on_celestial_sphere.png}}
\end{figure}
The equatorial coordinate system is an astronomical coordinate system using earth's equator as its reference plane (as opposed to the ecliptic coordinate system using the ecliptic for reference).
The position of an object on the celestial sphere is then defined with two angles $\alpha$ (right ascension) and $\delta$ (declination).
The declination~$\delta$ measures the height of the object from the celestial equator (projection of earth's equator on the celestial sphere), where the north celestial pole has declination $+\SI{90}{\degree}$.
The right ascension $\alpha$ is the angular distance of the object's projection on the equator plane to the vernal equinox, which is the intersection of ecliptic and equatorial plane.
It is because of this reference point, that the equatorial coordinate system is independent of earth's rotation.

\subsection{Orbital Elements}
\begin{figure}[h]
	\centering
	\includegraphics[width=0.75\textwidth]{./figures/orbital_elements.pdf}
	\caption{Illustration of the classical orbital elements of an arbitrary celestial body. Credit: \url{https://en.wikipedia.org/wiki/File:Orbit1.svg}}
\end{figure}


Keplarian Elements:
\begin{itemize}
	\item $a$: length of the semimajor axis of the ellipse
	
	\item $e$: eccentricity of the ellipse
	
	\item $i$: inclination
	
	\item $\Omega$: right ascension of the ascending node
	
	\item $\omega$: angular distance of the periapsis from the ascending node
	
	\item $\nu$: true anomaly defining the position of the orbiting body at a certain point in time
\end{itemize}


\section{Analysis}

Correction of exposure time\\
Transforming of Coordinates:
\begin{itemize}
	\item Determining earths position at time of observation $\nu$ true anomaly! using keplarian elements (wiki)
	
	\item Determining the position of the observatory on earths surface
\end{itemize}
Hypothesis of asteroid parameters for fit\\
Propagation of the Orbit with regards to time\\
Determining observation angles (on earth) from asteroid position\\
Minimizing $\chi^2$




\section{Conclusion}


\FloatBarrier
% BIBLIOGRAPHIE
\vspace{\fill}
% Maximale Anzahl der Einträge in Klammer
% Zitieren mit \cite{lamport94}
\begin{thebibliography}{19}

\bibitem{krane}
	Kenneth S. Krane,
	\emph{Introductory Nuclear Physics},
	John Wiley \& Sons 1988

\end{thebibliography}

% APPENDIX
\begin{appendix}
\section{Appendix}

\end{appendix}

\end{document}
