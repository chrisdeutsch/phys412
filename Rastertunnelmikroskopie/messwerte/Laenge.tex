% Table generated by Excel2LaTeX from sheet 'L�nge'
\begin{table}[htbp]
  \centering
    \begin{tabular}{SSSSS}
    \toprule
    {Liniennummer} & {L\"ange / \si{\angstrom}} &       & {Liniennummer} & {L\"ange / \si{\angstrom}} \\
    \midrule
    1     & 1,48  &       & 40    & 1,45 \\
    2     & 1,13  &       & 41    & 1,41 \\
    3     & 0,98  &       & 42    & 0,94 \\
    4     & 1,52  &       & 43    & 1,21 \\
    5     & 1,45  &       & 44    & 1,45 \\
    6     & 1,37  &       & 45    & 0,94 \\
    7     & 0,86  &       & 46    & 1,29 \\
    8     & 1,41  &       & 47    & 0,82 \\
    9     & 0,94  &       & 48    & 1,45 \\
    10    & 1,56  &       & 49    & 1,17 \\
    11    & 0,94  &       & 50    & 0,78 \\
    12    & 1,25  &       & 51    & 1,13 \\
    13    & 0,86  &       & 52    & 1,68 \\
    14    & 1,33  &       & 53    & 0,86 \\
    15    & 1,37  &       & 54    & 1,09 \\
    16    & 1,48  &       & 55    & 1,60 \\
    17    & 1,41  &       &       &  \\
    18    & 0,94  &       &       &  \\
    19    & 1,33  &       &       &  \\
    20    & 0,78  &       &       &  \\
    21    & 1,52  &       &       &  \\
    22    & 1,33  &       &       &  \\
    23    & 1,25  &       &       &  \\
    24    & 0,86  &       &       &  \\
    25    & 1,17  &       &       &  \\
    26    & 1,52  &       &       &  \\
    27    & 1,56  &       &       &  \\
    28    & 1,41  &       &       &  \\
    29    & 1,25  &       &       &  \\
    30    & 0,74  &       &       &  \\
    31    & 1,37  &       &       &  \\
    32    & 1,25  &       &       &  \\
    33    & 0,98  &       &       &  \\
    34    & 1,33  &       &       &  \\
    35    & 1,33  &       &       &  \\
    36    &       &       &       &  \\
    37    & 0,86  &       &       &  \\
    38    & 1,25  &       &       &  \\
    39    & 0,86  &       &       &  \\
    \bottomrule
    \end{tabular}%
  \caption{L\"ange der eingezeichneten Linien}
  \label{tab:laengen}%
\end{table}%
