% PAKETE UND DOKUMENTKONFIGURATION
\documentclass[11pt, a4paper]{article}

% Encoding für Umlaute
\usepackage[utf8]{inputenc}
\usepackage[T1]{fontenc}

% Silbentrennung
\usepackage[ngerman]{babel}

% erweiterte Matheumgebungen und Formelnummer mit Sectionnummer
\usepackage{amsmath}
\numberwithin{equation}{section}

% Braket Notation
\usepackage{braket}

% zusätzliche mathematische Schriftarten
\usepackage{amsfonts}

% verschiedene mathematische Symbole
\usepackage{amssymb}

% Einheiten setzen z.B. \SI{10}{\kilo\gram\meter\per\second\squared}
% Fehler: \SI{10 +- 0,2e-4}{\metre}
\usepackage{siunitx}
\sisetup{
  output-decimal-marker={,},
  separate-uncertainty
}

% Einheitendefinitionen
\DeclareSIUnit{\dBm}{dBm}

% Operatordefinitionen
\DeclareMathOperator{\erf}{erf}

% Randbreiten
\usepackage[left=3.5cm,right=3.5cm,top=3cm,bottom=3cm,twoside]{geometry}

% Bilder einfügen
\usepackage{graphicx}

% Verweise innerhalb des Dokuments
\usepackage{hyperref}
\hypersetup{
	colorlinks = true,
	allcolors = {black}
}

% bessere Tabellenlayouts
\usepackage{booktabs}
\usepackage{multirow}

% Seitenlayout (Kopfzeile)
\usepackage{fancyhdr}

% Float Barriers
\usepackage{placeins}

% Pakete für gedrehte Subfigures
\usepackage{caption}
\usepackage{subcaption}
\usepackage{rotating}

% Caption-Setup
\captionsetup{font={small}}
\renewcommand{\thefigure}{\thesection.\arabic{figure}}
\renewcommand{\thesubfigure}{\alph{subfigure}}
\renewcommand{\thetable}{\thesection.\arabic{table}}
\renewcommand{\thesubtable}{\alph{subtable}}

% Manuelle Silbentrennung
\hyphenation{Re-so-na-tor Mo-den-ab-stand Re-so-na-tor-län-ge}

% Tiefe des Inhaltsverzeichnisses (Level: 1 sections, 2 subsections,
% 3 subsubsections)
\setcounter{tocdepth}{3}

% FANCYHDR SETUP
\pagestyle{fancy}
\fancyhead[EL,OR]{\thepage}
\fancyhead[ER]{\leftmark}
\fancyhead[OL]{\rightmark}

\renewcommand{\sectionmark}[1]{
\markboth{\thesection{} #1}{\thesection{} #1}
}
\renewcommand{\subsectionmark}[1]{
\markright{\thesubsection{} #1}
}

% DOKUMENTINFORMATIONEN
\title{P523 \\ Beta Spektrometer}

\author{Christopher Deutsch\footnote{christopher.deutsch@uni-bonn.de} \and Christian Bespin\footnote{christian.bespin@uni-bonn.de}}

\date{\today}

\begin{document}

\begin{titlepage}

\maketitle

% DURCHFÜHRUNGSDATUM UND TUTOR
\begin{center}
\begin{tabular}{l r}
Durchführung: & 07./08. April 2015 \\
Gruppe: & $\alpha$ 6 \\
Tutor: & Yannick Wunderlich
\end{tabular}
\end{center}

% ZUSAMMENFASSUNG
\begin{abstract}
\noindent
\end{abstract}

\end{titlepage}

% INHALTSVERZEICHNIS
\tableofcontents
% Neue Seite nach TOC
\newpage

% INHALT VERSUCHSPROTOKOLL

\section{Einführung}


\section{Theorie}
\subsection{$\beta$-Zerfall}
\begin{itemize}
	\item schwache WW
	\item Elementarvertex?
\end{itemize}

\subsubsection{Zerfallsarten}
Man unterscheidet zwischen drei verschiedenen $\beta$-Zerfallsarten ($m(A,Z)$: Atommasse, $Q$: Zerfallsenergie):
\begin{itemize}
	\item \textbf{Beta-Plus-Zerfall ($\beta^+$):}
	\begin{align}
	\mathrm{p} \to \mathrm{n} + \mathrm{e}^+ + \nu_\mathrm{e}
	\end{align}
	\begin{align}
		Q_{\beta^+} = m( A, Z ) - m( A, Z - 1) - 2 m_\mathrm{e}
	\end{align}
	
	\item \textbf{Beta-Minus-Zerfall ($\beta^-$):}
	\begin{align}
	\mathrm{n} \to \mathrm{p} + \mathrm{e}^- + \overline{\nu}_\mathrm{e}
	\end{align}
	\begin{align}
	Q_{\beta^-} = m( A, Z ) - m( A, Z + 1)
	\end{align}
	
	\item \textbf{Elektroneneinfang ($\epsilon$):}
	\begin{align}
	\mathrm{p} + \mathrm{e}^- \to \mathrm{n} + \nu_\mathrm{e}
	\end{align}
	\begin{align}
	Q_{\epsilon} = m( A, Z ) - m( A, Z - 1) - E_\mathrm{Bind.}
	\end{align}
	Nach dem Einfang befindet sich das Atom in einem angeregten Zustand, sodass zusätlich die Bindungsenergie des eingefangenen Elektrons abgezogen werden muss.
\end{itemize}
Grundsätzlich ist ein Zerfall nur möglich, wenn die Zerfallsenergie $Q$ positiv ist (Energieerhaltung).

\subsubsection{Fermitheorie des $\beta$-Zerfalls}
\begin{itemize}
	\item Einführung: historische Relevanz des kontinuierlichen Spektrums (Dreikörperzerfall und kein Zweikörperzerfall)
	\item Spektrum
	\item erlaubte Übergänge
	\item Kurieplot
	\item verbotene Übergänge
	\item Einfluß des Coulombfeldes
\end{itemize}

\subsubsection{Innere Konversion und Augereffekt}

\subsection{Spektrometer}

\subsubsection{Grundlagen}

\subsubsection{Überblick über Spektrometertypen (?)}

\subsubsection{Semicircular Spectrometer}

\subsubsection{Doppelt-fokussierendes Spektrometer}

\subsection{Detektoren (?)}
\subsubsection{Halbleiterdetektoren}

\subsubsection{Hallsonde}

\subsubsection{Hysterese (?)}


\section{Durchführung und Auswertung}
Die ausführliche Durchführung ist der Versuchsanleitung \cite{anleitung} zu entnehmen.
Sollten Abweichungen bei der Durchführung auftreten, so werden diese im jeweiligen Unterkapitel dargestellt.

\subsection{Aufbau des Helium-Neon Experimentierlasers}

\subsection{Bestimmung der Wellenlänge des Lasers mit einem Reflexionsgitter}
\subsubsection{Durchführung}

\subsubsection{Auswertung}
\subsubsection{Vergleich mit der exakten Rechnung}
\subsection{Untersuchung der Polarisation des Lasers}
\subsubsection{Polarisationsgrad des Lasers}

\subsection{Messung des Strahlprofils und des Stabilitätsgebiets des Lasers}
\subsubsection{Messmethode}
\subsubsection{Zusammenhang zwischen Spaltbreite am Messschieber und Strahlradius}
\subsubsection{Experimentelle Bestimmung des Strahlprofils}

\subsection{Aufbau der optischen Diode}
\subsection{Optischer Spektrumanalysator}

\subsubsection{Durchführung}
\subsubsection{Messwerte}

\subsubsection{Bestimmung der Modenabstände}

\subsubsection{Ergebnisse und Fazit}

\subsection{Präzise Messung des Modenabstandes mittels einer optischen Schwebung}

\subsubsection{Mathematische Beschreibung der Funktionsweise eines Mischers}

\subsubsection{Messung des Modenabstandes durch Mischung mit der Hochfrequenz}

\subsection{Modenspektrum des Helium-Neon-Lasers}

\section{Fazit}

\clearpage
% BIBLIOGRAPHIE
\vspace{\fill}
% Maximale Anzahl der Einträge in Klammer
% Zitieren mit \cite{lamport94}
\begin{thebibliography}{19}
\bibitem{javan}
	A. Javan, W. R. Bennett, Jr., and D. R. Herriott,
	\emph{Population Inversion and Continuous Optical Maser Oscillation in a Gas Discharge Containing a He-Ne Mixture},
	Phys. Rev. Lett. 6, 106 – Published 1 February 1961

\bibitem{linden}
	S. Linden,
	Skript zur Vorlesung \emph{physik311: Optik und Wellenmechanik} (Stand: 31. Januar 2014),
	Physikalisches Institut, Universität Bonn

\bibitem{siegman}
	A. E. Siegman,
	\emph{Lasers},
	University Science Books 1986,
	Chapter 19: Stable Two-Mirror Resonators

\bibitem{anleitung}
	Physikalisches Praktikum IV: Atome, Moleküle, Festkörper,
	Versuchsbeschreibung \emph{P442: Laser} (Stand: 12. September 2014),
	Universität Bonn
	
\bibitem{messschieber_katalog}
	\emph{Mitutoyo Messgeräte-Katalog},
	ABSOLUTE AOS Digimatic Messschieber,\\
	\url{http://www2.mitutoyo.de/ebooks/german/handmessgeraete/index.html} (Letzter Abruf: 22. Dezember 2014)	

\bibitem{schawlow}
	A. L. Schawlow,
	\emph{Measuring the Wavelength of Light with a Ruler},
	Am. J. Phys., Volume 33, Issue 11 (1965)

\bibitem{NISTSpectra}
	Kramida, A., Ralchenko, Yu., Reader, J., and NIST ASD Team (2014).
	\emph{NIST Atomic Spectra Database} (ver. 5.2).
	\url{http://physics.nist.gov/asd} (Letzter Abruf: 18. Dezember 2014).
	National Institute of Standards and Technology, Gaithersburg, MD.
	
\bibitem{horowitz_hill}
	Paul Horowitz, Winfred Hill,
	\emph{The Art of Electronics Second Edition},
	Cambridge University Press 1989,
	Chapter 13.12: High frequency and high-speed techniques -- Radiofrequency circuit elements

\bibitem{iupac_periodic_table}
	International Union of Pure and Applied Chemistry,
	\emph{IUPAC Periodic Table of the Elements} (Stand: 1. Mai 2013),
	\url{http://iupac.org/reports/periodic_table/}

\bibitem{meschede}
	Dieter Meschede,
	\emph{Optik, Licht und Laser} (3. Auflage),
	Vieweg Teubner 2008
	
	
 
\end{thebibliography}

\clearpage

% APPENDIX
\begin{appendix}
\section{Anhang}
\subsection{Messwerte der Photospannung hinter dem Polarisator}

\subsection{Bestimmung des Strahlprofils im Resonator}

\subsection{Bestimmung der Modenabstände mit dem optischen Spektrumanalysator}

\end{appendix}

\end{document}
