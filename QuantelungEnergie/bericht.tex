% PAKETE UND DOKUMENTKONFIGURATION
\documentclass[11pt, a4paper]{article}

% Encoding für Umlaute
\usepackage[utf8]{inputenc}
\usepackage[T1]{fontenc}

% Silbentrennung
\usepackage[ngerman]{babel}

% erweiterte Matheumgebungen
\usepackage{amsmath}

% Braket Notation
\usepackage{braket}

% zusätzliche mathematische Schriftarten
\usepackage{amsfonts}

% verschiedene mathematische Symbole
\usepackage{amssymb}

% Einheiten setzen z.B. \SI{10}{\kilo\gram\meter\per\second\squared}
% Fehler: \SI{10 +- 0,2e-4}{\metre}
\usepackage{siunitx}
\sisetup{
  output-decimal-marker={,},
  separate-uncertainty
}

% Randbreiten
\usepackage[left=3.5cm,right=3.5cm,top=3cm,bottom=3cm,twoside]{geometry}

% Bilder einfügen
\usepackage{graphicx}

% Verweise innerhalb des Dokuments
\usepackage{hyperref}
\hypersetup{
	colorlinks = true,
	allcolors = {black}
}

% bessere Tabellenlayouts
\usepackage{booktabs}

% Seitenlayout (Kopfzeile)
\usepackage{fancyhdr}

% Float Barriers
\usepackage{placeins}

% Pakete für gedrehte Subfigures
\usepackage{caption}
\usepackage{subcaption}
\usepackage{rotating}

% Manuelle Silbentrennung
\hyphenation{Halb-werts-brei-te Fa-bry-Pé-rot-E-ta-lon Zee-man-E-ffekt Cad-mi-um-Lam-pe Kon-den-sor-lin-se}

% Tiefe des Inhaltsverzeichnisses (Level: 1 sections, 2 subsections,
% 3 subsubsections)
\setcounter{tocdepth}{3}

% FANCYHDR SETUP
\pagestyle{fancy}
\fancyhead[EL,OR]{\thepage}
\fancyhead[ER]{\leftmark}
\fancyhead[OL]{\rightmark}

\renewcommand{\sectionmark}[1]{
\markboth{Abschnitt \thesection{}: #1}{\thesection{} #1}
}
\renewcommand{\subsectionmark}[1]{
\markright{\thesubsection{} #1}
}

% DOKUMENTINFORMATIONEN
\title{P402 \\ Quantelung von Energie}

\author{Christopher Deutsch\footnote{christopher.deutsch@uni-bonn.de} \and Christian Bespin\footnote{christian.bespin@uni-bonn.de}}

\date{\today}

\begin{document}

\begin{titlepage}

\maketitle

% DURCHFÜHRUNGSDATUM UND ASSISTENT
\begin{center}
\begin{tabular}{l r}
Durchführung: & 1./2. Dezember 2014 \\
Gruppe: & $\alpha$ 2 \\
Assistent: & Dennis Proft
\end{tabular}
\end{center}

% ZUSAMMENFASSUNG
\begin{abstract}
\noindent

\end{abstract}

\end{titlepage}

% INHALTSVERZEICHNIS
\tableofcontents
% Neue Seite nach TOC
\newpage

% INHALT VERSUCHSPROTOKOLL

\section{Grundlagen / Theorie}






\clearpage

% BIBLIOGRAPHIE

% Maximale Anzahl der Einträge in Klammer
% Zitieren mit \cite{lamport94}
\begin{thebibliography}{9}

\bibitem{hecht}
	Eugene Hecht,
	\emph{Optik}.
	Oldenbourg,
	5. Auflage
	
\bibitem{siegmann}
	Anthony E. Siegmann,
	\emph{Lasers}.
	University Science Books,
	1986
	
\bibitem{demtroeder3}
	Wolfgang Demtröder,
	\emph{Experimentalphysik 3}.
	Springer Verlag,
	3. Auflage

\bibitem{np_richardson}
 Nobelprize.org,
 \emph{"The Nobel Prize in Physics 1928"}.
 Nobel Media AB 2014. Web. 15\\
 (\url{http://www.nobelprize.org/nobel_prizes/physics/laureates/1928/richardson-lecture.pdf} abgerufen am 15.11.2014)
 
\bibitem{haken_wolf}
	Hermann Haken, Hans Christoph Wolf,
	\emph{Atom- und Quantenphysik}.
	Springer Verlag,
	7. Auflage
 
\end{thebibliography}

% APPENDIX
\begin{appendix}

\end{appendix}

\end{document}
