% PAKETE UND DOKUMENTKONFIGURATION
\documentclass[10pt, a4paper]{article}

% Encoding für Umlaute
\usepackage[utf8]{inputenc}
\usepackage[T1]{fontenc}

% Silbentrennung
\usepackage[ngerman]{babel}

% erweiterte Matheumgebungen
\usepackage{amsmath}

% zusätzliche mathematische Schriftarten
\usepackage{amsfonts}

% verschiedene mathematische Symbole
\usepackage{amssymb}

% Einheiten setzen z.B. \SI{10}{\kilo\gram\meter\per\second\squared}
% Fehler: \SI{10 +- 0,2e-4}{\metre}
\usepackage{siunitx}
\sisetup{
  output-decimal-marker={,},
  separate-uncertainty
}

% Randbreiten
\usepackage[left=3.5cm,right=3.5cm,top=3cm,bottom=3cm,twoside]{geometry}

% Bilder einfügen
\usepackage{graphicx}

% Verweise innerhalb des Dokuments
\usepackage{hyperref}
\hypersetup{
	colorlinks = true,
	allcolors = {black}
}

% bessere Tabellenlayouts
\usepackage{booktabs}

% Tiefe des Inhaltsverzeichnisses (Level: 1 sections, 2 subsections,
% 3 subsubsections)
\setcounter{tocdepth}{2}

% manuelle Angabe zur Silbentrennung (mehrere Wörter mit Leerzeichen in {} schreiben)
\hyphenation{Rönt-gen-strah-lung}

% DOKUMENTINFORMATIONEN
\title{P428 \\ Röntgenstrahlung und Materialanalyse}

\author{Christopher Deutsch\footnote{christopher.deutsch@uni-bonn.de} \and Christian Bespin\footnote{christian.bespin@uni-bonn.de}}

\date{\today}

\begin{document}

\maketitle

% DURCHFÜHRUNGSDATUM UND ASSISTENT
\begin{center}
\begin{tabular}{l r}
Durchführung: & 3./4. November 2014 \\
Gruppe: & $\alpha$ 2 \\
Assistent: & Peter Klassen
\end{tabular}
\end{center}

% ZUSAMMENFASSUNG
\begin{abstract}
\noindent
% Text
\end{abstract}

% INHALTSVERZEICHNIS
\tableofcontents
% Neue Seite nach TOC
\newpage

% INHALT VERSUCHSPROTOKOLL

\section{Grundlagen}
\subsection{Röntgenstrahlung}
  Als Röntgenstrahlung wird der Teil des elektromagnetischen Spektrums mit Energien zwischen einigen keV und mehreren MeV \cite{demtroeder} bezeichnet.
  Sie entsteht durch Beschuss geeigneter Kristalle mit Elektronen, die zuvor eine Beschleunigungsstrecke (in diesem Versuch \SI{35}{\kilo\volt}) durchlaufen haben.
  Die Elektronen erzeugen dabei auf zwei verschiedene Arten Röntgenstrahlung:
  \begin{itemize}
  \item \textbf{Bremsstrahlung} entsteht durch Streuung der Elektronen am Kristall, da diese bei jeder Geschwindigkeitsänderung Energie in Form von Strahlung abgeben.
  Das entstehende Spektrum ist kontinuierlich und wird hauptsächlich durch die Beschleunigungsspannung begrenzt.
  \item \textbf{charakteristische Strahlung} entsteht dadurch, dass die beschleunigten Elektronen im Bohrmodell ein Elektron auf einer inneren Schale des Atoms herauslöst.
  Der so freigewordene Platz auf dieser Schale wird nun durch ein Elektron aus einer weiter außen liegenden Schale aufgefüllt, welches dabei Energie in Form von Röntgenstrahlung abgibt.
  Diese Energie entspricht der Energiedifferenz der Elektronen in den am Sprung beteiligten Schalen und ist charakteristisch für jedes Element.
  Das so entstandene Spektrum ist darum diskret.
  \end{itemize}
Zum Nachweis von Röntgenstrahlung werden in diesem Versuch Geiger-Müller-Zählrohre, Röntgenenergiedetektor und Röntgenfilm verwendet.
Der Röntgenfilm ist mit einer speziellen Beschichtung, die bei Auftreffen von Röntgenstrahlung Licht aussendet, versehen, welches wiederum eine lichtempfindliche Emulsion belichtet und so ein Bild erzeugt.
Die Funktionsweise der beiden anderen Nachweismethoden wird weiter unten erklärt.

\begin{itemize}
  \item Feinstruktur (char. Spektrum?)
\end{itemize}

\subsection{Bragg-Reflexion}

\begin{itemize}
  \item Röntgenbeugung
  \item Bragg-Bedingung
  \item Glanzwinkel
\end{itemize}
 
\subsection{Röntgenfluoreszenz}
  Als Röntgenfluoreszenz bezeichnet man die Entstehung sekundärer, fluoreszierender Röntgenstrahlung.
  Dies geschieht ähnlich wie die Erzeugung charakteristischer Röntgenstrahlung, nur dass nun bereits erzeugte Röntgenstrahlung anstatt den energiereichen Elektronen zur Bestrahlung eines Elements verwendet werden.
  Diese sekundäre Röntgenstrahlung weist wieder ein charakteristisches Spektrum auf, aus dem man auf das beschossene Element schließen kann.
  Das Verfahren zur Elementbestimmung (oder Anteilen an Elementen in dem Target (?)) nennt sich daher \textbf{Röntgenfluoreszenzanalyse}.
  
\begin{itemize}
  \item Bestimmung der Massenanteile einzelner Komponenten
\end{itemize}
  
\subsection{Laue-Verfahren}

\begin{itemize}
  \item Laue--Bedingung
  \item Millersche Indizes (reziprokes Gitter)
  \item Elementarzelle
  \item Glanzwinkel
  \item Netzebenenabstand
\end{itemize}  

\subsection{Geiger-Müller-Zählrohr}

\begin{itemize}
  \item Aufbau
  \item Funktionsweise
  \item Totzeit
\end{itemize}
  
\subsection{Röntgenenergiedetektor}

\begin{itemize}
  \item Aufbau
  \item Funktionsweise
  \item PIN-Photodiode
  \item Vielkanalanalysator
\end{itemize}



\section{Versuchsdurchführung}

\subsection{Versuchsbeschreibung}

\subsection{Aufbau}

\subsubsection{Hinweis zu einem der Geräte oder whatever}

\section{Messdaten}

\section{Auswertung}

\section{Diskussion}

\section{Zusammenfassung}

% BIBLIOGRAPHIE

% Maximale Anzahl der Einträge in Klammer
% Zitieren mit \cite{lamport94}
\begin{thebibliography}{9}

% Beispiel
\bibitem{lamport94}
  Leslie Lamport,
  \emph{\LaTeX: a document preparation system}.
  Addison Wesley, Massachusetts,
  2nd edition,
  1994.
  
\bibitem{demtroeder}
	Wolfgang Demtröder,
	\emph{Experimentalphysik 3}.
	Springer Verlag,
	3. Auflage
\end{thebibliography}


\newpage

% APPENDIX
\begin{appendix}

\end{appendix}

\end{document}
