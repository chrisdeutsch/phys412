% PAKETE UND DOKUMENTKONFIGURATION
\documentclass[11pt, a4paper]{article}

% Encoding für Umlaute
\usepackage[utf8]{inputenc}
\usepackage[T1]{fontenc}

% Silbentrennung
\usepackage[ngerman]{babel}

% erweiterte Matheumgebungen und Formelnummer mit Sectionnummer
\usepackage{amsmath}
\numberwithin{equation}{section}

% Braket Notation
\usepackage{braket}
\usepackage{isotope}

% zusätzliche mathematische Schriftarten
\usepackage{amsfonts}

% verschiedene mathematische Symbole
\usepackage{amssymb}

% Einheiten setzen z.B. \SI{10}{\kilo\gram\meter\per\second\squared}
% Fehler: \SI{10 +- 0,2e-4}{\metre}
\usepackage{siunitx}
\sisetup{
  output-decimal-marker={,},
  separate-uncertainty
}

% Einheitendefinitionen
\DeclareSIUnit{\skt}{Skt.}

% Operatordefinitionen
\DeclareMathOperator{\erf}{erf}

% Randbreiten
\usepackage[left=3.5cm,right=3.5cm,top=3cm,bottom=3cm,twoside]{geometry}

% Bilder einfügen
\usepackage{graphicx}

% Verweise innerhalb des Dokuments
\usepackage{hyperref}
\hypersetup{
	colorlinks = true,
	allcolors = {black}
}

% bessere Tabellenlayouts
\usepackage{booktabs}
\usepackage{multirow}
\usepackage{multicol}

% Seitenlayout (Kopfzeile)
\usepackage{fancyhdr}

% Float Barriers
\usepackage{placeins}

% Pakete für gedrehte Subfigures
\usepackage{caption}
\usepackage{subcaption}
\usepackage{rotating}

% Caption-Setup
\captionsetup{font={small}}
\renewcommand{\thefigure}{\thesection.\arabic{figure}}
\renewcommand{\thesubfigure}{\alph{subfigure}}
\renewcommand{\thetable}{\thesection.\arabic{table}}
\renewcommand{\thesubtable}{\alph{subtable}}

% Manuelle Silbentrennung
\hyphenation{Re-so-na-tor Mo-den-ab-stand Re-so-na-tor-län-ge}

% Tiefe des Inhaltsverzeichnisses (Level: 1 sections, 2 subsections,
% 3 subsubsections)
\setcounter{tocdepth}{3}

% FANCYHDR SETUP
\pagestyle{fancy}
\fancyhead[EL,OR]{\thepage}
\fancyhead[ER]{\leftmark}
\fancyhead[OL]{\rightmark}

\renewcommand{\sectionmark}[1]{
\markboth{\thesection{} #1}{\thesection{} #1}
}
\renewcommand{\subsectionmark}[1]{
\markright{\thesubsection{} #1}
}

% DOKUMENTINFORMATIONEN
\title{P523 \\ Beta Spektrometer}

\author{Christopher Deutsch\footnote{christopher.deutsch@uni-bonn.de} \and Christian Bespin\footnote{christian.bespin@uni-bonn.de}}

\date{\today}

\begin{document}

\begin{titlepage}

\maketitle

% DURCHFÜHRUNGSDATUM UND TUTOR
\begin{center}
\begin{tabular}{l r}
Durchführung: & 07./08. April 2015 \\
Gruppe: & $\alpha$ 6 \\
Tutor: & Yannick Wunderlich
\end{tabular}
\end{center}

% ZUSAMMENFASSUNG
\begin{abstract}
\noindent
\end{abstract}

\end{titlepage}

% INHALTSVERZEICHNIS
\tableofcontents
% Neue Seite nach TOC
\newpage

% INHALT VERSUCHSPROTOKOLL

\section{Einführung}

In diesem Praktikumsversuch werden Spektren und Maximalenergie der emittierten Teilchen verschiedener $\beta$-Strahler vermessen bzw. bestimmt.
Der $\beta$-Zerfall stellt eine Form der Kernumwandlung dar und sendet dabei Elektronen oder Positronen aus, die ein kontinuierliches Energiespektrum aufweisen, welches häufig auch diskrete, charakteristische Linien aufweist, die zur Kalibration des verwendeten Spektrometers genutzt werden.

\section{Theorie}
\subsection{$\beta$-Zerfall}
\begin{itemize}
	\item schwache WW
	\item Elementarvertex?
\end{itemize}

\subsubsection{Zerfallsarten}
Man unterscheidet zwischen drei verschiedenen $\beta$-Zerfallsarten ($m(A,Z)$: Atommasse, $Q$: Zerfallsenergie):
\begin{itemize}
	\item \textbf{Beta-Plus-Zerfall ($\beta^+$):}
	\begin{align}
	\mathrm{p} \to \mathrm{n} + \mathrm{e}^+ + \nu_\mathrm{e}
	\end{align}
	\begin{align}
		Q_{\beta^+} = m( A, Z ) - m( A, Z - 1) - 2 m_\mathrm{e}
	\end{align}
	
	\item \textbf{Beta-Minus-Zerfall ($\beta^-$):}
	\begin{align}
	\mathrm{n} \to \mathrm{p} + \mathrm{e}^- + \overline{\nu}_\mathrm{e}
	\end{align}
	\begin{align}
	Q_{\beta^-} = m( A, Z ) - m( A, Z + 1)
	\end{align}
	
	\item \textbf{Elektroneneinfang ($\epsilon$):}
	\begin{align}
	\mathrm{p} + \mathrm{e}^- \to \mathrm{n} + \nu_\mathrm{e}
	\end{align}
	\begin{align}
	Q_{\epsilon} = m( A, Z ) - m( A, Z - 1) - E_\mathrm{Bind.}
	\end{align}
	Nach dem Einfang befindet sich das Atom in einem angeregten Zustand, sodass zusätlich die Bindungsenergie des eingefangenen Elektrons abgezogen werden muss.
\end{itemize}
Grundsätzlich ist ein Zerfall nur möglich, wenn die Zerfallsenergie $Q$ positiv ist (Energieerhaltung).

\subsubsection{Fermitheorie des $\beta$-Zerfalls}
Fermi lieferte 1934 den ersten theoretischen Ansatz zur Beschreibung des $\beta$-Zerfalls, nachdem von Pauli bereits die Existenz eines dritten Teilchens, des Neutrinos gefordert worden war.
Das beobachtete Zerfallsspektrum zeigte eine kontinuierliche Energieverteilung mit einer oberen Grenze, die als Differenz zwischen Ausgangs- und Endzustand verstanden werden kann.
Wenn der $\beta$-Zerfall als Zweikörperzerfall betrachtet wird, erwartet man für alle $\beta$-Teilchen die gleiche Energie (die gerade dieser Differenz entspricht).\cite{krane}
In der beobachteten Energieverteilung fehlte also immer ein Beitrag, der später dem Neutrino zugeordnet wurde und man begann, den Zerfall als Dreikörperzerfall zu betrachten.
\\
\\
Fermi beschreibt die Wahrscheinlichkeit, dass ein Teilchen mit Impuls $p$ emittiert wird als quantenmechanische Übergangswahrscheinlichkeit, die sich aus der Störungstheorie ergibt.
Für ein Elektron, das im Impulsintervall zwischen $p$ und $p+\text{d}p$ emittiert wird, ist die Wahrscheinlichkeit pro Zeiteinheit dann gegeben als \cite{mayer-kuckuk}:
\begin{align}
	N(p)\text{d}p = \frac{2\pi}{\hbar}\left|\Braket{f|H|i}\right|^2\frac{\text{d}n}{\text{d}E_0}
\end{align}
\begin{itemize}
	\item Einführung: historische Relevanz des kontinuierlichen Spektrums (Dreikörperzerfall und kein Zweikörperzerfall)
	\item Spektrum
	\item erlaubte Übergänge
	\item Kurieplot
	\item verbotene Übergänge
	\item Einfluß des Coulombfeldes
\end{itemize}

Spektrum nach Krane:
\begin{align}
	N(p) \propto p^2 \left( Q - T_\mathrm{e} \right)^2 F(Z, p) \left| M_{fi} \right|^2 S(p, q)
\end{align}

Aufpassen bei der Definition von $\epsilon_0$ im Kurie-Plot von Riezler-Kopitzki.
Da kann gut und gerne mal $\pm \SI{511}{\kilo\electronvolt}$ rauskommen wenn man die falsche Definition verwendet.

Man definiert:
\begin{align}
	\epsilon = 1 + \frac{E_\mathrm{kin}}{m_\mathrm{e} c^2} \\
	\eta = \frac{p}{m_\mathrm{e} c} \\
	\epsilon_0 = 1 + \frac{Q}{m_\mathrm{e} c^2} \\
\end{align}
Unsinnige Definition $\epsilon_0$ aber damit folgt Riezler-Kopitzki Form.

\begin{align}
	N(\epsilon) = N(p) \frac{\mathrm{d}p}{\mathrm{d}\epsilon}\propto N(p) \frac{\epsilon}{\eta}
\end{align}

\subsubsection{Innere Konversion und Augereffekt}

\subsection{Das $\beta$-Spektrometer}
Das Spektrometer ist ein Instrument, welches es ermöglicht, eine Vielzahl von Teilchen nach gewissen Eigenschaften wie zum Beispiel Energie, Impuls oder Masse zu trennen und dabei die Anzahl der auftretenden Teilchen in einem gewissen (Energie-, Impuls-, Massen-) Intervall zu zählen.
Den resultierenden Graphen nennt man ein Spektrum.

\subsubsection{Grundlagen}
In diesem Versuch wird ein magnetisches Spektrometer verwendet, welches die Ablenkung von geladenen Teilchen im Magnetfeld ausnutzt.
Betrachtet man ein Elektron, welches sich in einem homogenen Magnetfeld $\vec{B}$ mit der Geschwindigkeit $\vec{v}$ senkrecht zum Magnetfeld bewegt, so wirkt die Lorentzkraft als Zentripetalkraft und das Elektron beschreibt eine Kreisbahn mit Radius $\rho$.
\begin{align}
	F_\mathrm{Z} &\stackrel{!}{=} F_\mathrm{L} \nonumber\\
	e v B &= \frac{\gamma m_\mathrm{e} v^2}{\rho}
\end{align}
Mit dem relativistischem Impuls $p = \gamma m v$ folgt sofort:
\begin{align}
	p = e B \rho
	\label{eq:impuls_radius}
\end{align}
Man sieht, dass die Größe $B \rho$ charakteristisch für den Impuls ist und gleichzeitig eine Aussage über die Dimension des Spektrometers macht.
Daher werden Impulse in der $\beta$-Spektroskopie als $B \rho$-Werte angegeben.

(Kram für eventuell verwendete einheitenlose Größen $\epsilon$, $\eta$, wasweißich)

\begin{align}
	B \rho = \frac{1}{c e} \sqrt{T \left( 2 E_0 + T \right)}
	\label{eq:b_rho}
\end{align}

\subsubsection{Halbkreisförmiges Spektrometer}
(GRAFIK)
Ein halbkreisförmiges Spektrometer nutzt ein homogenes Magnetfeld senkrecht zur Flugrichtung der Elektronen um diese auf eine Kreisbahn zu lenken.
Nach Gleichung \eqref{eq:impuls_radius} ist das Produkt von Magnetfeld $B$ und Bahnradius $\rho$ charakteristisch für den Impuls des Elektrons.

Nachteil nur Fokussierung in der $\rho$-Ebene führt zu geringer Transmission.
Vorteil: einfache Konstruktion, Magnetfeld leicht zu messen, gut geeignet um große Spektralbereiche aufzunehmen (konstante Magnetfeldstärke)

\subsubsection{Doppelt-fokussierendes Spektrometer}
(GRAFIK)
Betatron-Oszillationen ($\omega_0$: Elektronen Umlauffrequenz) (Zitat Hillert)
\begin{align}
	\omega_\rho = \omega_0 \sqrt{1-n} \qquad \omega_z = \omega_0 \sqrt{n}
\end{align}
mit dem Feldindex:
\begin{align}
	n = - \frac{\rho}{B_z(\rho)} \frac{\mathrm{d} B_z(\rho)}{\mathrm{d} \rho}
\end{align}
Während einer halben Periode der jeweiligen Betatronschwingung läuft das Elektron auf der Kreisbahn einen Winkel von:
\begin{align}
	\phi_\rho &= \omega_0 \cdot \frac{\pi}{\omega_\rho} & \phi_z &= \omega_0 \cdot \frac{\pi}{\omega_z}\\
	\phi_\rho &= \pi \left[ 1+ \frac{\rho_0 B^\prime(\rho_0)}{B(\rho_0)} \right]^{-\frac{1}{2}} & \phi_z &= \pi \left[ -\frac{\rho_0 B^\prime(\rho_0)}{B(\rho_0)} \right]^{-\frac{1}{2}} \label{eq:betatronwinkel}
\end{align}
ab, wobei $\rho_0$ den Designorbit kennzeichnet.
An Gleichung \eqref{eq:betatronwinkel} ließt man nach quadrieren ab:
\begin{align}
	\frac{1}{\phi_\rho^2} + \frac{1}{\phi_z^2} = \frac{1}{\pi^2} \label{eq:betatronwinkelverhaeltnis}
\end{align}
Um eine stigmatische Abbildung zu erlangen, das heißt der Fokus der $\rho$ und der $z$-Ebene liegen im selben Punkt, muss die Bedingung $\phi_\rho = \phi_z$ gestellt werden.
Damit folgt aus Gleichung \eqref{eq:betatronwinkelverhaeltnis} der gesamte Winkel der Elektronenbahn im Spektrometer.
Dieser ergibt sich zu $\phi = \pi \sqrt{2}$, weshalb das doppelt-fokussierende Spektrometer auch $\pi \sqrt{2}$-Spektrometer genannt wird.
Gleichzeitig folgt aus der gestellten Anforderung $\phi_\rho = \phi_z$ mit Gleichung \eqref{eq:betatronwinkel} die Bestimmungsgleichung für das magnetische Feld:
\begin{align}
	B^\prime(\rho) = - \frac{1}{2 \rho} B(\rho)
\end{align}
Mit der Randbedingung $B(\rho_0) = B_0$ folgt als allgemeine Lösung:
\begin{align}
	B(\rho) = B_0 \sqrt{\frac{\rho_0}{\rho}}
\end{align}
Der Vorteil gegenüber dem halbkreisförmigen Spektrometer liegt darin, dass das doppelt-fokussierende ebenfalls in der $z$-Ebene fokussierend wirkt und damit höhere Transmissionen möglich sind.

\subsection{Detektoren (?)}
\subsubsection{Halbleiterdetektoren}

\subsubsection{Hallsonde}

\subsubsection{Hysterese (?)}


\section{Durchführung und Auswertung}
Die ausführliche Durchführung ist der Versuchsanleitung \cite{anleitung} zu entnehmen.
Sollten Abweichungen bei der Durchführung auftreten, so werden diese im jeweiligen Unterkapitel dargestellt.

\subsection{Bestimmung der Offsetspannung der Hallsonde}

Zur Messung des Magnetfelds des Spektrometers wird eine Hallsonde verwendet, die jedoch auch bei ausgeschaltetem Magnetfeld eine Spannung $\neq 0$ \si{\volt} anzeigt.
Um die tatsächliche Offsetspannung zu ermitteln, nutzt man die Hysterese des Magneten aus und misst für beide Stromrichtungen, die am Magneten angelegt werden die jeweiligen Remanenzfelder (als Hallspannung $U_\text{H}$) gemessen.
\begin{figure}[h]
	\centering
	\includegraphics[width=0.5\textwidth]{./figures/hysterese.pdf}
	\caption{Skizzierte Hysteresekurve. Der tatsächliche Spannungswert für $I=\SI{0}{\ampere}$ ergibt sich aus Mittelung von $U_\text{+}$ und $U_\text{-}$.}
	\label{fig:hysterese}
\end{figure}
Bei jeder Messung wird dann das arithmetische Mittel (mit $U_0$ bezeichnet) von $U_\text{+}$ und $U_\text{-}$ berechnet und im Anschluss über alle $U_0$ wieder gemittelt, um einen guten Wert für die Offsetspannung $U_0$ zu erhalten.
\begin{table}[h]
	\centering
	\begin{tabular}{SSSSSSS}
\toprule
{$U_+$ / \si{\skt}} & {$\Delta U_+$ / \si{\skt}} & {$U_-$ / \si{\skt}} & {$\Delta U_-$ / \si{\skt}} &  & {$U_0$ / \si{\skt}} & {$\Delta U_0$ / \si{\skt}} \\ \midrule
9.4 & 0.1    & 5.0 & 0.1    &  & 7.20   & 0.08   \\
9.4 & 0.1    & 5.0 & 0.1    &  & 7.20   & 0.08   \\
9.3 & 0.1    & 5.0 & 0.1    &  & 7.15   & 0.08   \\
9.3 & 0.1    & 5.0 & 0.1    &  & 7.15   & 0.08   \\
9.3 & 0.1    & 4.9 & 0.1    &  & 7.10   & 0.08   \\
9.3 & 0.1    & 5.0 & 0.1    &  & 7.15   & 0.08   \\ \midrule
    &        &     & {$\overline{U_0}$ / \si{\skt}:}    &  & 7.16   & 0.04   \\ \bottomrule
\end{tabular}
	\caption{Messwerte und Auswertung zur Bestimmung der Offsetspannung der Hallsonde. Die unterste Zeile ergibt sich aus Mittelung über alle $U_0$.}
	\label{tab:offset}
\end{table}
Der Fehler der Mittelung über $U_\text{+}$ und $U_\text{-}$ ergibt sich mit Gauß'scher Fehlerfortpflanzung aus
\begin{align}
	\Delta U_0 = \sqrt{(\Delta U_\text{+})^2 + (\Delta U_\text{-})^2}
\end{align}
Den im weiteren Verlauf benutzten Wert für die Offsetspannung $\overline{U_0}$ erhält man, indem über alle $U_0$ gemittelt wird. Da die Fehler auf $U_0$ gleich groß sind, ergibt sich für $\Delta\overline{U_0}$
\begin{align}
	\Delta\overline{U_0} = \frac{\Delta U_0}{\sqrt{6}}
\end{align}
und alle gemessenen Hallspannungen werden in der folgenden Auswertung um
\begin{align}
	\overline{U_0} = \SI{7.16 +- 0.03}{\skt}
	\label{hall_korrektur}
\end{align}
korrigiert.

\subsection{Kalibration des Spektrometers}
Zur Kalibration des Spektrometers wird das Spektrum von \isotope[137]{Cs} bei Transmissionen von \SI{1}{\percent} und \SI{4}{\percent} vermessen.
Von Interesse sind dabei die Konversionslinien des angeregten Isotops \isotope[137]{Ba}, welche zur Impulskalibrierung genutzt werden können.
Der angeregte Zustand von \isotope[137]{Ba} hat eine Zerfallsenergie von
\begin{align*}
	Q_\gamma = \SI{661,660}{\kilo\electronvolt}\text{.}
\end{align*}
Bei innerer Konversion überträgt sich diese Energie auf ein kernnahes Elektron, welches aus dem Atom gelöst wird.
Nach der Ionisation hat das Elektron eine kinetische Energie von
\begin{align}
	T = Q_\gamma - E_\mathrm{B} \text{,}
	\label{eq:kin_energie_konversion}
\end{align}
wobei $E_\mathrm{B}$ die Bindungsenergie des jeweiligen Elektrons im Atom bezeichnet.
Im Spektrum sind zwei Konversionslinien zu erkennen, welche zur K- und L-Schale von Barium zuzuordnen sind.
Weiterhin muss beachtet werden, dass im Magnetfeld des Spektrometer eine Zeemanaufspaltung der Niveaus der L-Schale zu beobachten ist.
Zunächst soll die Konversionslinie des Elektrons der K-Schale mit der Bindungsenergie
\begin{align*}
E_\mathrm{K} = \SI{37,441}{\kilo\electronvolt}
\end{align*}
betrachtet werden.
Mit den Gleichungen \eqref{eq:b_rho} und \eqref{eq:kin_energie_konversion} kann der Impuls des Elektrons in der für die Spektroskopie üblichen Größe
\begin{align*}
	\left(B \rho \right)_\mathrm{K} = \SI{3379}{G.cm}
\end{align*}
angegeben werden.

Schließlich soll die Zeemanaufspaltung der L-Schale betrachtet werden, welche sich in drei diskrete Energieniveaus aufspaltet.
Für die Bindungsenergie der aufgespalteten Niveaus ist gegeben:
\begin{align*}
E_{\mathrm{L}_{\mathrm{I}}} = \SI{5,987}{\kilo\electronvolt} \quad
E_{\mathrm{L}_{\mathrm{II}}} = \SI{5,624}{\kilo\electronvolt} \quad
E_{\mathrm{L}_{\mathrm{III}}} = \SI{5,247}{\kilo\electronvolt}
\end{align*}
Da die Aufspaltung mit dem Spektroskop nicht auflösbar ist, muss eine sinnvolle Mittlung durchgeführt werden.
Dazu betrachtet man die Entartungsgrade der einzelnen Niveaus
\begin{align*}
g_{\mathrm{L}_{\mathrm{I}}} = \num{1} \quad
g_{\mathrm{L}_{\mathrm{II}}} = \num{2} \quad
g_{\mathrm{L}_{\mathrm{III}}} = \num{1}
\end{align*}
und führt eine dementsprechend gewichtete Mittlung durch:
\begin{align*}
E_\mathrm{L} = \frac{E_{\mathrm{L}_{\mathrm{I}}} + 2E_{\mathrm{L}_{\mathrm{II}}} + E_{\mathrm{L}_{\mathrm{III}}}}{4} \approx \SI{5,621}{\kilo\electronvolt}
\end{align*}
In Analogie zur K-Schale kann der Impuls des Elektrons berechnet werden:
\begin{align}
\left(B \rho \right)_\mathrm{L} = \SI{3497}{G.cm}
\end{align}

\subsubsection{Untergrundmessungen}

Zur späteren Bereinigung der Messwerte werden vorher Untergrundmessungen durchgeführt.
Für den Untergrund des Caesium-Zerfalls bei 1\% und 4\% Transmission sind die in Tabelle \ref{tab:untergrund_cs} notierten Werte für die Counts $N$ gemessen worden.
\begin{table}[h]
	\centering
	\begin{tabular}{SS}
\toprule
{$N$ (1\% Transm.)}  & {$N$ (4\% Transm.)}   \\ \midrule
 21     & 6         \\
 24     & 8         \\
 21     & 10        \\
 21     & 8         \\
 20     & 5         \\
 21     & 13        \\
 18     & 9         \\
 15     & 3         \\
 22     & 9         \\
18     & 9         \\\midrule
\multicolumn{2}{c}{{Mittelwerte $N_0$:}} \\
{\num{20.1+-2.6}} & {\num{8+-2.8}} \\ \bottomrule
\end{tabular}

	\caption{Untergrundmessung von \isotope[137]{Cs} bei 1\% und 4\% Transmission}
	\label{tab:untergrund_cs}
\end{table}
Die Messwerte wurden dabei in der untersten Zeile gemittelt und der Fehler aus der Standardabweichung berechnet.

\subsubsection{Grobspektrum von Barium}
Zunächst wurde grob das Spektrum von Barium aufgenommen, um die ungefähren Positionen der Konversionslinien zu finden.
Dieses ist in Abbildung \ref{fig:ba_t4_grob} dargestellt. 
Um die Dispersion des Spektrometers zu berücksichtigen wurden die gemessenen und bereits korrigierten Zählraten wie in der Praktikumsanleitung erwähnt, durch die korrigierte Hallspannung dividiert werden, so dass das gemessene Impulsintervall proportional zum Magnetfeld ist.
\begin{figure}[h]
	\centering
	\input{./plots/barium_t4_grob.tex}
	\caption{Barium Grobspektrum Transmission 4}
	\label{fig:ba_t4_grob}
\end{figure}
Die korrigierten Hallspannungen erhält man mit \eqref{hall_korrektur} und Anwenden von:
\begin{align}
U_\mathrm{H}^\mathrm{Korr} &= U_\mathrm{H} - U_0 \\
\Delta U_\mathrm{H}^\mathrm{Korr} &= \sqrt{\Delta U_\mathrm{H}^2 + \Delta U_0^2} \qquad \text{(Gauß'sche Fehlerfortpflanzung)}
\end{align}
Für den Fehler der Hallspannung wurde dabei
\begin{align}
	\Delta U_\mathrm{H} = \SI{0.1}{\skt}
\end{align}
angenommen, was der letzten ablesbaren Stelle der Digitalanzeige entspricht.
Da die Zählraten poissonverteilt sind, kann für den Fehler
\begin{align}
	\Delta N = \sqrt{N}
\end{align}
angenommen werden.
Für die Bereinigung der Zählraten mit den oben gefundenen Werten für den Hintergrund gilt
\begin{align}
	N_\text{Korr = }N - N_0 \qquad \Delta N_\text{Korr} = \sqrt{\Delta N^2 + \Delta N_0^2}
\end{align}
wobei erneut die Gauß'sche Fehlerfortpflanzung für den Fehler angewendet wurde.
Die oben angesprochene Dispersionskorrektur 
\begin{align}
	n = \frac{N_\text{Korr}}{U_\mathrm{H}^\mathrm{Korr}} \\	
	\text{FEHLER NOCH AUSRECHNEN}
\end{align}
ändert die Zählraten so, dass diese nun als eine Impulsverteilung aufgefasst werden können. 

\begin{figure}[h]
	\centering
	\input{./plots/barium_t4_fein.tex}
	\caption{Barium Feinspektrum Transmission 4}
	\label{fig:ba_t4_fein}
\end{figure}

\begin{figure}[h]
	\centering
	\input{./plots/barium_t1_fein.tex}
	\caption{Barium Feinspektrum Transmission 1}
	\label{fig:ba_t1_fein}
\end{figure}

\begin{figure}[h]
	\centering
	% GNUPLOT: LaTeX picture with Postscript
\begingroup
  \makeatletter
  \providecommand\color[2][]{%
    \GenericError{(gnuplot) \space\space\space\@spaces}{%
      Package color not loaded in conjunction with
      terminal option `colourtext'%
    }{See the gnuplot documentation for explanation.%
    }{Either use 'blacktext' in gnuplot or load the package
      color.sty in LaTeX.}%
    \renewcommand\color[2][]{}%
  }%
  \providecommand\includegraphics[2][]{%
    \GenericError{(gnuplot) \space\space\space\@spaces}{%
      Package graphicx or graphics not loaded%
    }{See the gnuplot documentation for explanation.%
    }{The gnuplot epslatex terminal needs graphicx.sty or graphics.sty.}%
    \renewcommand\includegraphics[2][]{}%
  }%
  \providecommand\rotatebox[2]{#2}%
  \@ifundefined{ifGPcolor}{%
    \newif\ifGPcolor
    \GPcolortrue
  }{}%
  \@ifundefined{ifGPblacktext}{%
    \newif\ifGPblacktext
    \GPblacktexttrue
  }{}%
  % define a \g@addto@macro without @ in the name:
  \let\gplgaddtomacro\g@addto@macro
  % define empty templates for all commands taking text:
  \gdef\gplbacktext{}%
  \gdef\gplfronttext{}%
  \makeatother
  \ifGPblacktext
    % no textcolor at all
    \def\colorrgb#1{}%
    \def\colorgray#1{}%
  \else
    % gray or color?
    \ifGPcolor
      \def\colorrgb#1{\color[rgb]{#1}}%
      \def\colorgray#1{\color[gray]{#1}}%
      \expandafter\def\csname LTw\endcsname{\color{white}}%
      \expandafter\def\csname LTb\endcsname{\color{black}}%
      \expandafter\def\csname LTa\endcsname{\color{black}}%
      \expandafter\def\csname LT0\endcsname{\color[rgb]{1,0,0}}%
      \expandafter\def\csname LT1\endcsname{\color[rgb]{0,1,0}}%
      \expandafter\def\csname LT2\endcsname{\color[rgb]{0,0,1}}%
      \expandafter\def\csname LT3\endcsname{\color[rgb]{1,0,1}}%
      \expandafter\def\csname LT4\endcsname{\color[rgb]{0,1,1}}%
      \expandafter\def\csname LT5\endcsname{\color[rgb]{1,1,0}}%
      \expandafter\def\csname LT6\endcsname{\color[rgb]{0,0,0}}%
      \expandafter\def\csname LT7\endcsname{\color[rgb]{1,0.3,0}}%
      \expandafter\def\csname LT8\endcsname{\color[rgb]{0.5,0.5,0.5}}%
    \else
      % gray
      \def\colorrgb#1{\color{black}}%
      \def\colorgray#1{\color[gray]{#1}}%
      \expandafter\def\csname LTw\endcsname{\color{white}}%
      \expandafter\def\csname LTb\endcsname{\color{black}}%
      \expandafter\def\csname LTa\endcsname{\color{black}}%
      \expandafter\def\csname LT0\endcsname{\color{black}}%
      \expandafter\def\csname LT1\endcsname{\color{black}}%
      \expandafter\def\csname LT2\endcsname{\color{black}}%
      \expandafter\def\csname LT3\endcsname{\color{black}}%
      \expandafter\def\csname LT4\endcsname{\color{black}}%
      \expandafter\def\csname LT5\endcsname{\color{black}}%
      \expandafter\def\csname LT6\endcsname{\color{black}}%
      \expandafter\def\csname LT7\endcsname{\color{black}}%
      \expandafter\def\csname LT8\endcsname{\color{black}}%
    \fi
  \fi
  \setlength{\unitlength}{0.0500bp}%
  \begin{picture}(7200.00,5040.00)%
    \gplgaddtomacro\gplbacktext{%
      \csname LTb\endcsname%
      \put(946,704){\makebox(0,0)[r]{\strut{} 0}}%
      \put(946,1156){\makebox(0,0)[r]{\strut{} 20}}%
      \put(946,1609){\makebox(0,0)[r]{\strut{} 40}}%
      \put(946,2061){\makebox(0,0)[r]{\strut{} 60}}%
      \put(946,2513){\makebox(0,0)[r]{\strut{} 80}}%
      \put(946,2966){\makebox(0,0)[r]{\strut{} 100}}%
      \put(946,3418){\makebox(0,0)[r]{\strut{} 120}}%
      \put(946,3870){\makebox(0,0)[r]{\strut{} 140}}%
      \put(946,4323){\makebox(0,0)[r]{\strut{} 160}}%
      \put(946,4775){\makebox(0,0)[r]{\strut{} 180}}%
      \put(1078,484){\makebox(0,0){\strut{} 0}}%
      \put(1794,484){\makebox(0,0){\strut{} 500}}%
      \put(2509,484){\makebox(0,0){\strut{} 1000}}%
      \put(3225,484){\makebox(0,0){\strut{} 1500}}%
      \put(3941,484){\makebox(0,0){\strut{} 2000}}%
      \put(4656,484){\makebox(0,0){\strut{} 2500}}%
      \put(5372,484){\makebox(0,0){\strut{} 3000}}%
      \put(6087,484){\makebox(0,0){\strut{} 3500}}%
      \put(6803,484){\makebox(0,0){\strut{} 4000}}%
      \put(176,2739){\rotatebox{-270}{\makebox(0,0){\strut{}$U_H$ / \si{\skt}}}}%
      \put(3940,154){\makebox(0,0){\strut{}$B \rho$ / \si{G.cm}}}%
    }%
    \gplgaddtomacro\gplfronttext{%
      \csname LTb\endcsname%
      \put(5816,4602){\makebox(0,0)[r]{\strut{}Messwerte}}%
      \csname LTb\endcsname%
      \put(5816,4382){\makebox(0,0)[r]{\strut{}f(x)}}%
    }%
    \gplbacktext
    \put(0,0){\includegraphics{./plots/kalibration}}%
    \gplfronttext
  \end{picture}%
\endgroup

	\caption{Kalibration}
	\label{fig:kalibration}
\end{figure}
\begin{align}
	\lambda \approx \SI{22.4 +- 0.1}{G.cm\per\skt}
\end{align}

\subsection{Auflösungsvermögen des Spektrometers}

\begin{table}[h]
	\centering
	\begin{tabular}{SS}
\toprule
{Messung} & {N}  \\
\midrule
1  & 6  \\
2  & 8  \\
3  & 10 \\
4  & 8  \\
5  & 5  \\
6  & 13 \\
7  & 9  \\
8  & 3  \\
9  & 9  \\
10 & 9  \\
\midrule
{$\overline{N}$} & {$\Delta N$} \\
8.0 & 2.8 \\
\bottomrule
\end{tabular}
	\caption{Untergrund \isotope[137]{Ba} bei 4}
	\label{tab:untergrund_ba4}
\end{table}
\subsection{Bestimmung der Zerfallsenergie}


\section{Fazit}

\clearpage
% BIBLIOGRAPHIE
\vspace{\fill}
% Maximale Anzahl der Einträge in Klammer
% Zitieren mit \cite{lamport94}
\begin{thebibliography}{19}
\bibitem{krane}
	Kenneth S. Krane,
	\emph{Introductory Nuclear Physics},
	John Wiley \& Sons 1988

\bibitem{linden}
	S. Linden,
	Skript zur Vorlesung \emph{physik311: Optik und Wellenmechanik} (Stand: 31. Januar 2014),
	Physikalisches Institut, Universität Bonn

\bibitem{mayer-kuckuk}
	Theo Mayer-Kuckuk,
	\emph{Kernphysik - Eine Einführung} (7. Auflage),
	Teubner 2002,

\bibitem{anleitung}
	Physikalisches Praktikum V: Kern- und Teilchenphysik,
	Versuchsbeschreibung \emph{P523: $\beta$-Spektrometer} (Stand: Januar 2015),
	Universität Bonn
	
\bibitem{messschieber_katalog}
	\emph{Mitutoyo Messgeräte-Katalog},
	ABSOLUTE AOS Digimatic Messschieber,\\
	\url{http://www2.mitutoyo.de/ebooks/german/handmessgeraete/index.html} (Letzter Abruf: 22. Dezember 2014)	

\bibitem{schawlow}
	A. L. Schawlow,
	\emph{Measuring the Wavelength of Light with a Ruler},
	Am. J. Phys., Volume 33, Issue 11 (1965)

\bibitem{NISTSpectra}
	Kramida, A., Ralchenko, Yu., Reader, J., and NIST ASD Team (2014).
	\emph{NIST Atomic Spectra Database} (ver. 5.2).
	\url{http://physics.nist.gov/asd} (Letzter Abruf: 18. Dezember 2014).
	National Institute of Standards and Technology, Gaithersburg, MD.
	
\bibitem{horowitz_hill}
	Paul Horowitz, Winfred Hill,
	\emph{The Art of Electronics Second Edition},
	Cambridge University Press 1989,
	Chapter 13.12: High frequency and high-speed techniques -- Radiofrequency circuit elements

\bibitem{iupac_periodic_table}
	International Union of Pure and Applied Chemistry,
	\emph{IUPAC Periodic Table of the Elements} (Stand: 1. Mai 2013),
	\url{http://iupac.org/reports/periodic_table/}

\bibitem{meschede}
	Dieter Meschede,
	\emph{Optik, Licht und Laser} (3. Auflage),
	Vieweg Teubner 2008
	
	
 
\end{thebibliography}

\clearpage

% APPENDIX
\begin{appendix}
\section{Anhang}
\subsection{Messwerte der Photospannung hinter dem Polarisator}

\subsection{Bestimmung des Strahlprofils im Resonator}

\subsection{Bestimmung der Modenabstände mit dem optischen Spektrumanalysator}

\end{appendix}

\end{document}
