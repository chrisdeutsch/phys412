% PAKETE UND DOKUMENTKONFIGURATION
\documentclass[10pt, a4paper]{article}

% Encoding für Umlaute
\usepackage[utf8]{inputenc}
\usepackage[T1]{fontenc}

% Silbentrennung
\usepackage[ngerman]{babel}

% erweiterte Matheumgebungen
\usepackage{amsmath}

% zusätzliche mathematische Schriftarten
\usepackage{amsfonts}

% verschiedene mathematische Symbole
\usepackage{amssymb}

% Einheiten setzen z.B. \SI{10}{\kilo\gram\meter\per\second\squared}
% Fehler: \SI{10 +- 0,2e-4}{\metre}
\usepackage{siunitx}
\sisetup{
  output-decimal-marker={,},
  separate-uncertainty
}

% Randbreiten
\usepackage[left=3.5cm,right=3.5cm,top=3cm,bottom=3cm,twoside]{geometry}

% Bilder einfügen
\usepackage{graphicx}

% Verweise innerhalb des Dokuments
\usepackage{hyperref}
\hypersetup{
	colorlinks = true,
	allcolors = {black}
}

% bessere Tabellenlayouts
\usepackage{booktabs}

% Tiefe des Inhaltsverzeichnisses (Level: 1 sections, 2 subsections,
% 3 subsubsections)
\setcounter{tocdepth}{2}

% manuelle Angabe zur Silbentrennung (mehrere Wörter mit Leerzeichen in {} schreiben)
\hyphenation{Rönt-gen-strah-lung}

% DOKUMENTINFORMATIONEN
\title{P428 \\ Röntgenstrahlung und Materialanalyse}

\author{Christopher Deutsch\footnote{christopher.deutsch@uni-bonn.de} \and Christian Bespin\footnote{christian.bespin@uni-bonn.de}}

\date{\today}

\begin{document}

\maketitle

% DURCHFÜHRUNGSDATUM UND ASSISTENT
\begin{center}
\begin{tabular}{l r}
Durchführung: & 3./4. November 2014 \\
Gruppe: & $\alpha$ 2 \\
Assistent: & Peter Klassen
\end{tabular}
\end{center}

% ZUSAMMENFASSUNG
\begin{abstract}
\noindent
% Text
\end{abstract}

% INHALTSVERZEICHNIS
\tableofcontents
% Neue Seite nach TOC
\newpage

% INHALT VERSUCHSPROTOKOLL

\section{Grundlagen}
\subsection{Röntgenstrahlung}
Als Röntgenstrahlung wird der Teil des elektromagnetischen Spektrums mit Photon-Energien zwischen \SI{100}{\electronvolt} und \SI{100}{\kilo\electronvolt} bezeichnet, wobei diese Grenzen nicht scharf sind und deshalb oft hinsichtlich der Strahlungsquelle eine Einordnung vollzogen wird.

\subsubsection{Erzeugung}
Es gibt zwei typische Erzeugungsmethoden von Röntgenstrahlung, welche auf dem Beschuss eines \emph{Targets} mit einem hochenergetischen Elektronenstrahl bestehen.
Die beiden Strahlungsarten, welche in der Praxis oft zusammen auftreten sind:
\begin{itemize}
  \item \textbf{Bremsstrahlung:} Die bei der Streuung der geladenen Elektronen an den Kernen des Targets vollzogene Geschwindigkeitsänderung des Elektrons, führt zur Emission eines Photons.
  Das so entstandene Spektrum ist kontinuierlich, wobei eine maximale Photonenenergie gegeben ist, durch die kinetische Energie der einfallenden Elektronen, welche durch deren Beschleunigungsspannung $U$ gegeben ist.
  \begin{align}
    &E_\mathrm{kin. e^-} = h \cdot \nu_\mathrm{max} = \frac{hc}{\lambda_\mathrm{min}} \nonumber\\
    &\Rightarrow \lambda_\mathrm{min} = \frac{h c}{e U} \quad \text{(Duane-Hunt-Gesetz)}
  \end{align}
  (BILD VON SPEKTRALVERTEILUNG und evtl. Kramers-Gesetz)
  
  Eine Quelle für reine Bremsstrahlung sind Synchrotronstrahlungsquellen.
  
  \item \textbf{charakteristische Strahlung:} Die einfallende hochenergetische Elektronenstrahlung ionisiert ein Elektron der inneren Schale eines Targetatoms.
  Dadurch entsteht ein freier Zustand, welcher durch ein Elektron einer weiter äußeren Schale eingenommen werden kann.
  Durch den Übergang wird ein Photon emittiert, dessen Energie gleich der Energiedifferenz der beiden Zustände ist.
  Aufgrund dieser Abhängigkeit von der elektronisches Struktur des Targetatoms, ist das emittierte, diskrete Spektrum charakteristisch für das verwendete (Target)-Material.
  Außerdem weist das emittierte Spektrum die Aufspaltung der Energieniveaus aufgrund von Fein- und Hyperfeinstruktur auf.
  (BILD VON SPEKTRALVERTEILUNG)
\end{itemize}
In der Praxis treten beide Effekte bei der Erzeugung von Röntgenstrahlen in sogennanten Röntgenröhren auf.

Eine \textbf{Röntgenröhre} besteht aus einem evakuierten Glaskolben mit einer Anordnung von geheizter Kathode und Anode aus Targetmaterial.
Zwischen Kathode und Anode wird die Beschleunigungsspannung $U$ angelegt, sodass beim Heizen der Kathode mit der Heizspannung $U_\mathrm{Heiz}$ aufgrund des glühelektrischen Effekts ein Teil der Elektronen die Austrittsarbeit der Kathode überwinden kann und durch die Beschleunigungsspannung $U$ in Richtung der Anode beschleunigt werden.
Nachdem die Beschleunigungsspannung $U$ durchlaufen wurde, treffen die Elektronen mit der Energie $e U$ auf das Anodenmaterial und erzeugen dabei Bremsstrahlung sowie charakteristische Strahlung, welche durch ein für Röntgenstrahlung durchlässiges Fenster im Glaskolben aus der Röhre austreten können.

\subsubsection{Nachweis}
Zum Nachweis von Röntgenstrahlung werden in diesem Versuch Geiger-Müller-Zählrohre, Röntgenenergiedetektor und Röntgenfilm verwendet.
Der Röntgenfilm ist mit einer speziellen Beschichtung, die bei Auftreffen von Röntgenstrahlung Licht aussendet (Schicht besser erläutern), versehen, welches wiederum eine lichtempfindliche Emulsion(?) belichtet und so ein Bild erzeugt.(Wikipedia?)
Die Funktionsweise der beiden anderen Nachweismethoden wird weiter unten erklärt.

Was noch erwähnt werden sollte:
\begin{itemize}
  \item Elektron-Loch Bildung in pn-Grenzschichten
  \item Szintillationszähler
\end{itemize}

\subsection{Bragg-Reflexion}

\begin{itemize}
  \item Röntgenbeugung
  \item Bragg-Bedingung
  \item Glanzwinkel
\end{itemize}
 
\subsection{Röntgenfluoreszenz}
  Als Röntgenfluoreszenz bezeichnet man die Entstehung sekundärer, fluoreszierender Röntgenstrahlung.
  Dies geschieht ähnlich wie die Erzeugung charakteristischer Röntgenstrahlung, nur dass nun bereits erzeugte Röntgenstrahlung anstatt den energiereichen Elektronen zur Bestrahlung eines Elements verwendet werden.
  Diese sekundäre Röntgenstrahlung weist wieder ein charakteristisches Spektrum auf, aus dem man auf das beschossene Element schließen kann.
  Das Verfahren zur Elementbestimmung (oder Anteilen an Elementen in dem Target (?)) nennt sich daher \textbf{Röntgenfluoreszenzanalyse}.
  
\begin{itemize}
  \item Bestimmung der Massenanteile einzelner Komponenten
\end{itemize}
  
\subsection{Laue-Verfahren}

\begin{itemize}
  \item Laue--Bedingung
  \item Millersche Indizes (reziprokes Gitter)
  \item Elementarzelle
  \item Glanzwinkel
  \item Netzebenenabstand
\end{itemize}  

\subsection{Geiger-Müller-Zählrohr}

\begin{itemize}
  \item Aufbau
  \item Funktionsweise
  \item Totzeit
\end{itemize}
  
\subsection{Röntgenenergiedetektor}

\begin{itemize}
  \item Aufbau
  \item Funktionsweise
  \item PIN-Photodiode
  \item Vielkanalanalysator
\end{itemize}



\section{Versuchsdurchführung}

\subsection{Versuchsbeschreibung}

\subsection{Aufbau}

\subsubsection{Hinweis zu einem der Geräte oder whatever}

\section{Messdaten}

\section{Auswertung}

\section{Diskussion}

\section{Zusammenfassung}

% BIBLIOGRAPHIE

% Maximale Anzahl der Einträge in Klammer
% Zitieren mit \cite{lamport94}
\begin{thebibliography}{9}

% Beispiel
\bibitem{lamport94}
  Leslie Lamport,
  \emph{\LaTeX: a document preparation system}.
  Addison Wesley, Massachusetts,
  2nd edition,
  1994.
  
\bibitem{demtroeder}
	Wolfgang Demtröder,
	\emph{Experimentalphysik 3}.
	Springer Verlag,
	3. Auflage
\end{thebibliography}


\newpage

% APPENDIX
\begin{appendix}

\end{appendix}

\end{document}
