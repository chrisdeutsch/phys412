% PAKETE UND DOKUMENTKONFIGURATION
\documentclass[11pt, a4paper]{article}

% Encoding für Umlaute
\usepackage[utf8]{inputenc}
\usepackage[T1]{fontenc}

% Silbentrennung
\usepackage[ngerman]{babel}

% erweiterte Matheumgebungen und Formelnummer mit Sectionnummer
\usepackage{amsmath}
\numberwithin{equation}{section}

% Braket Notation
\usepackage{braket}

% zusätzliche mathematische Schriftarten
\usepackage{amsfonts}

% verschiedene mathematische Symbole
\usepackage{amssymb}

% Einheiten setzen z.B. \SI{10}{\kilo\gram\meter\per\second\squared}
% Fehler: \SI{10 +- 0,2e-4}{\metre}
\usepackage{siunitx}
\sisetup{
  output-decimal-marker={,},
  separate-uncertainty
}

% Randbreiten
\usepackage[left=3.5cm,right=3.5cm,top=3cm,bottom=3cm,twoside]{geometry}

% Bilder einfügen
\usepackage{graphicx}

% Verweise innerhalb des Dokuments
\usepackage{hyperref}
\hypersetup{
	colorlinks = true,
	allcolors = {black}
}

% bessere Tabellenlayouts
\usepackage{booktabs}
\usepackage{multirow}

% Seitenlayout (Kopfzeile)
\usepackage{fancyhdr}

% Float Barriers
\usepackage{placeins}

% Pakete für gedrehte Subfigures
\usepackage{caption}
\usepackage{subcaption}
\usepackage{rotating}

% Caption-Setup
\captionsetup{font={small}}
\renewcommand{\thefigure}{\thesection.\arabic{figure}}
\renewcommand{\thesubfigure}{\alph{subfigure}}
\renewcommand{\thetable}{\thesection.\arabic{table}}
\renewcommand{\thesubtable}{\alph{subtable}}

% Manuelle Silbentrennung
\hyphenation{}

% Tiefe des Inhaltsverzeichnisses (Level: 1 sections, 2 subsections,
% 3 subsubsections)
\setcounter{tocdepth}{3}

% FANCYHDR SETUP
\pagestyle{fancy}
\fancyhead[EL,OR]{\thepage}
\fancyhead[ER]{\leftmark}
\fancyhead[OL]{\rightmark}

\renewcommand{\sectionmark}[1]{
\markboth{\thesection{} #1}{\thesection{} #1}
}
\renewcommand{\subsectionmark}[1]{
\markright{\thesubsection{} #1}
}

% DOKUMENTINFORMATIONEN
\title{P442 \\ Laser}

\author{Christopher Deutsch\footnote{christopher.deutsch@uni-bonn.de} \and Christian Bespin\footnote{christian.bespin@uni-bonn.de}}

\date{\today}

\begin{document}

\begin{titlepage}

\maketitle

% DURCHFÜHRUNGSDATUM UND TUTOR
\begin{center}
\begin{tabular}{l r}
Durchführung: & 15./16. Dezember 2014 \\
Gruppe: & $\alpha$ 2 \\
Tutor: & Tobias Macha
\end{tabular}
\end{center}

% ZUSAMMENFASSUNG
\begin{abstract}
\noindent

\end{abstract}

\end{titlepage}

% INHALTSVERZEICHNIS
\tableofcontents
% Neue Seite nach TOC
\newpage

% INHALT VERSUCHSPROTOKOLL

\section{Einführung}

In diesem Praktikumsversuch wird ein Helium-Neon-Laser (HeNe-Laser) als Experimentierlaser aufgebaut und seine Eigenschaften wie beispielsweise Strahlprofil, Modenabstände und Polarisation für verschiedene Resonatorlängen vermessen.
Der HeNe-Laser wurde erstmals 1960 von Javan, Bennett und Herriott \cite{javan} entwickelt und stellt den ersten Laser dar, der kontinuierliches Laserlicht erzeugte.

\section{Theorie}

\section{Durchführung und Auswertung}
Die ausführliche Durchführung ist der Versuchsanleitung \cite{anleitung} zu entnehmen.
Sollten Abweichungen bei der Durchführung auftreten, so werden diese im jeweiligen Unterkapitel dargestellt.

\subsection{Aufbau des Helium-Neon Experimentierlasers}
Nach der Justierung des Lasers wurde für jede der vier untersuchten Resonatorlängen $L$ die Länge des Resonators mit einem Maßband vermessen, wobei der Fehler der Längenmessung auf $\Delta L = \SI{0.4}{\centi\metre}$ abgeschätzt wird.
Diese Abschätzung begründet sich darin, dass die exakte Lage der Spiegel in der Fassungen nicht bekannt ist.
Anschließend bestimmen wir die mittlere Spannung an der Photodiode $U_\mathrm{PD}$ mithilfe der \textit{Measure}-Funktion des Oszilloskops, wobei deren Fehler anhand der beobachteten Schwankungen auf $\Delta U_\mathrm{PD} = \SI{0.2}{\milli\volt}$ geschätzt wird.
Unter der Annahme, dass die Photodiode im linearen Bereich arbeitet, was aufgrund des \SI{50}{\ohm}-Abschlusswiderstandes am Oszilloskop gewährleistet ist, kann mithilfe des Umrechnungsfaktors $\SI{1}{\milli\watt} / \SI{17.5}{\milli\volt}$ aus der Versuchsbeschreibung \cite{anleitung} eine Umrechnung in eine Leistung erfolgen:
\begin{align}
	P = \frac{\SI{1}{\milli\watt}}{\SI{17.5}{\milli\volt}} \cdot U_\mathrm{PD}
	\label{eq:umrechnung_watt}
\end{align}
Dabei erfolgt die Fehlerfortpflanzung mit demselben Umrechnungsfaktor.
Die gemessenen Größen und die Umrechnung wurde in Tabelle \ref{tab:intensitaeten} durchgeführt.
\begin{table}[h]
	\centering
	\begin{tabular}{SSSSSS}
\toprule
{$L$ / \si{\centi\metre}} & {$\Delta L$ / \si{\centi\metre}} & {$U_\mathrm{PD}$ / \si{\milli\volt}} & {$\Delta U_\mathrm{PD}$ / \si{\milli\volt}} & {$P$ / \si{\milli\watt}} & {$\Delta P$ / \si{\milli\watt}} \\
\midrule
50.0 & 0.4 & 66.1 & 0.2 & 3.78 & 0.02 \\
69.0 & 0.4 & 60.4 & 0.2 & 3.45 & 0.02 \\
79.5 & 0.4 & 58.0 & 0.2 & 3.31 & 0.02 \\
89.2 & 0.4 & 45.0 & 0.2 & 2.57 & 0.02 \\
\bottomrule
\end{tabular}
	\caption{Gemessene Leistungen bei verschiedenen Resonatorlängen. Die Photospannungen wurden um den Untergrund bei blockiertem Laserstrahl korregiert.}
	\label{tab:intensitaeten}
\end{table}
Auffällig ist die Abnahme der Intensität mit der Resonatorlänge, welche dadurch erklärt werden kann, dass das vom Strahl eingenommene Volumen in der Entladungsröhre mit steigender Länge $L$ abnimmt.
Das verringerte Volumen führt dazu, dass die Verstärkung des aktiven Mediums aufgrund von Sättigung abnimmt.
Außerdem längerer Weg durch Luft (Streuung) vermutlich untergeordnete Rolle.

\subsection{Bestimmung der Wellenlänge des Lasers mit einem Reflexionsgitter}
\subsubsection{Durchführung}
\begin{figure}[h]
	\centering
	\includegraphics[width=0.9\textwidth]{./figures/wellenlaenge_lineal.pdf}
	\caption{Wellenlängenbestimmung nach Schawlow \cite{schawlow} und die im Versuch gemessenen Längen $d$}
	\label{fig:schawlow}
\end{figure}
Zur Bestimmung der Wellenlänge des Helium-Neon Lasers nach Schawlow \cite{schawlow} wird ein Stahlmaßstab mit eingravierter Skaleneinteilung als Reflexionsgitter verwendet.
Dazu wird der Laserstrahl in flachem Winkel auf die Millimeter-Skala des Maßstabes geleitet, sodass ein scharfes Interferenzmuster auf dem Whiteboard in einer Entfernung (von der Mitte der beleuchteten Stelle am Lineal) von $x_0 = \SI{252 +- 2}{\centi\metre}$ entsteht.
Anschließend werden die Maxima des Musters markiert, wobei darauf geachtet werden muss, dass bei der nullten Beugungsordnung begonnen wird, um negative Beugungsordnungen zu vermeiden.
Die nullte Beugungsordnung kann daran erkannt werden, dass sie maximale Intensität aufweist.
Nachdem ausreichend viele Ordnungen markiert wurden, wird das Lineal aus dem Strahlengang entfernt und der Auftreffpunkt des Laserstrahls auf dem Whiteboard markiert.
Schließlich wird mit einem Maßband die Position $d$ aller Beugungsmaxima relativ zum Auftreffpunkt des Lasers ohne Lineal gemessen (vergleiche dazu auch Abbildung \ref{fig:schawlow}).
Die Schärfe des Interferenzmusters ließ es dabei zu, die Längen $d$ mit einem abgeschätzten Fehler von $\Delta d = \SI{0.4}{\centi\metre}$ zu bestimmen, wobei die Hauptfehlerquelle in der Bestimmung des Intensitätsmaximums liegt und nicht in der Längenmessung mit dem Maßband.

\subsubsection{Auswertung}
In \cite{schawlow} wird gezeigt, dass für die Gittergleichung in Kleinwinkelnäherung ($\frac{y_n}{x_0} \ll 1$) gegeben ist:
\begin{align}
	n \lambda = \frac{1}{2} g \left( \frac{y_n^2 - y_0^2}{x_0^2} \right) \text{,}
	\label{eq:gittergleichung_schawlow}
\end{align}
wobei $g$ die Gitterkonstante ist und ansonsten die Notation aus Abbildung \ref{fig:schawlow} verwendet wird.
Um aus den gemessenen Werten $d_n$ die Wellenlänge zu bestimmen, muss zunächst die Position $y_0$ des reflektierten Strahls auf dem Schirm berechnet werden, welche gemäß Abbildung \ref{fig:schawlow} gegeben ist durch:
\begin{align}
	y_0 = \frac{d_0}{2} \text{,}
\end{align}
sodass mit dem gemessenen Wert $d_0 = \SI{19.7 +- 0.4}{\centi\metre}$ folgt:
\begin{align}
	y_0 = \SI{9.85 +- 0.20}{\centi\metre} \text{.}
\end{align}
So kann die Position $y_n$ des $n$-ten Interferenzmaximums berechnet werden durch:
\begin{align}
	y_n &= d_n - y_0 \\
	\Delta y_n &= \sqrt{\Delta d_n^2 + \Delta y_0^2}
\end{align}
und anschließend mit der genäherten Gittergleichung \ref{eq:gittergleichung_schawlow} die Wellenlänge des Lasers für die einzelnen Maxima berechnet werden, wobei für die Gitterkonstante $g = \SI{1}{\milli\metre}$ gilt, da die Beugung an der Millimeter-Skala des Maßstabes stattfindet. 
Der Fehler der berechneten Wellenlänge $\Delta \lambda$ ist gegeben durch Gauß'sche Fehlerfortpflanzung unter der Vernachlässigung des Fehlers in der Gitterkonstanten $g$:
\begin{align}
	\Delta \lambda = \frac{g}{2 n} \sqrt{\left( \frac{2 y_n}{x_0^2} \right)^2 \cdot \Delta y_n^2 + \left( \frac{2 y_0}{x_0^2} \right)^2 \cdot \Delta y_0^2 + \left(\frac{2\left(y_n^2 - y_0^2 \right)}{x_0^3}\right)^2 \cdot \Delta x_0^2} \text{.}
\end{align}
Die gemessenen Abstände $d$ sowie die berechneten Werte wurden in Tabelle \ref{tab:wellenlaengen_berechnung} aufgetragen.
\begin{table}[h]
	\centering
	\begin{tabular}{SSSSSSS}
\toprule
{$n$} & {$d_n$ / \si{\centi\metre}} & {$\Delta d_n$ / \si{\centi\metre}} & {$y_n$ / \si{\centi\metre}} & {$\Delta y_n$ / \si{\centi\metre}} & {$\lambda$ / \si{\nano\metre}} & {$\Delta \lambda$ / \si{\nano\metre}} \\
\midrule
1  & 23.2 & 0.4 & 13.35 & 0.45 & 639 & 100 \\
2  & 26.0 & 0.4 & 16.15 & 0.45 & 645 & 60  \\
3  & 28.1 & 0.4 & 18.25 & 0.45 & 619 & 46  \\
4  & 30.2 & 0.4 & 20.35 & 0.45 & 624 & 38  \\
5  & 32.0 & 0.4 & 22.15 & 0.45 & 620 & 34  \\
6  & 34.0 & 0.4 & 24.15 & 0.45 & 638 & 31  \\
7  & 35.4 & 0.4 & 25.55 & 0.45 & 625 & 28  \\
8  & 37.1 & 0.4 & 27.25 & 0.45 & 635 & 27  \\
9  & 38.5 & 0.4 & 28.65 & 0.45 & 633 & 25  \\
10 & 39.7 & 0.4 & 29.85 & 0.45 & 625 & 24  \\
11 & 41.3 & 0.4 & 31.45 & 0.45 & 639 & 23  \\
12 & 42.4 & 0.4 & 32.55 & 0.45 & 632 & 22  \\
13 & 43.5 & 0.4 & 33.65 & 0.45 & 627 & 21  \\
14 & 44.8 & 0.4 & 34.95 & 0.45 & 632 & 21  \\
15 & 46.0 & 0.4 & 36.15 & 0.45 & 635 & 20  \\
16 & 47.1 & 0.4 & 37.25 & 0.45 & 635 & 20  \\
17 & 47.9 & 0.4 & 38.05 & 0.45 & 626 & 19  \\
18 & 49.2 & 0.4 & 39.35 & 0.45 & 635 & 19  \\
19 & 50.4 & 0.4 & 40.55 & 0.45 & 641 & 19  \\
\bottomrule
\end{tabular}
	\caption{Messdaten und Berechnung zur Wellenlängenbestimmung}
	\label{tab:wellenlaengen_berechnung}
\end{table}
Um aus den berechneten Wellenlängen für die einzelnen Maxima eine Wellenlänge zu bestimmen, wird der varianzgewichtete Mittelwert:
\begin{align}
	\bar{\lambda} = \frac{\sum_i \frac{\lambda_i}{\Delta \lambda_i^2}}{\sum_i \frac{1}{\Delta \lambda_i^2}}
\end{align}
verwendet, wobei dessen Fehler gegeben ist durch:
\begin{align}
\Delta \bar{\lambda} = \frac{1}{\sqrt{\sum_i \frac{1}{\Delta \lambda_i^2}}} \text{.}
\end{align}
Mit den Werten aus Tabelle \ref{tab:wellenlaengen_berechnung} folgt für die varianzgewichtete mittlere Wellenlänge:
\begin{align}
	\bar{\lambda} = \SI{632.3 +- 5.6}{\nano\metre} \text{.}
\end{align}
Der Vergleich mit dem Literaturwert \cite{NISTSpectra} für den ($\mathrm{5s} \rightarrow \mathrm{3p}$)-Übergang von Neon:
\begin{align}
	\lambda = \SI{632.81646}{\nano\metre}
\end{align}
liefert eine gute Übereinstimmung innerhalb des statistischen Fehlers.

\subsubsection{Vergleich mit der exakten Rechnung}
Im Folgenden soll der Einfluss der Näherung bei der Berechnung der Wellenlänge untersucht werden.
Dazu betrachtet man die Gittergleichung mit dem Einfallswinkel $\alpha$ und den Ausfallswinkel der $n$-ten Interferenzordnung $\beta_n$, welche jeweils zur Gitterebene gemessen werden (vergleiche Abbildung \ref{fig:schawlow}):
\begin{align}
	n \lambda = g \left( \cos(\alpha) - \cos(\beta_n) \right) \text{.}
	\label{eq:wellenlaenge_exakt}
\end{align}
Wegen $\alpha = \beta_0$ kann der Kosinus ausgedrückt werden als (Satz des Pythagoras):
\begin{align}
	\cos(\alpha) &= \frac{x_0}{\sqrt{x_0^2 + y_0^2}} \\
	\cos(\beta_n) &= \frac{x_0}{\sqrt{x_0^2 + y_n^2}} \text{,}
\end{align}
womit die relative Abweichung der Näherung von der exakten Rechnung aus den Gleichungen \ref{eq:gittergleichung_schawlow} und \ref{eq:wellenlaenge_exakt} folgt:
\begin{align}
	\frac{\Delta \lambda}{\lambda} = \frac{\lambda_\mathrm{KWN} - \lambda}{\lambda} = \frac{1}{2} \frac{\frac{y_n^2 - y_0^2}{x_0^2}}{ \frac{x_0}{\sqrt{x_0^2 + y_0^2}} - \frac{x_0}{\sqrt{x_0^2 + y_n^2}} } - 1
\end{align}
Diese Funktion ist in Abhängigkeit von $y_n$ in Abbildung \ref{fig:abweichung_wellenlaenge} aufgetragen, wobei für $y_0$ und $x_0$ die im Versuch gemessenen Werte verwendet werden.
\begin{figure}[h]
	\centering
	\input{./plots/abweichung_wellenlaenge.tex}
	\caption{Relative Abweichung der Näherungsformel von der exakten Berechnung bei der im Versuch verwendeten Konfiguration: $y_0 = \SI{9.85}{\centi\metre}$, $x_0 = \SI{252}{\centi\metre}$. Der blaue Pfeil markiert die größte vermessene Interferenzordnung $n = 19$.}
	\label{fig:abweichung_wellenlaenge}
\end{figure}
Man sieht, dass die Abweichung für größere Ordnungen $n$ und damit größere Abstände $y_n$ rapide zunimmt, da $\frac{y_n}{x_0} \ll 1$ nicht mehr gegeben ist.
Demnach weist die größte vermessene Ordnung $n = \num{19}$ die größte relative Abweichung auf, welche wie man in Abbildung \ref{fig:abweichung_wellenlaenge} ablesen kann $\Delta \lambda / \lambda \approx \SI{2}{\percent}$ beträgt.
Dies entspricht schon einer Überschätzung der Wellenlänge von mehr als \SI{10}{\nano\metre} und liegt damit schon in der Größenordnung des statistischen Fehlers der Einzelmessungen.
Bemerkenswert ist, dass wir die Tendenz die Wellenlänge zu überschätzen in Messdaten nicht beobachtet werden kann, weshalb davon ausgegangen werden muss, dass ein systematischer Fehler diese kompensiert.
Eine mögliche Ursache kann sein, dass das Lineal einen kleinen Winkel zur Senkrechten auf der Wand aufwies und somit eine tendenzielle Unterschätzung der berechneten Wellenlänge zur Folge hat.

\subsection{Untersuchung der Polarisation des Lasers}
\label{ssec:polarisation}
Zur Untersuchung der Polarisation des hier verwendeten Experimentierlasers wird der Intensitätsverlauf an einer Photodiode in Abhängigkeit des Winkels eines davor stehenden Linearpolarisators betrachtet.
Dieser ist in der Lage, einfallendes Licht in eine vorgegebene Richtung linear zu polarisieren.
Für den Fall, dass das einfallende Licht bereits linear polarisiert ist (wie für den hier verwendeten Laser angenommen werden kann), gilt für die hinter dem Polarisator registrierte Intensität (Gesetz von Malus):
\begin{align}
I=I_0\cdot\cos^2\varphi
\end{align}
Der Winkel $\varphi$ am Polarisator wurde in Schritten von \SI{10}{\degree} gedreht und jeweils die Photospannung $U_\text{P}$ an der Photodiode in \si{\milli\volt} notiert.
In Tabelle \ref{tab:malus} sind die Messwerte bei einer Resonatorlänge von \SI{50}{\centi\metre} festgehalten worden.
Der Fehler $\Delta\varphi$ für den Winkel ergibt sich aus der Tatsache, dass der Polarisator in Schritten von \SI{2}{\degree} eingestellt werden konnte, der Fehler für die Photospannung wurde aufgrund der Schwankungen, die zu beobachten waren (es wurde mit der \emph{Measure}-Funktion des Oszilloskops gemessen), auf \SI{0.3}{\milli\volt} abgeschätzt.
Diese Messwerte werden im Anschluss mit \eqref{eq:umrechnung_watt} in Intensitäten umgerechnet, in ein Diagramm eingetragen und mit \texttt{Gnuplot} eine Kurve der Form
\begin{align}
f(x) = a\cdot\cos^2(x-b) + c
\end{align}
an die Daten angepasst.
Dabei beschreibt $a$ die Intensität $I_0$, der Parameter $b$ berücksichtigt eine eventuelle Abweichung, dass die Polarisatoreinstellung $\varphi=0$ nicht der Polarisationsrichtung des Lasers entspricht und $c$ beschreibt den auftretenden Untergrund der Spannungs- / Intensitätsmessung.
\begin{figure}
	\centering
	\begin{tabular}{SSSS}
	\toprule
	{Drehwinkel $\varphi / \si{\degree}$} & {Fehler $\Delta\varphi / \si{\degree}$} & {Intensität $I / \si{\milli\volt}$} & {Fehler $\Delta I / \si{\milli\volt}$} \\
	\midrule
	0   & 1 & 39.50 & 0.3 \\
	10  & 1 & 40.00 & 0.3 \\
	20  & 1 & 38.00 & 0.3 \\
	30  & 1 & 33.80 & 0.3 \\
	40  & 1 & 27.90 & 0.3 \\
	50  & 1 & 20.20 & 0.3 \\
	60  & 1 & 13.40 & 0.3 \\
	70  & 1 & 6.45  & 0.3 \\
	80  & 1 & 2.62  & 0.3 \\
	90  & 1 & 0.60  & 0.3 \\
	100 & 1 & 1.10  & 0.3 \\
	110 & 1 & 3.50  & 0.3 \\
	120 & 1 & 8.60  & 0.3 \\
	130 & 1 & 14.70 & 0.3 \\
	140 & 1 & 21.20 & 0.3 \\
	150 & 1 & 27.70 & 0.3 \\
	160 & 1 & 33.40 & 0.3 \\
	170 & 1 & 37.00 & 0.3 \\
	180 & 1 & 38.50 & 0.3 \\
	190 & 1 & 37.50 & 0.3 \\
	200 & 1 & 34.70 & 0.3 \\
	210 & 1 & 29.60 & 0.3 \\
	220 & 1 & 22.90 & 0.3 \\
	230 & 1 & 16.30 & 0.3 \\
	240 & 1 & 9.50  & 0.3 \\
	250 & 1 & 4.10  & 0.3 \\
	260 & 1 & 1.30  & 0.3 \\
	270 & 1 & 0.30  & 0.3 \\
	280 & 1 & 2.00  & 0.3 \\
	290 & 1 & 5.60  & 0.3 \\
	300 & 1 & 10.90 & 0.3 \\
	310 & 1 & 17.60 & 0.3 \\
	320 & 1 & 24.40 & 0.3 \\
	330 & 1 & 30.50 & 0.3 \\
	340 & 1 & 35.60 & 0.3 \\
	350 & 1 & 38.70 & 0.3 \\
	360 & 1 & 39.50 & 0.3 \\
	\bottomrule
\end{tabular}
	\caption{Anpassung einer zu erwartenden Kurve an die Messwerte}
	\label{fig:malus}
\end{figure}
Man kann an der Anpassung gut erkennen, dass die Intensität wie erwartet dem Gesetz von Malus folgt.
Dies bestätigt auch die Annahme, dass der Laser durch die an der Entladungsröhre befestigten Brewster-Fenster mit linear polarisiertem Licht betrieben wird.
Die Abweichungen der Anpassung (das reduzierte $\chi^2$ von \num{25.4} bestätigt die schlechte Übereinstimmung) von den Messwerten lassen sich durch eine nicht ideale Justage erklären.
ERKLÄRUNG
Grund für die Annahme ist, dass das Maximum bei \SI{180}{\degree} nicht den zu erwartenden gleichen Wert annimmt, wie die Maxima bei \SI{0}{\degree} bzw. \SI{360}{\degree}.\\
\\
Nachfolgend soll der Polarisationsgrad des Lasers bestimmt werden.
Dieser ist gegeben durch:
\begin{align}
P=\frac{|I_{\parallel}-I_{\perp}|}{I_{\parallel}+I_{\perp}}
\end{align}
Der Fehler ergibt sich mit Gauß'scher Fehlerfortpflanzung zu:
\begin{align}
\Delta P=\sqrt{\left(1+\frac{I_{\perp}}{(I_{\parallel}+I_{\perp})^2}\right)^2\left(\Delta I_{\parallel}\right)^2 + \left(\frac{I_{\perp}}{(I_{\parallel}+I_{\perp})^2}-\frac{1}{I_{\parallel}+I_{\perp}}\right)^2\left(\Delta I_{\perp}\right)^2}
\end{align}
Hierbei wird angenommen, dass $I_\parallel$ die Intensität ist, bei der der Linearpolarisaor parallel zur Polarisationsrichtung des Lichts steht und so ein Transmissionsmaximum erreicht wird.
Für $I_\perp$ ist die Polarisationsrichtung des Lichts senkrecht zur Richtung des Polarisators, es wird also ein Transmissionsminimum erreicht.
Dazu wählen wir aus Tabelle \ref{tab:malus} die Werte für $\varphi=\SI{90}{\degree}$ als Transmissionsminimum und ausgehend davon liegt das Transmissionsmaximum bei $\varphi=\SI{0}{\degree}$.
Der in der Anpassung (s.o.) bestimmte Parameter $b=\SI{1.7+-0.6}{\degree}$, der die Lage der Extrema genau festlegt, wird dabei vernachlässigt, da er recht klein ist und hierfür keine Messwerte vorliegen. 
Dann ergibt sich $P$ mit $I_\parallel=\SI{2.26+-0.02}{\milli\watt}$ und $I_\perp=\SI{0.03+-0.02}{\milli\watt}$ (die Photospannungen wurden mit \eqref{eq:umrechnung_watt} abermals in Intensitäten umgerechnet) zu:
\begin{align}
P=\num{0.97+-0.03}
\end{align}

\subsection{Messung des Strahlprofils und des Stabilitätsgebiets des Lasers}

\subsection{Aufbau der optischen Diode}

Die optische Diode wird in diesem Versuch durch eine Kombination von Linearpolarisator und $\lambda/4$-Platte erreicht.
Sie wird gemäß der Praktikumsanleitung aufgebaut und justiert.
Zur korrekten Einstellung der Diode wird der Polarisator nun so gedreht, dass das transmittierte Licht ein Maximum erreicht.
Hierfür findet man bei der Einstellung von \SI{254}{\degree} das Transmissionsminimum und verdreht den Polarisator um weitere \SI{90}{\degree} (dies folgt aus dem Gesetz von Malus, siehe Abschnitt \ref{ssec:polarisation}), so dass das Transmissionsmaximum bei einer Einstellung von \SI{344}{\degree} erreicht wird.\\
\\
Vor der optischen Diode kann bei der Resonatorlänge \SI{50}{\centi\metre} eine Intensität von \SI{66.2+-0.1}{\milli\volt} gemessen werden, hinter der Diode beträgt die Intensität \SI{47.2+-0.1}{\milli\volt}.
Die Leistungsverluste an der Diode (z.B. Reflexionen an der Vorderseite der $\lambda/4$-Platte) belaufen sich somit auf ca. \SI{1}{\milli\watt}.
Dabei wurde wie in \eqref{eq:umrechnung_watt} der Umrechnungsfaktor \num{17.5} zwischen der gemessenen Intensität in \si{\milli\volt} und \si{\milli\watt} verwendet.

\subsection{Optischer Spektrumanalysator}

\subsection{Präzise Messung des Modenabstandes mittels einer optischen Schwebung}
Die Überlagerung des elektrischen Feldes zweier Axialmoden mit den Kreisfrequenzen $\omega_1$ und $\omega_2$ auf einer Photodiode kann geschrieben werden als:
\begin{align}
	E(t) = E_1 \sin(\omega_1 t) + E_2 \sin(\omega_2 t)
	\label{eq:superposition_efeld}
\end{align}
Da die Photodiode die Intensität $I$ des auftreffenden Lichts misst und diese gegeben ist durch $I \propto E^2$ erhält man durch quadrieren von Gleichung \ref{eq:superposition_efeld} und Anwendung der trigonometrischen Additionstheoreme das resultierende Frequenzspektrum der Intensität, welche aus den Frequenzen $\omega_1 + \omega_2, |\omega_1 - \omega_2|$ sowie den zweiten Harmonischen $2\omega_1, 2\omega_2$ besteht.
Mit Ausnahme von dem Heterodynsignal mit der Frequenz $|\omega_1 - \omega_2|$, welche in der Größenordnung von einigen $\SI{100}{\mega\hertz}$ liegt, sind die restlichen auftretenden Frequenzen so groß ($\sim \si{\peta\hertz}$), dass diese aufgrund der endlichen Bandbreite von Koaxialkabel und Photodiode ($\sim \si{\giga\hertz}$) nicht gemessen werden können.

Auch die Frequenz der optischen Schwebung ist so hoch, dass diese nicht direkt gemessen werden kann.
Daher wird in einer zweiten Stufe das Signal der Schwebung mit dem eines Hochfrequenzgenerators der Frequenz $\nu_\mathrm{HF}$ gemischt.
Die Wirkung eines solchen \textbf{Mischers} auf zwei Eingangssignale $x(t) = \hat{x} \sin(\omega_1 t)$ und $y(t) = \hat{y} \sin(\omega_2 t)$ lässt sich auf zwei verschiedenen Weisen realisieren \cite{horowitz_hill}:
\begin{itemize}
	\item \textbf{multiplikative Mischung:} Durch Multiplikation der beiden Signale $x(t)$ und $y(t)$ führt die Anwendung der trigonometrischen Identität:
	\begin{align}
		\sin(\omega_1 t) \sin(\omega_2 t) = \frac{1}{2} \cos\left[ (\omega_1 - \omega_2) t \right] - \frac{1}{2} \cos\left[ (\omega_1 + \omega_2) t\right]
	\end{align}
	auf das resultierende Signal:
	\begin{align}
		(x\cdot y)(t) = \frac{\hat{x} \, \hat{y}}{2} \left\{ \cos\left[ (\omega_1 - \omega_2) t \right] - \cos\left[ (\omega_1 + \omega_2) t\right]\right\} \text{.}
	\end{align}
	Das resultierende Spektrum enthält sowohl die Summe als auch die Differenz der beiden Eingangsfrequenzen.
	\item \textbf{Mischung durch Anwendung einer nichtlinearen Operation auf die Summe der Signale:} Um die Differenzfrequenz $|\omega_1 - \omega_2|$ zu bestimmen, ist es nicht nötig das Produkt der beiden Eingänge zu bilden.
	Alternativ reicht es aus die Summe beider Signale zu bilden und eine nichtlineare Operation auf diese durchzuführen.
	
\end{itemize}


\begin{align}
	\Delta \nu &= \frac{c}{2 L} \\
	c &= 2 L \nu_\mathrm{RF}\\
	\Delta c &= 2 \sqrt{ L^2 \cdot \Delta \nu_\mathrm{RF}^2 + \nu_\mathrm{RF}^2 \cdot \Delta L^2}
\end{align}

\section{Fazit}


% BIBLIOGRAPHIE
\vspace{\fill}
% Maximale Anzahl der Einträge in Klammer
% Zitieren mit \cite{lamport94}
\begin{thebibliography}{9}
\bibitem{javan}
A. Javan, W. R. Bennett, Jr., and D. R. Herriott,
\emph{Population Inversion and Continuous Optical Maser Oscillation in a Gas Discharge Containing a He-Ne Mixture},
Phys. Rev. Lett. 6, 106 – Published 1 February 1961

\bibitem{anleitung}
	Physikalisches Praktikum IV: Atome, Moleküle, Festkörper,
	Versuchsbeschreibung \emph{P442: Laser} (Stand: 12. September 2014),
	Universität Bonn
\bibitem{schawlow}
	A. L. Schawlow,
	\emph{Measuring the Wavelength of Light with a Ruler},
	Am. J. Phys., Volume 33, Issue 11 (1965)

\bibitem{NISTSpectra}
	Kramida, A., Ralchenko, Yu., Reader, J., and NIST ASD Team (2014).
	\emph{NIST Atomic Spectra Database} (ver. 5.2).
	\url{http://physics.nist.gov/asd} (Letzter Abruf: 18. Dezember 2014).
	National Institute of Standards and Technology, Gaithersburg, MD.
	
\bibitem{horowitz_hill}
	Paul Horowitz, Winfred Hill,
	\emph{The Art of Electronics Second Edition},
	Cambridge University Press 1989,
	Chapter 13.12: High frequency and high-speed techniques -- Radiofrequency circuit elements
 
\end{thebibliography}

\clearpage

% APPENDIX
\begin{appendix}
\section{Anhang}
\subsection{Messwerte der Photospannung hinter dem Polarisator}
\begin{table}[h]
	\centering
	\begin{tabular}{SSSS}
	\toprule
	{Drehwinkel $\varphi / \si{\degree}$} & {Fehler $\Delta\varphi / \si{\degree}$} & {Intensität $I / \si{\milli\volt}$} & {Fehler $\Delta I / \si{\milli\volt}$} \\
	\midrule
	0   & 1 & 39.50 & 0.3 \\
	10  & 1 & 40.00 & 0.3 \\
	20  & 1 & 38.00 & 0.3 \\
	30  & 1 & 33.80 & 0.3 \\
	40  & 1 & 27.90 & 0.3 \\
	50  & 1 & 20.20 & 0.3 \\
	60  & 1 & 13.40 & 0.3 \\
	70  & 1 & 6.45  & 0.3 \\
	80  & 1 & 2.62  & 0.3 \\
	90  & 1 & 0.60  & 0.3 \\
	100 & 1 & 1.10  & 0.3 \\
	110 & 1 & 3.50  & 0.3 \\
	120 & 1 & 8.60  & 0.3 \\
	130 & 1 & 14.70 & 0.3 \\
	140 & 1 & 21.20 & 0.3 \\
	150 & 1 & 27.70 & 0.3 \\
	160 & 1 & 33.40 & 0.3 \\
	170 & 1 & 37.00 & 0.3 \\
	180 & 1 & 38.50 & 0.3 \\
	190 & 1 & 37.50 & 0.3 \\
	200 & 1 & 34.70 & 0.3 \\
	210 & 1 & 29.60 & 0.3 \\
	220 & 1 & 22.90 & 0.3 \\
	230 & 1 & 16.30 & 0.3 \\
	240 & 1 & 9.50  & 0.3 \\
	250 & 1 & 4.10  & 0.3 \\
	260 & 1 & 1.30  & 0.3 \\
	270 & 1 & 0.30  & 0.3 \\
	280 & 1 & 2.00  & 0.3 \\
	290 & 1 & 5.60  & 0.3 \\
	300 & 1 & 10.90 & 0.3 \\
	310 & 1 & 17.60 & 0.3 \\
	320 & 1 & 24.40 & 0.3 \\
	330 & 1 & 30.50 & 0.3 \\
	340 & 1 & 35.60 & 0.3 \\
	350 & 1 & 38.70 & 0.3 \\
	360 & 1 & 39.50 & 0.3 \\
	\bottomrule
\end{tabular}
	\caption{Messwerte der Spannung an der Photodiode in Abhängigkeit des Winkels am Linearpolarisator}
	\label{tab:malus}
\end{table}

\end{appendix}

\end{document}
