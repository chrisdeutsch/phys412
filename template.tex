% PAKETE UND DOKUMENTKONFIGURATION
\documentclass[10pt, a4paper]{article}

% Encoding für Umlaute
\usepackage[utf8]{inputenc}

% Silbentrennung
\usepackage[ngerman]{babel}

% erweiterte Matheumgebungen
\usepackage{amsmath}

%
\usepackage{amsfonts}

%
\usepackage{amssymb}

% Einheiten setzen z.B. \SI{10}{\kilo\gram\meter\per\second\squared}
% Fehler: \SI{10 +- 0,2e-4}{\metre}
\usepackage{siunitx}
\sisetup{
  output-decimal-marker={,},
  separate-uncertainty
}

% Randbreiten
\usepackage[left=2cm,right=2cm,top=2cm,bottom=2cm]{geometry}

% Bilder einfügen
\usepackage{graphicx}


% bessere Tabellenlayouts
\usepackage{booktabs}

%Textumflossene Grafiken, z.B. \begin{wrapfigure}[14]{R}[1pt]{0.5\textwidth}
\usepackage{wrapfig}

% Tiefe des Inhaltsverzeichnisses (Level: 1 sections, 2 subsections,
% 3 subsubsections)
\setcounter{tocdepth}{2}


% DOKUMENTINFORMATIONEN
\title{P401 \\ Elektronische Übergänge in Atomen}

\author{Christopher Deutsch \and Christian Bespin}

\date{\today}

\begin{document}

\maketitle

% DURCHFÜHRUNGSDATUM UND ASSISTENT
\begin{center}
\begin{tabular}{l r}
Durchführung: & 20./21. Oktober 2014 \\
Gruppe: & 2 $\alpha$ \\
Assistent: & Hans Wurst
\end{tabular}
\end{center}

% ZUSAMMENFASSUNG
\begin{abstract}
% Text
\end{abstract}

% INHALTSVERZEICHNIS
\tableofcontents
% Neue Seite nach TOC
\newpage

% INHALT VERSUCHSPROTOKOLL

\section{Grundlagen / Theorie}

\section{Versuchsdurchführung}

\subsection{Versuchsbeschreibung}

\subsection{Aufbau}

\subsubsection{Hinweis zu einem der Geräte oder whatever}

\section{Messdaten}

\section{Auswertung}

\section{Diskussion}

\section{Zusammenfassung}

% BIBLIOGRAPHIE

% Maximale Anzahl der Einträge in Klammer
% Zitieren mit \cite{lamport94}
\begin{thebibliography}{9}

% Beispiel
\bibitem{lamport94}
  Leslie Lamport,
  \emph{\LaTeX: a document preparation system}.
  Addison Wesley, Massachusetts,
  2nd edition,
  1994.
\end{thebibliography}

\end{document}
