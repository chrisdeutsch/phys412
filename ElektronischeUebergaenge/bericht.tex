% PAKETE UND DOKUMENTKONFIGURATION
\documentclass[10pt, a4paper]{article}

% Encoding für Umlaute
\usepackage[utf8]{inputenc}

% Silbentrennung
\usepackage[ngerman]{babel}

% erweiterte Matheumgebungen
\usepackage{amsmath}

% zusätzliche mathematische Schriftarten
\usepackage{amsfonts}

% verschiedene mathematische Symbole
\usepackage{amssymb}

% Einheiten setzen z.B. \SI{10}{\kilo\gram\meter\per\second\squared}
% Fehler: \SI{10 +- 0,2e-4}{\metre}
\usepackage{siunitx}
\sisetup{
  output-decimal-marker={,},
  separate-uncertainty
}

% Randbreiten
\usepackage[left=3.5cm,right=3.5cm,top=3cm,bottom=3cm,twoside]{geometry}

% Bilder einfügen
\usepackage{graphicx}

% Verweise innerhalb des Dokuments
\usepackage{hyperref}
\hypersetup{
	colorlinks = true,
	allcolors = {black}
}

% bessere Tabellenlayouts
\usepackage{booktabs}

% Tiefe des Inhaltsverzeichnisses (Level: 1 sections, 2 subsections,
% 3 subsubsections)
\setcounter{tocdepth}{2}


% DOKUMENTINFORMATIONEN
\title{P401 \\ Elektronische Übergänge in Atomen}

\author{Christopher Deutsch\footnote{christopher.deutsch@uni-bonn.de} \and Christian Bespin\footnote{christian.bespin@uni-bonn.de}}

\date{\today}

\begin{document}

\maketitle

% DURCHFÜHRUNGSDATUM UND ASSISTENT
\begin{center}
\begin{tabular}{l r}
Durchführung: & 17./18. November 2014 \\
Gruppe: & $\alpha$ 2 \\
Assistent: & ASSISTENT
\end{tabular}
\end{center}

% ZUSAMMENFASSUNG
\begin{abstract}
\noindent
% Text
\end{abstract}

% INHALTSVERZEICHNIS
\tableofcontents
% Neue Seite nach TOC
\newpage

% INHALT VERSUCHSPROTOKOLL

\section{Grundlagen / Theorie}

\subsection{Zeeman-Effekt}

\subsubsection{Verhalten von Atomen in äußeren Feldern}

\begin{itemize}
\item Zeeman Effekt
\item Niveauaufspaltung, Auswahlregeln und Übergänge von Cadmium
\end{itemize}

\subsubsection{Fabry-Pérot-Etalon}

\begin{itemize}
\item Funktionsprinzip
\item Auflösungsvermögen
\item Finesse
\end{itemize}

\subsubsection{Natürliche Linienbreite und Linienverbreiterung}

\subsubsection{$\frac{\lambda}{4}$-Platte}

\subsubsection{Hall-Sonde}

\subsection{Franck-Hertz-Versuch}

\subsubsection{Röhrenelektronik}

\begin{itemize}
\item Glühemission
\item Raumladung
\item Elektronenbewegung in Triodensystem
\end{itemize}

\subsubsection{Thermodynamik und Stoßprozesse}

\begin{itemize}
\item Gaskinetik
\item Zwei-Phasen-System
\item (in-)elastische Stöße
\item mittlere freie Weglänge
\item Stoßquerschnitt
\item Gegenfeldmethode
\end{itemize}

\subsubsection{Quecksilberatom}

\begin{itemize}
\item LS- / jj-Kopplung
\item Termschema
\item Auswahlregeln
\item Resonanzabsorption
\end{itemize}

\section{Versuchsdurchführung}

\subsection{Versuchsbeschreibung}

\subsection{Aufbau}

\subsubsection{Hinweis zu einem der Geräte oder whatever}

\section{Messdaten}

\section{Auswertung}

\section{Diskussion}

\section{Zusammenfassung}

% BIBLIOGRAPHIE

% Maximale Anzahl der Einträge in Klammer
% Zitieren mit \cite{lamport94}
\begin{thebibliography}{9}

% Beispiel
\bibitem{lamport94}
  Leslie Lamport,
  \emph{\LaTeX: a document preparation system}.
  Addison Wesley, Massachusetts,
  2nd edition,
  1994.
\end{thebibliography}

\newpage

% APPENDIX
\begin{appendix}

\end{appendix}

\end{document}
