% PAKETE UND DOKUMENTKONFIGURATION
\documentclass[10pt, a4paper]{article}

% Encoding für Umlaute
\usepackage[utf8]{inputenc}

% Silbentrennung
\usepackage[ngerman]{babel}

% erweiterte Matheumgebungen
\usepackage{amsmath}

%
\usepackage{amsfonts}

%
\usepackage{amssymb}

% Einheiten setzen z.B. \SI{10}{\kilo\gram\meter\per\second\squared}
% Fehler: \SI{10 +- 0,2e-4}{\metre}
\usepackage{siunitx}
\sisetup{
  output-decimal-marker={,},
  separate-uncertainty
}

% Randbreiten
\usepackage[left=3cm,right=4cm,top=3cm,bottom=3cm,twoside]{geometry}

% Bilder einfügen
\usepackage{graphicx}

% Tiefe des Inhaltsverzeichnisses (Level: 1 sections, 2 subsections,
% 3 subsubsections)
\setcounter{tocdepth}{2}

% DOKUMENTINFORMATIONEN
\title{P422 \\ Rastertunnelmikroskopie}

\author{Christopher Deutsch \and Christian Bespin}

\date{\today}

\begin{document}
  
\maketitle

% DURCHFÜHRUNGSDATUM UND ASSISTENT
\begin{center}
\begin{tabular}{l r}
Durchführung: & 20./21. Oktober 2014 \\
Gruppe: & 2 $\alpha$ \\
Assistent: & Peter Klassen
\end{tabular}
\end{center}

% ZUSAMMENFASSUNG
\begin{abstract}
% Text
\end{abstract}

% INHALTSVERZEICHNIS
\tableofcontents
% Neue Seite nach TOC
\newpage

% INHALT VERSUCHSPROTOKOLL
\section{Grundlagen}

\subsection{Tunneleffekt und Tunnelstrom}
Der Tunneleffekt ist ein quantenmechanisches Phänomen, welches einem einlaufenden Teilchen der Energie $E$ erlaubt, eine klassisch unüberwindbare Potentialschwelle der Höhe $V_0 > E$ mit endlicher Wahrscheinlichkeit zu durchqueren.
Im Gegensatz zum klassischen Fall, bei dem das Teilchen am Potentialwall reflektiert wird, nimmt bei der quantenmechanischen Beschreibung das Betragsquadrat der Wellenfunktion $|\Psi|^2$ und damit die Aufenthaltswahrscheinlichkeitdichte des Teilchens, exponentiell mit der Eindringtiefe $d$ ab.

Dieser Effekt wird beim Rastertunnelmikroskop (RTM) ausgenutzt, indem eine Spannung zwischen dem zu untersuchenden (leitenden) Objekt und der Spitze des RTM angelegt wird. Dadurch können Elektronen aus der Probe in die Spitze tunneln (oder umgekehrt), was zu einem Stromfluss $I_T$ führt.
Dieser Tunnelstrom ist gegeben durch \cite{binning}:
\begin{equation}
  I_T \propto \exp(-\alpha d)
\end{equation}
also abhängig vom Abstand Spitze-Probe $d$, sowie von deren elektronischen Eigenschaften.

\subsection{Funktionsweise des Rastertunnelmikroskops}
\subsubsection{Aufbau}
\subsubsection{Piezoeffekt}
In Elementarzellen mit nicht-inversionssymmetrischer Ladungsverteilung kann, beim Anlegen eines externen elektrischen Feldes, ein elektrisches Dipolmoment induziert werden.
Diese Polarisation führt zu einer Längenkontraktion/-ausdehnung der Elementarzelle aufgrund der Asymmetrie der Ladungsverteilung.
Dieser Effekt wird makroskopisch als Längenänderung eines piezoelektrischen Zylinders bei angelegter Spannung sichtbar.

\subsubsection{Betriebsmodi}
\subsubsection{Regelkreis}
\subsubsection{Auflösungsvermögen}

\subsection{Spitzenherstellung}
\subsubsection{Reißen von Platin-Iridium Spitzen}
\subsubsection{Ätzen von Wolfram Spitzen}

\subsection{Kristallstruktur von Graphit}

\subsection{Verwandte Rastermethoden}

\section{Versuchsdurchführung}

\section{Messdaten}

\section{Auswertung}

\section{Diskussion}

\section{Zusammenfassung}

% BIBLIOGRAPHIE

% Maximale Anzahl der Einträge in Klammer
% Zitieren mit \cite{lamport94}
\begin{thebibliography}{9}

% Beispiel
\bibitem{binning}
  G. Binning et al., Phys. Rev. Lett. 49, 57 (1982),
  \emph{Surface Studies by Scanning Tunneling Microscopy}.
\end{thebibliography}

\end{document}
