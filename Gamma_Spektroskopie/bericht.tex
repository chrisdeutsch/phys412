% PAKETE UND DOKUMENTKONFIGURATION
\documentclass[11pt, a4paper]{article}

% Encoding für Umlaute
\usepackage[utf8]{inputenc}
\usepackage[T1]{fontenc}

% Silbentrennung
\usepackage[ngerman]{babel}

% erweiterte Matheumgebungen und Formelnummer mit Sectionnummer
\usepackage{amsmath}
\numberwithin{equation}{section}

% Braket Notation
\usepackage{braket}
\usepackage{isotope}
\usepackage{mhchem}

% zusätzliche mathematische Schriftarten
\usepackage{amsfonts}

% verschiedene mathematische Symbole
\usepackage{amssymb}

% Einheiten setzen z.B. \SI{10}{\kilo\gram\meter\per\second\squared}
% Fehler: \SI{10 +- 0,2e-4}{\metre}
\usepackage{siunitx}
\sisetup{
  output-decimal-marker={,},
  separate-uncertainty
}

% Einheitendefinitionen
\DeclareSIUnit{\skt}{Skt.}
\DeclareSIUnit{\gauss}{G}

% Operatordefinitionen
\DeclareMathOperator{\erf}{erf}

% Randbreiten
\usepackage[left=3.5cm,right=3.5cm,top=3cm,bottom=3cm,twoside]{geometry}

% Bilder einfügen
\usepackage{graphicx}

% Verweise innerhalb des Dokuments
\usepackage{hyperref}
\hypersetup{
	colorlinks = true,
	allcolors = {black}
}

% bessere Tabellenlayouts
\usepackage{booktabs}
\usepackage{multirow}
\usepackage{multicol}

% Seitenlayout (Kopfzeile)
\usepackage{fancyhdr}

% Float Barriers
\usepackage{placeins}

% Pakete für gedrehte Subfigures
\usepackage{caption}
\usepackage{subcaption}
\usepackage{rotating}

% Paket für textumflossene Abbildungen und Tabellen
\usepackage{wrapfig}

\usepackage{float}

% Caption-Setup
\captionsetup{font={small}}
\renewcommand{\thefigure}{\thesection.\arabic{figure}}
\renewcommand{\thesubfigure}{\alph{subfigure}}
\renewcommand{\thetable}{\thesection.\arabic{table}}
\renewcommand{\thesubtable}{\alph{subtable}}

% Manuelle Silbentrennung
\hyphenation{Re-so-na-tor Mo-den-ab-stand Re-so-na-tor-län-ge Spek-t-ro-me-ter}

% Tiefe des Inhaltsverzeichnisses (Level: 1 sections, 2 subsections,
% 3 subsubsections)
\setcounter{tocdepth}{3}

% FANCYHDR SETUP
\pagestyle{fancy}
\fancyhead[EL,OR]{\thepage}
\fancyhead[ER]{\leftmark}
\fancyhead[OL]{\rightmark}

\renewcommand{\sectionmark}[1]{
\markboth{\thesection{} #1}{\thesection{} #1}
}
\renewcommand{\subsectionmark}[1]{
\markright{\thesubsection{} #1}
}

% DOKUMENTINFORMATIONEN
\title{P521 \\ Gamma-Spektroskopie mit Szintillations- und Halbleiterdetektoren}

\author{Christopher Deutsch\footnote{christopher.deutsch@uni-bonn.de} \and Christian Bespin\footnote{christian.bespin@uni-bonn.de}}

\date{\today}

\begin{document}

\begin{titlepage}

\maketitle

% DURCHFÜHRUNGSDATUM UND TUTOR
\begin{center}
\begin{tabular}{l r}
Durchführung: & 07./08. April 2015 \\
Gruppe: & $\alpha$ 6 \\
Tutor: & Yannick Wunderlich
\end{tabular}
\end{center}

% ZUSAMMENFASSUNG
\begin{abstract}
\noindent

\end{abstract}

\end{titlepage}

% INHALTSVERZEICHNIS
\tableofcontents
% Neue Seite nach TOC
\newpage

% INHALT VERSUCHSPROTOKOLL

\section{Einführung}

\section{Theorie}

\subsection{Radioaktiver Zerfall}

\subsection{Wechselwirkung von $\gamma$-Quanten mit Materie}

\subsection{Szintillatoren, Halbleiterdetektor, Photomulti etc.}

\subsection{Spektrum, Vielkanalanalysator}

\subsection{Auflösungsvermögen, Breite, Unterschiede der Detektoren}

\subsection{Nachweiswahrscheinlichkeit}

\subsection{Termschemata}


\section{Versuchsaufbau}

\section{Durchführung und Auswertung}
Die ausführliche Durchführung ist der Versuchsanleitung \cite{anleitung} zu entnehmen.
Sollten Abweichungen bei der Durchführung auftreten, so werden diese im jeweiligen Unterkapitel dargestellt.

\section{Fazit}

\FloatBarrier
% BIBLIOGRAPHIE
\vspace{\fill}
% Maximale Anzahl der Einträge in Klammer
% Zitieren mit \cite{lamport94}
\begin{thebibliography}{19}

\bibitem{krane}
	Kenneth S. Krane,
	\emph{Introductory Nuclear Physics},
	John Wiley \& Sons 1988

\bibitem{mayer-kuckuk}
	Theo Mayer-Kuckuk,
	\emph{Kernphysik - Eine Einführung} (7. Auflage),
	Teubner 2002

\bibitem{siegbahn}
	K. Siegbahn,
	\emph{Alpha-, Beta- and Gamma-Ray Spectroscopy},
	Elsevier Science Ltd. 1965

\bibitem{hillert}
	W. Hillert,
	\emph{physics612: Accelerator Physics},
	Universität Bonn 2014

\bibitem{anleitung}
	Physikalisches Praktikum V: Kern- und Teilchenphysik,
	Versuchsbeschreibung \emph{P523: $\beta$-Spektrometer} (Stand: Januar 2015),
	Universität Bonn	

\bibitem{fermi_function}
	Venkataramaiah, P.; Gopala, K.; Basavaraju, A.; Suryanarayana, S.S.; Sanjeevia, H.
	\emph{A simple relation for the Fermi function},
	Journal of Physics G 11 (3): 359-364

\bibitem{riezler}
	Riezler, W.; Kopitzki, K.
	\emph{Kernphysikalisches Praktikum},
	Teubner 1963

\bibitem{tl_literatur}
  C.J. Chiara, F.G. Kondev,
  Nuclear Data Sheets 111,141 (2010),
  \url{http://www.nndc.bnl.gov/nudat2/decaysearchdirect.jsp?nuc=204TL&unc=nds}
  (Letzter Aufruf: 16. April 2015)

\bibitem{na_literatur}
  R.B. Firestone,
  Nuclear Data Sheets 106, 1 (2005)
  \url{http://www.nndc.bnl.gov/nudat2/decaysearchdirect.jsp?nuc=22NA&unc=nds}
  (Letzter Aufruf: 16. April 2015) 

\end{thebibliography}

% APPENDIX
\begin{appendix}
\section{Anhang}
Auf den folgenden Seiten sind der Vollständigkeit halber alle gemessenen Werte sowie die jeweils zur Auswertung berechneten Werte zusammengetragen.
\clearpage

\end{appendix}

\end{document}
