% PAKETE UND DOKUMENTKONFIGURATION
\documentclass[11pt, a4paper]{article}

% Encoding für Umlaute
\usepackage[utf8]{inputenc}
\usepackage[T1]{fontenc}

% Silbentrennung
\usepackage[ngerman]{babel}

% erweiterte Matheumgebungen
\usepackage{amsmath}

% Braket Notation
\usepackage{braket}

% zusätzliche mathematische Schriftarten
\usepackage{amsfonts}

% verschiedene mathematische Symbole
\usepackage{amssymb}

% Einheiten setzen z.B. \SI{10}{\kilo\gram\meter\per\second\squared}
% Fehler: \SI{10 +- 0,2e-4}{\metre}
\usepackage{siunitx}
\sisetup{
  output-decimal-marker={,},
  separate-uncertainty
}

% Randbreiten
\usepackage[left=3.5cm,right=3.5cm,top=3cm,bottom=3cm,twoside]{geometry}

% Bilder einfügen
\usepackage{graphicx}

% Verweise innerhalb des Dokuments
\usepackage{hyperref}
\hypersetup{
	colorlinks = true,
	allcolors = {black}
}

% bessere Tabellenlayouts
\usepackage{booktabs}

% Seitenlayout (Kopfzeile)
\usepackage{fancyhdr}

% Float Barriers
\usepackage{placeins}

% Pakete für gedrehte Subfigures
\usepackage{caption}
\usepackage{subcaption}
\usepackage{rotating}

% Manuelle Silbentrennung
\hyphenation{Halb-werts-brei-te Fa-bry-Pé-rot-E-ta-lon Zee-man-E-ffekt Cad-mi-um-Lam-pe Kon-den-sor-lin-se}

% Tiefe des Inhaltsverzeichnisses (Level: 1 sections, 2 subsections,
% 3 subsubsections)
\setcounter{tocdepth}{3}

% FANCYHDR SETUP
\pagestyle{fancy}
\fancyhead[EL,OR]{\thepage}
\fancyhead[ER]{\leftmark}
\fancyhead[OL]{\rightmark}

\renewcommand{\sectionmark}[1]{
\markboth{Abschnitt \thesection{}: #1}{\thesection{} #1}
}
\renewcommand{\subsectionmark}[1]{
\markright{\thesubsection{} #1}
}

% DOKUMENTINFORMATIONEN
\title{P402 \\ Quantelung von Energie}

\author{Christopher Deutsch\footnote{christopher.deutsch@uni-bonn.de} \and Christian Bespin\footnote{christian.bespin@uni-bonn.de}}

\date{\today}

\begin{document}

\begin{titlepage}

\maketitle

% DURCHFÜHRUNGSDATUM UND ASSISTENT
\begin{center}
\begin{tabular}{l r}
Durchführung: & 1./2. Dezember 2014 \\
Gruppe: & $\alpha$ 2 \\
Assistent: & Dennis Proft
\end{tabular}
\end{center}

% ZUSAMMENFASSUNG
\begin{abstract}
\noindent

\end{abstract}

\end{titlepage}

% INHALTSVERZEICHNIS
\tableofcontents
% Neue Seite nach TOC
\newpage

% INHALT VERSUCHSPROTOKOLL

\section{Grundlagen / Theorie}

\subsection{Photoelektrischer Effekt}
Photoeffekt umfasst drei verschiedene Arten der Wechselwirkung von Licht mit Materie.
Betrachtet wird der äußere photoelektrische Effekt, das heißt das Herauslösen von Elektronen aus Metalloberflächen (oder Halbleiter) durch Photonen.
Erstmalige Deutung ("Licht" gibt Energie in Paketen welche Photonen genannt werden ab) von Einstein 1905 (Nobelpreis).
Energie eines Photons:
\begin{align}
	E_\gamma = h \cdot \nu
\end{align}




\subsection{Photozelle}

\subsubsection{Aufbau und Wirkung}
Photozelle: eine Form von Elektronenröhre
Zwei Elektroden im evakuierten Glaskolben (freie Weglänge $\Lambda$).
Elektroden: Photokathode (hierauf treffen die Photonen) und Anode (die hier aufgefangenen Elektronen verursachen den Photostrom).

Messung: Intensitäten oberhalb einer Grenzfrequenz (Austrittsarbeit);

Betriebsmodi: Saugspannung (Anode + Kathode -) oder Gegenspannung (Anode - Kathode +).


Elektronen in der Photokathode absorbieren Photonen der Energie $E_\gamma$.
Ist die Energie der Elektronen anschließend hoch genug (größer als die Austrittsarbeit), so können die Elektronen aus dem Metall austreten.
Die Energiebilanz des gelösten Elektrons (ohne Messkreis):
\begin{align}
E_e = E_\gamma - W_\mathrm{A} = h \cdot \nu - W_\mathrm{A}
\end{align}

\subsubsection{Photostromverlauf}

\begin{align}
	I_\mathrm{Photo} \propto (E_\mathrm{kin} - e \cdot U_0)^2
\end{align}

\subsubsection{Austrittsarbeit und Kontaktpotential}
Kontaktpotentialdifferenz zweier Metall ist gleich der Differenz ihrer Austrittsarbeiten (Der Photoeffekt Klaus/Herrmann)
Austrittsarbeit ist Abstand Fermi-Niveau - Kontinuum.
Bringt man zwei unterschiedliche Leiter mit verschiedenen Austrittsarbeiten in Kontakt, so entsteht ein Ladungsfluss der die Fermi-Niveaus beider Leiter ausgleicht.

\begin{align}
	U_\mathrm{K} = \frac{\phi_1 - \phi_2}{e}
\end{align}



\clearpage

% BIBLIOGRAPHIE

% Maximale Anzahl der Einträge in Klammer
% Zitieren mit \cite{lamport94}
\begin{thebibliography}{9}

\bibitem{hecht}
	Eugene Hecht,
	\emph{Optik}.
	Oldenbourg,
	5. Auflage
	
\bibitem{siegmann}
	Anthony E. Siegmann,
	\emph{Lasers}.
	University Science Books,
	1986
	
\bibitem{demtroeder3}
	Wolfgang Demtröder,
	\emph{Experimentalphysik 3}.
	Springer Verlag,
	3. Auflage

\bibitem{np_richardson}
 Nobelprize.org,
 \emph{"The Nobel Prize in Physics 1928"}.
 Nobel Media AB 2014. Web. 15\\
 (\url{http://www.nobelprize.org/nobel_prizes/physics/laureates/1928/richardson-lecture.pdf} abgerufen am 15.11.2014)
 
\bibitem{haken_wolf}
	Hermann Haken, Hans Christoph Wolf,
	\emph{Atom- und Quantenphysik}.
	Springer Verlag,
	7. Auflage
 
\end{thebibliography}

% APPENDIX
\begin{appendix}

\end{appendix}

\end{document}
