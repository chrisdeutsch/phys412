% PACKAGES AND DOCUMENT CONFIGURATIONS
\documentclass[10pt, a4paper]{article}

% Encoding für Umlaute
\usepackage[utf8]{inputenc}

% Silbentrennung
\usepackage[ngerman]{babel}

% erweiterte Matheumgebungen
\usepackage{amsmath}

%
\usepackage{amsfonts}

%
\usepackage{amssymb}

% Einheiten setzen z.B. \SI{10}{\kilo\gram\meter\per\second\squared}
\usepackage{siunitx}

\usepackage[left=2cm,right=2cm,top=2cm,bottom=2cm]{geometry}

% Bilder einfügen
\usepackage{graphicx}

% bessere Tabellenlayouts
\usepackage{booktabs}

%Textumflossene Grafiken, z.B. \begin{wrapfigure}[14]{R}[1pt]{0.5\textwidth}
\usepackage{wrapfig}

% DOCUMENT INFORMATION

\title{P401 \\ Elektronische Übergänge in Atomen}

\author{Christopher Deutsch \and Christian Bespin}

\date{\today}

\begin{document}

\maketitle

\begin{center}
\begin{tabular}{l r}
Durchführungsdatum: & 20. Oktober 2014 \\
Gruppe: & $\alpha$ 2 \\
Assistent: & Hans Wurst
\end{tabular}
\end{center}

% ABSTRACT

\begin{abstract}
% Abstract text
\end{abstract}

% Inhaltsverzeichnis %
\setcounter{tocdepth}{2} % zeigt subsubsections (entspricht der 2) nicht im Inhaltsverzeichnis an %
\tableofcontents


% Seitenumbruch %
\newpage

\section{Grundlagen / Theorie}

\section{Versuchsdurchführung}

\subsection{Versuchsbeschreibung}

\subsection{Aufbau}

\subsubsection{Hinweis zu einem der Geräte oder whatever}

\section{Messdaten}

\section{Auswertung}

\section{Diskussion}

\section{Zusammenfassung}

% BIBLIOGRAPHY


\end{document}
