% PAKETE UND DOKUMENTKONFIGURATION
\documentclass[10pt, a4paper]{article}

% Encoding für Umlaute
\usepackage[utf8]{inputenc}

% Silbentrennung
\usepackage[ngerman]{babel}

% erweiterte Matheumgebungen
\usepackage{amsmath}

%
\usepackage{amsfonts}

%
\usepackage{amssymb}

% Einheiten setzen z.B. \SI{10}{\kilo\gram\meter\per\second\squared}
\usepackage{siunitx}

% Randbreiten
\usepackage[left=2cm,right=2cm,top=2cm,bottom=2cm]{geometry}

% Bilder einfügen
\usepackage{graphicx}

% Tiefe des Inhaltsverzeichnisses (Level: 1 sections, 2 subsections,
% 3 subsubsections)
\setcounter{tocdepth}{2}

% DOKUMENTINFORMATIONEN
\title{P422 \\ Rastertunnelmikroskopie}

\author{Christopher Deutsch \and Christian Bespin}

\date{\today}

\begin{document}
  
\maketitle

% DURCHFÜHRUNGSDATUM UND ASSISTENT
\begin{center}
\begin{tabular}{l r}
Durchführung: & 20./21. Oktober 2014 \\
Gruppe: & 2 $\alpha$ \\
Assistent: & Peter Klassen
\end{tabular}
\end{center}

% ZUSAMMENFASSUNG
\begin{abstract}
% Text
\end{abstract}

% INHALTSVERZEICHNIS
\tableofcontents
% Neue Seite nach TOC
\newpage

% INHALT VERSUCHSPROTOKOLL
\section{Grundlagen}

\subsection{Tunneleffekt und Tunnelstrom}

\subsection{Funktionsweise des Rastertunnelmikroskops}
\subsubsection{Aufbau}
\subsubsection{Piezoeffekt}
\subsubsection{Betriebsmodi}
\subsubsection{Regelkreis}
\subsubsection{Auflösungsvermögen}

\subsection{Spitzenherstellung}
\subsubsection{Reißen von Platin-Iridium Spitzen}
\subsubsection{Ätzen von Wolfram Spitzen}

\subsection{Kristallstruktur von Graphit}

\subsection{Verwandte Rastermethoden}

\section{Versuchsdurchführung}

\section{Messdaten}

\section{Auswertung}

\section{Diskussion}

\section{Zusammenfassung}

% BIBLIOGRAPHIE

% Maximale Anzahl der Einträge in Klammer
% Zitieren mit \cite{lamport94}
\begin{thebibliography}{9}

% Beispiel
\bibitem{binning}
  G. Binning et al., Phys. Rev. Lett. 49, 57 (1982),
  \emph{Surface Studies by Scanning Tunneling Microscopy}.
\end{thebibliography}

\end{document}
