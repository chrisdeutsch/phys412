% PAKETE UND DOKUMENTKONFIGURATION
\documentclass[11pt, a4paper]{article}

% Encoding für Umlaute
\usepackage[utf8]{inputenc}
\usepackage[T1]{fontenc}

% Silbentrennung
\usepackage[ngerman]{babel}

% erweiterte Matheumgebungen und Formelnummer mit Sectionnummer
\usepackage{amsmath}
\numberwithin{equation}{section}

% Braket Notation
\usepackage{braket}

% zusätzliche mathematische Schriftarten
\usepackage{amsfonts}

% verschiedene mathematische Symbole
\usepackage{amssymb}

% Einheiten setzen z.B. \SI{10}{\kilo\gram\meter\per\second\squared}
% Fehler: \SI{10 +- 0,2e-4}{\metre}
\usepackage{siunitx}
\sisetup{
  output-decimal-marker={,},
  separate-uncertainty
}

% Randbreiten
\usepackage[left=3.5cm,right=3.5cm,top=3cm,bottom=3cm,twoside]{geometry}

% Bilder einfügen
\usepackage{graphicx}

% Verweise innerhalb des Dokuments
\usepackage{hyperref}
\hypersetup{
	colorlinks = true,
	allcolors = {black}
}

% bessere Tabellenlayouts
\usepackage{booktabs}
\usepackage{multirow}

% Seitenlayout (Kopfzeile)
\usepackage{fancyhdr}

% Float Barriers
\usepackage{placeins}

% Pakete für gedrehte Subfigures
\usepackage{caption}
\usepackage{subcaption}
\usepackage{rotating}

% Caption-Setup
\captionsetup{font={small}}
\renewcommand{\thefigure}{\thesection.\arabic{figure}}
\renewcommand{\thesubfigure}{\alph{subfigure}}
\renewcommand{\thetable}{\thesection.\arabic{table}}
\renewcommand{\thesubtable}{\alph{subtable}}

% Manuelle Silbentrennung
\hyphenation{}

% Tiefe des Inhaltsverzeichnisses (Level: 1 sections, 2 subsections,
% 3 subsubsections)
\setcounter{tocdepth}{3}

% FANCYHDR SETUP
\pagestyle{fancy}
\fancyhead[EL,OR]{\thepage}
\fancyhead[ER]{\leftmark}
\fancyhead[OL]{\rightmark}

\renewcommand{\sectionmark}[1]{
\markboth{\thesection{} #1}{\thesection{} #1}
}
\renewcommand{\subsectionmark}[1]{
\markright{\thesubsection{} #1}
}

% DOKUMENTINFORMATIONEN
\title{P442 \\ Laser}

\author{Christopher Deutsch\footnote{christopher.deutsch@uni-bonn.de} \and Christian Bespin\footnote{christian.bespin@uni-bonn.de}}

\date{\today}

\begin{document}

\begin{titlepage}

\maketitle

% DURCHFÜHRUNGSDATUM UND TUTOR
\begin{center}
\begin{tabular}{l r}
Durchführung: & 15./16. Dezember 2014 \\
Gruppe: & $\alpha$ 2 \\
Tutor: & Tobias Macha
\end{tabular}
\end{center}

% ZUSAMMENFASSUNG
\begin{abstract}
\noindent

\end{abstract}

\end{titlepage}

% INHALTSVERZEICHNIS
\tableofcontents
% Neue Seite nach TOC
\newpage

% INHALT VERSUCHSPROTOKOLL

\section{Einführung}

\section{Theorie}

\section{Durchführung und Auswertung}

\subsection{Aufbau des Helium-Neon Experimentierlasers und Charakterisierung der Intensität}

\subsection{Bestimmung der Wellenlänge des Lasers mit einem Reflexionsgitter}

\subsection{Untersuchung der Polarisation des Lasers}

Zur Untersuchung der Polarisation des hier verwendeten Experimentierlasers wird der Intensitätsverlauf an einer Photodiode in Abhängigkeit des Winkels eines davor stehenden Linearpolarisators betrachtet.
Dieser ist in der Lage, einfallendes Licht in eine vorgegebene Richtung linear zu polarisieren.
Für den Fall, dass das einfallende Licht bereits linear polarisiert ist (wie für den hier verwendeten Laser angenommen werden kann), gilt für die hinter dem Polarisator registrierte Intensität \cite{gerthsen}:
\begin{align}
I=I_0\cdot\cos^2\alpha
\end{align}
Der Winkel am Polarisationsfilter wurde in Schritten von \SI{10}{\degree} gedreht und jeweils die Intensität an der Photodiode in \si{\milli\volt} notiert.
In Tabelle \ref{tab:intensitaet} sind die Messwerte festgehalten worden.
\begin{table}
	\centering
	\begin{tabular}{SSSS}
	\toprule
	{Drehwinkel $\varphi / \si{\degree}$} & {Fehler $\Delta\varphi / \si{\degree}$} & {Intensität $I / \si{\milli\volt}$} & {Fehler $\Delta I / \si{\milli\volt}$} \\
	\midrule
	0   & 1 & 39.50 & 0.3 \\
	10  & 1 & 40.00 & 0.3 \\
	20  & 1 & 38.00 & 0.3 \\
	30  & 1 & 33.80 & 0.3 \\
	40  & 1 & 27.90 & 0.3 \\
	50  & 1 & 20.20 & 0.3 \\
	60  & 1 & 13.40 & 0.3 \\
	70  & 1 & 6.45  & 0.3 \\
	80  & 1 & 2.62  & 0.3 \\
	90  & 1 & 0.60  & 0.3 \\
	100 & 1 & 1.10  & 0.3 \\
	110 & 1 & 3.50  & 0.3 \\
	120 & 1 & 8.60  & 0.3 \\
	130 & 1 & 14.70 & 0.3 \\
	140 & 1 & 21.20 & 0.3 \\
	150 & 1 & 27.70 & 0.3 \\
	160 & 1 & 33.40 & 0.3 \\
	170 & 1 & 37.00 & 0.3 \\
	180 & 1 & 38.50 & 0.3 \\
	190 & 1 & 37.50 & 0.3 \\
	200 & 1 & 34.70 & 0.3 \\
	210 & 1 & 29.60 & 0.3 \\
	220 & 1 & 22.90 & 0.3 \\
	230 & 1 & 16.30 & 0.3 \\
	240 & 1 & 9.50  & 0.3 \\
	250 & 1 & 4.10  & 0.3 \\
	260 & 1 & 1.30  & 0.3 \\
	270 & 1 & 0.30  & 0.3 \\
	280 & 1 & 2.00  & 0.3 \\
	290 & 1 & 5.60  & 0.3 \\
	300 & 1 & 10.90 & 0.3 \\
	310 & 1 & 17.60 & 0.3 \\
	320 & 1 & 24.40 & 0.3 \\
	330 & 1 & 30.50 & 0.3 \\
	340 & 1 & 35.60 & 0.3 \\
	350 & 1 & 38.70 & 0.3 \\
	360 & 1 & 39.50 & 0.3 \\
	\bottomrule
\end{tabular}
	\caption{Messwerte der Intensität an der Photodiode in Abhängigkeit des Winkels am Linearpolarisator}
	\label{tab:intensitaet}
\end{table}

\begin{figure}
	\centering
	\begin{tabular}{SSSS}
	\toprule
	{Drehwinkel $\varphi / \si{\degree}$} & {Fehler $\Delta\varphi / \si{\degree}$} & {Intensität $I / \si{\milli\volt}$} & {Fehler $\Delta I / \si{\milli\volt}$} \\
	\midrule
	0   & 1 & 39.50 & 0.3 \\
	10  & 1 & 40.00 & 0.3 \\
	20  & 1 & 38.00 & 0.3 \\
	30  & 1 & 33.80 & 0.3 \\
	40  & 1 & 27.90 & 0.3 \\
	50  & 1 & 20.20 & 0.3 \\
	60  & 1 & 13.40 & 0.3 \\
	70  & 1 & 6.45  & 0.3 \\
	80  & 1 & 2.62  & 0.3 \\
	90  & 1 & 0.60  & 0.3 \\
	100 & 1 & 1.10  & 0.3 \\
	110 & 1 & 3.50  & 0.3 \\
	120 & 1 & 8.60  & 0.3 \\
	130 & 1 & 14.70 & 0.3 \\
	140 & 1 & 21.20 & 0.3 \\
	150 & 1 & 27.70 & 0.3 \\
	160 & 1 & 33.40 & 0.3 \\
	170 & 1 & 37.00 & 0.3 \\
	180 & 1 & 38.50 & 0.3 \\
	190 & 1 & 37.50 & 0.3 \\
	200 & 1 & 34.70 & 0.3 \\
	210 & 1 & 29.60 & 0.3 \\
	220 & 1 & 22.90 & 0.3 \\
	230 & 1 & 16.30 & 0.3 \\
	240 & 1 & 9.50  & 0.3 \\
	250 & 1 & 4.10  & 0.3 \\
	260 & 1 & 1.30  & 0.3 \\
	270 & 1 & 0.30  & 0.3 \\
	280 & 1 & 2.00  & 0.3 \\
	290 & 1 & 5.60  & 0.3 \\
	300 & 1 & 10.90 & 0.3 \\
	310 & 1 & 17.60 & 0.3 \\
	320 & 1 & 24.40 & 0.3 \\
	330 & 1 & 30.50 & 0.3 \\
	340 & 1 & 35.60 & 0.3 \\
	350 & 1 & 38.70 & 0.3 \\
	360 & 1 & 39.50 & 0.3 \\
	\bottomrule
\end{tabular}
	\caption{Anpassung einer zu erwartenden Kurve an die Messwerte}
	\label{fig:malus}
\end{figure}

\subsection{Messung des Strahlprofils und des Stabilitätsgebiets des Lasers}

\subsection{Aufbau der optischen Diode}

\subsection{Optischer Spektrumanalysator}

\subsection{Präzise Messung des Modenabstandes mittels einer optischen Schwebung}

\section{Fazit}


% BIBLIOGRAPHIE
\vspace{\fill}
% Maximale Anzahl der Einträge in Klammer
% Zitieren mit \cite{lamport94}
\begin{thebibliography}{9}
\bibitem{gerthsen}
	Dieter Meschede,
	\emph{Gerthsen Physik}.
	Springer Verlag,
	23. Auflage

\bibitem{crc}
	David R. Lide (ed),
	\emph{CRC Handbook of Chemistry and Physics},
	84th Edition. CRC Press. Boca Raton, Florida, 2003;
	Section 12: Properties of Solids --
	\emph{Electron Work Function of the Elements}
 
\end{thebibliography}

\clearpage

% APPENDIX
\begin{appendix}
\section{Anhang}


\end{appendix}

\end{document}
