% PAKETE UND DOKUMENTKONFIGURATION
\documentclass[11pt, a4paper]{article}

% Encoding für Umlaute
\usepackage[utf8]{inputenc}
\usepackage[T1]{fontenc}

% Silbentrennung
\usepackage[ngerman]{babel}

% erweiterte Matheumgebungen
\usepackage{amsmath}

% Braket Notation
\usepackage{braket}

% zusätzliche mathematische Schriftarten
\usepackage{amsfonts}

% verschiedene mathematische Symbole
\usepackage{amssymb}

% Einheiten setzen z.B. \SI{10}{\kilo\gram\meter\per\second\squared}
% Fehler: \SI{10 +- 0,2e-4}{\metre}
\usepackage{siunitx}
\sisetup{
  output-decimal-marker={,},
  separate-uncertainty
}

% Randbreiten
\usepackage[left=3.5cm,right=3.5cm,top=3cm,bottom=3cm,twoside]{geometry}

% Bilder einfügen
\usepackage{graphicx}

% Verweise innerhalb des Dokuments
\usepackage{hyperref}
\hypersetup{
	colorlinks = true,
	allcolors = {black}
}

% bessere Tabellenlayouts
\usepackage{booktabs}
\usepackage{multirow}

% Seitenlayout (Kopfzeile)
\usepackage{fancyhdr}

% Float Barriers
\usepackage{placeins}

% Pakete für gedrehte Subfigures
\usepackage{caption}
\usepackage{subcaption}
\usepackage{rotating}

% Caption-Setup
\captionsetup{font={small}}
\renewcommand{\thefigure}{\thesection.\arabic{figure}}
\renewcommand{\thesubfigure}{\alph{subfigure}}
\renewcommand{\thetable}{\thesection.\arabic{table}}
\renewcommand{\thesubtable}{\alph{subtable}}

% Manuelle Silbentrennung
\hyphenation{Halb-werts-brei-te Fa-bry-Pé-rot-E-ta-lon Zee-man-E-ffekt Cad-mi-um-Lam-pe Kon-den-sor-lin-se}

% Tiefe des Inhaltsverzeichnisses (Level: 1 sections, 2 subsections,
% 3 subsubsections)
\setcounter{tocdepth}{3}

% FANCYHDR SETUP
\pagestyle{fancy}
\fancyhead[EL,OR]{\thepage}
\fancyhead[ER]{\leftmark}
\fancyhead[OL]{\rightmark}

\renewcommand{\sectionmark}[1]{
\markboth{Abschnitt \thesection{}: #1}{\thesection{} #1}
}
\renewcommand{\subsectionmark}[1]{
\markright{\thesubsection{} #1}
}

% DOKUMENTINFORMATIONEN
\title{P402 \\ Quantelung von Energie}

\author{Christopher Deutsch\footnote{christopher.deutsch@uni-bonn.de} \and Christian Bespin\footnote{christian.bespin@uni-bonn.de}}

\date{\today}

\begin{document}

\begin{titlepage}

\maketitle

% DURCHFÜHRUNGSDATUM UND ASSISTENT
\begin{center}
\begin{tabular}{l r}
Durchführung: & 1./2. Dezember 2014 \\
Gruppe: & $\alpha$ 2 \\
Assistent: & Dennis Proft
\end{tabular}
\end{center}

% ZUSAMMENFASSUNG
\begin{abstract}
\noindent

\end{abstract}

\end{titlepage}

% INHALTSVERZEICHNIS
\tableofcontents
% Neue Seite nach TOC
\newpage

% INHALT VERSUCHSPROTOKOLL

\section{Grundlagen / Theorie}

\subsection{Photoelektrischer Effekt}
Photoeffekt umfasst drei verschiedene Arten der Wechselwirkung von Licht mit Materie.
Betrachtet wird der äußere photoelektrische Effekt, das heißt das Herauslösen von Elektronen aus Metalloberflächen (oder Halbleiter) durch Photonen.
Erstmalige Deutung ("Licht" gibt Energie in Paketen welche Photonen genannt werden ab) von Einstein 1905 (Nobelpreis).
Energie eines Photons:
\begin{align}
	E_\gamma = h \cdot \nu
\end{align}




\subsection{Photozelle}

\subsubsection{Aufbau und Wirkung}
Photozelle: eine Form von Elektronenröhre
Zwei Elektroden im evakuierten Glaskolben (freie Weglänge $\Lambda$).
Elektroden: Photokathode (hierauf treffen die Photonen) und Anode (die hier aufgefangenen Elektronen verursachen den Photostrom).

Messung: Intensitäten oberhalb einer Grenzfrequenz (Austrittsarbeit);

Betriebsmodi: Saugspannung (Anode + Kathode -) oder Gegenspannung (Anode - Kathode +).


Elektronen in der Photokathode absorbieren Photonen der Energie $E_\gamma$.
Ist die Energie der Elektronen anschließend hoch genug (größer als die Austrittsarbeit), so können die Elektronen aus dem Metall austreten.
Die Energiebilanz des gelösten Elektrons (ohne Messkreis):
\begin{align}
E_e = E_\gamma - W_\mathrm{A} = h \cdot \nu - W_\mathrm{A}
\end{align}

\subsubsection{Photostromverlauf}

\begin{align}
	I_\mathrm{Photo} \propto (E_\mathrm{kin} - e \cdot U_0)^2
\end{align}

\subsubsection{Austrittsarbeit und Kontaktpotential}
Kontaktpotentialdifferenz zweier Metall ist gleich der Differenz ihrer Austrittsarbeiten (Der Photoeffekt Klaus/Herrmann)
Austrittsarbeit ist Abstand Fermi-Niveau - Kontinuum.
Bringt man zwei unterschiedliche Leiter mit verschiedenen Austrittsarbeiten in Kontakt, so entsteht ein Ladungsfluss der die Fermi-Niveaus beider Leiter ausgleicht.

\begin{align}
	U_\mathrm{K} = \frac{\phi_1 - \phi_2}{e}
\end{align}

\section{Bestimmung des Planckschen Wirkungsquantums}


\section{Balmer-Serie}

\subsection{Auswertung}
\subsubsection{Bestimmung der Gitterkonstanten}
\begin{figure}[h]
	\centering
	\includegraphics[width=0.65\textwidth]{./figures/winkelverhaeltniss.pdf}
	\caption{Winkelverhältnis zwischen Gitter und optischer Bank}
	\label{fig:winkelverhaeltnis}
\end{figure}
Aus Abbildung \ref{fig:winkelverhaeltnis} folgt der Zusammenhang zwischen gemessenen Winkeln und Ein-/Ausfallswinkel des Strahls auf das Gitter: 
\begin{align*}
	\alpha &= \omega_\text{G} \\
	\beta &= \omega_\text{G} + \omega_\text{B} - \SI{180}{\degree}
\end{align*}
Korrektur des Ausfallswinkels aufgrund der Messung von $d$ mit dem Objektiv:
\begin{align*}
\Delta \beta = -\frac{d}{f} \text{ in KWN}
\end{align*}
$f$ Brennweite des Fernrohrobjektivs ($\SI{300}{\milli\metre}$).
So folgt der tatsächliche Ausfallswinkel gemäß:
\begin{align*}
	\beta^\prime = \beta + \Delta \beta
\end{align*}
Zur Bestimmung der Gitterkonstanten tragen wir $\sin(\alpha) + \sin(\beta)$ gegen $\lambda$ auf, um mit der Gittergleichung
\begin{align*}
	\sin(\alpha) + \sin(\beta) = \frac{1}{g} \cdot \lambda
\end{align*}
zu erhalten.
Wir wählen diese Form der Gleichung, da bei einer Anpassung mit üblichen Least-Squares-Verfahren nur Fehler in $y$-Richtung beachtet werden.
Da der Fehler der Wellenlänge $\lambda$ gegen den der Sinus-Summe vernachlässigbar ist, wurde diese Form gewählt (\#machmalrichtig).
Wir erhalten für die Anpassung:
\begin{align}
	\sin(\alpha) + \sin(\beta) = \SI{2439.6 +- 2.1}{\per\milli\metre} \cdot \lambda \text{,}
\end{align}
was einer Gitterkonstanten in reziproken Einheiten von:
\begin{align*}
	\frac{1}{g} = (\num{2439.6 +- 2.1}) \, \frac{\mathrm{Linien}}{\si{\milli\metre}}
\end{align*}
entspricht.
Nach Kehrwertbildung ergibt sich:
\begin{align*}
	g = \SI{409.90 +- 0.36}{\nano\metre} \text{.}
\end{align*}

\subsection{Isotopieaufspaltung mit Okular}

\subsubsection{Durchführung}
\label{sec:balmer_okular}

\subsection{Isotopieaufspaltung mit CCD Kamera}

\subsubsection{Durchführung}

Wir nutzen den Versuchsaufbau wie in \ref{sec:balmer_okular} beschrieben, ersetzen das Okular jedoch durch eine CCD Kamera, die wir mit dem Computer verbinden.
Im Programm VideoCom wird die Brennweite der auf die CCD-Zeile abbildenden Linse (\SI{300}{\milli\meter}) eingetragen, so dass jedem der 2048 ein Winkel zugeordnet werden kann, den wir im folgenden mit $\alpha$ bezeichnen.
Um die Linien aufzufinden nutzen wir das Okular auf einem zusätzlichen Reiter (um die vertikale Ausrichtung bei jedem Durchgang nicht anpassen zu müssen) und drehen das Gitter entsprechend, bis die Linie möglichst zentral liegt.
Dann setzen wir den Reiter mit der Kamera anstelle des Okulars auf die optische Bank und betrachten das Intensitätsprofil auf dem Bildschirm, um die Abbildungslinse so zu verschieben, bis wir scharf auf die Kamera abbilden.
Wir vergrößern nun den Bildausschnitt, um den Bereich der Linie besser betrachten zu können und verändern die Spaltbreite so, dass wir eine möglichst hohe Intensität und gleichzeitig eine gute Aufspaltung messen.
Da dies nicht immer eindeutig gelang haben wir zu jeder Linie mehrere Messungen durchgeführt und zur Auswertung jeweils die Messung gewählt, bei der die Aufspaltung gut zu erkennen war.
Wegen der teilweise hohen Schwankungen der Intensität haben wir die Messwerte im Programm über eine gewisse Zeit mitteln lassen, bis mit dem Auge beinahe keine Schwankungen mehr zu erkennen waren.

\clearpage

\begin{table}[h]
	\centering
	\begin{tabular}{SSSSSSSS}
	\toprule
	{$\dfrac{\lambda}{\si{\nano\metre}}$} & {$\dfrac{m}{\si{\nano\ampere\tothe{1/2}\per\volt}}$} & {$\dfrac{\Delta m}{\si{\nano\ampere\tothe{1/2}\per\volt}}$} & {$\dfrac{b}{\si{\nano\ampere\tothe{1/2}}}$} & {$\dfrac{\Delta b}{\si{\nano\ampere\tothe{1/2}}}$} & {$\chi_\mathrm{red.}^2$} \\
	\midrule
	\multirow{2}{*}{365} & -2.557 & 0.026 & 4.353 & 0.026 & 1.10 \\
	 & -2.539 & 0.023 & 4.345 & 0.021 & 0.75 \\
	 \midrule
	\multirow{2}{*}{405} & -1.959 & 0.033 & 2.777 & 0.028 & 3.57 \\
	 & -1.957 & 0.024 & 2.772 & 0.020  & 1.81 \\
	 \midrule
	\multirow{2}{*}{436} & -3.000 & 0.032 & 3.655 & 0.023 & 1.71 \\
	 & -3.003 & 0.046 & 3.661 & 0.033 & 3.75 \\
	 \midrule
	\multirow{2}{*}{546} & -4.542 & 0.012 & 3.203 & 0.005 & 0.05 \\
	 & -4.464 & 0.043 & 3.189 & 0.015 & 0.64 \\
	 \midrule
	\multirow{2}{*}{578} & -3.771 & 0.036 & 2.230 & 0.011 & 0.89 \\
	 & -3.860 & 0.017 & 2.285 & 0.006 & 0.19 \\
	\bottomrule
\end{tabular}

	\caption{Fit Resulatate}
	\label{tab:photoeffekt_fit_results}
\end{table}
\begin{table}[h]
	\centering
	\begin{tabular}{SSS}
	\toprule
	{$\nu$ / \si{\tera\hertz}} & {$U_0$ / \si{\volt}} & {$\Delta U_0$ / \si{\volt}} \\
	\midrule
	821.35 & 1.708 & 0.013 \\
	740.23 & 1.417 & 0.016 \\
	687.60 & 1.219 & 0.013 \\
	549.07 & 0.706 & 0.002 \\
	518.67 & 0.592 & 0.003 \\
	\bottomrule
\end{tabular}

	\caption{Linearisierung zur Bestimmung von h}
	\label{tab:photoeffekt_planck_fitdaten}
\end{table}

\begin{table}[h]
	\centering
	\begin{tabular}{SSS}
	\toprule
	{$U$ / \si{\volt}} & {$I-I_0$ / \si{\nano\ampere}} & {$\Delta (I-I_0)$ / \si{\nano\ampere}} \\
	\midrule
0.001 & 18.927 & 0.381 \\
0.201 & 14.727 & 0.297 \\
0.400 & 11.327 & 0.229 \\
0.583 & 8.167  & 0.165 \\
0.790 & 5.347  & 0.109 \\
0.992 & 3.287  & 0.068 \\
1.185 & 1.817  & 0.038 \\
1.402 & 0.817  & 0.018 \\
1.595 & 0.337  & 0.009 \\
1.810 & 0.067  & 0.003 \\
2.016 & 0.027  & 0.003 \\
	\bottomrule
\end{tabular}

	\caption{Messdaten 365 1}
	\label{tab:365nm_1}
\end{table}

\clearpage

\begin{table}
\centering
\begin{subtable}{0.5\textwidth}
\centering

<tabular-environment>

\caption{<subcaption>}
\end{subtable}%
\begin{subtable}{0.5\textwidth}
\centering

<tabular-environment>

\caption{<subcaption>}
\end{subtable}

\caption{<main caption>}
\end{table}

\section{Grafikstorage}

\begin{figure}
\centering
\input{./plots/gitterkonstante.tex}
\caption{Gitterkonstante}
\label{fig:gitterkonstante}
\end{figure}

\begin{figure}
\centering
\input{./plots/aufspaltung/rot2.tex}
\caption{Isotopieaufspaltung rot}
\label{fig:aufspaltung_rot}
\end{figure}

\begin{figure}
\centering
\input{./plots/aufspaltung/tuerkis1.tex}
\caption{Isotopieaufspaltung tuerkis}
\label{fig:aufspaltung_tuerkis}
\end{figure}


\begin{figure}
\centering
\input{./plots/aufspaltung/blau1.tex}
\caption{Isotopieaufspaltung blau}
\label{fig:aufspaltung_blau}
\end{figure}

\clearpage

\begin{figure}
	\centering
	\input{./plots/photo/kennlinien_365nm.tex}
	\caption{Kennlinien 365nm}
	\label{fig:kennlinien_365nm}
\end{figure}

\begin{figure}
	\centering
	\input{./plots/photo/kennlinien_405nm.tex}
	\caption{Kennlinien 405nm}
	\label{fig:kennlinien_405nm}
\end{figure}

\begin{figure}
	\centering
	\input{./plots/photo/kennlinien_436nm.tex}
	\caption{Kennlinien 436nm}
	\label{fig:kennlinien_436nm}
\end{figure}

\begin{figure}
	\centering
	\input{./plots/photo/kennlinien_546nm.tex}
	\caption{Kennlinien 546nm}
	\label{fig:kennlinien_546nm}
\end{figure}

\begin{figure}
	\centering
	\input{./plots/photo/kennlinien_578nm.tex}
	\caption{Kennlinien 578nm}
	\label{fig:kennlinien_578nm}
\end{figure}

\begin{figure}
	\centering
	\input{./plots/planck_quantum_linearisierung.tex}
	\caption{Linearisierung zur Bestimmung von $h$}
	\label{fig:lin_h}
\end{figure}

% BIBLIOGRAPHIE

% Maximale Anzahl der Einträge in Klammer
% Zitieren mit \cite{lamport94}
\begin{thebibliography}{9}

\bibitem{hecht}
	Eugene Hecht,
	\emph{Optik}.
	Oldenbourg,
	5. Auflage
	
\bibitem{siegmann}
	Anthony E. Siegmann,
	\emph{Lasers}.
	University Science Books,
	1986
	
\bibitem{demtroeder3}
	Wolfgang Demtröder,
	\emph{Experimentalphysik 3}.
	Springer Verlag,
	3. Auflage

\bibitem{np_richardson}
 Nobelprize.org,
 \emph{"The Nobel Prize in Physics 1928"}.
 Nobel Media AB 2014. Web. 15\\
 (\url{http://www.nobelprize.org/nobel_prizes/physics/laureates/1928/richardson-lecture.pdf} abgerufen am 15.11.2014)
 
\bibitem{haken_wolf}
	Hermann Haken, Hans Christoph Wolf,
	\emph{Atom- und Quantenphysik}.
	Springer Verlag,
	7. Auflage
 
\end{thebibliography}

% APPENDIX
\begin{appendix}

\end{appendix}

\end{document}
